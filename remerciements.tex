% TODO

Enfin la dernière page à écrire! Combien de fois ai-je entendu que "les
remerciements, c'est le plus facile". Et pourtant, je suis partagé entre la
satisfaction d'arriver au bout de ce travail, et la crainte d'oublier une des
nombreuses personnes qui m'ont aidé à y arriver.

% Directeurs

Je tiens à commencer par remercier mes encadrants de thèse, Emmanuel Chailloux
et Sarah Zennou. Sans leurs conseils pertienents et leurs nombreuses relectures,
je n'aurais pas arriver au bout de ce travail.

% Rapporteurs
% Autres membres

Merci également à Sandrine Blazy et à Pierre Jouvelot d'avoir accepté de
rapporter mon manuscrit, et pour leurs remarques qui ont permis d'améliorer sa
qualité. Je veux également remercier les autres membres du jury, Gilles Muller
et Vincent Simonet, ainsi qu'Olivier Levillain qui y a sa place officieusement.

% Ecole doc

Si cette aventure a pu être menée à bien, c'est également grâce au travail
réalisé par l'équipe de l'école doctorale, et en particulier à Marylin Galopin
et Christian Queinnec qui ont toujours su m'aider dans cette véritable quête
administrative qu'est le doctorat. Je souhaite d'ailleurs le meilleur à Bertrand
Granado pour reprendre les rênes de l'EDITE.

% Equipe enseignante

Le pendant du travail de recherche est traditionnellement celui de
l'enseignement ; dans mon cas l'expérience pédagogique a été un peu courte, mais
elle a été mais très agréable grâce aux équipes enseignantes des cours de
Programmation et Données Génériques (LI220, "l'UE des chefs") et de Techniques
Évenementielles et Réactives (LI357, "l'UE Magnum").

% Financement

Ce projet a commencé chez Airbus Group Innovations (alors EADS Innovation
Works), alors que je n'étais qu'un étudiant ingénieur intéressé par la
compilation. Merci à Wenceslas Godard et Charles Hymans de m'avoir offert ce
stage, puis de l'étendre à ce projet de thèse. La suite a été plus quelque peu
nébuleuse, et je remercie chaudement Axel Tillequin et toute l'équipe SE/IT pour
la confiance qu'il ont pu m'offrir en m'accueillant dans un cadre exceptionnel
pour un jeune chercheur. Merci une fois de plus à Sarah pour avoir accepté de
m'encadrer lors de cette thèse.

% TODO gens d'eads

% APR

Mes remerciements vont également à une autre équipe qui m'a accueilli pendant
ces années: l'équipe APR, et en particulier sa directrice Michèle Soria.
Les différentes thématiques de recherche abordées dans le couloir on permis
d'étendre mes horizons de jeune chercheur. Je remercie également le personnel
administratif qui a facilité mes conditions de travail durant toutes ces années

% TODO assistants etc

% 325

\setlength{\marginparwidth}{30mm}
\marginpar{\scriptsize Si vous êtes juste là pour les blagues, ça
commence ici.
\begin{tikzpicture}
\draw[->] (0,0) to ++ (2cm,0);
\end{tikzpicture}
}
Le bureau 26-00-325 a participé à sa manière à l'élaboration de ce document. La
pause café et son rituel piston ont permis de débattre de jour après jour des
avantages des foncteurs applicatifs, de la nécessité d'un salaire minimal, de la
meilleure manière d'écrire un interprète Basic ou de la stratégie qui nous
permettrait d'en venir à bout de ce niveau 5 de Jamestown, le tout bien sûr
schémas, rébus et contrepèteries à l'appui.
Merci donc à Vivien (dont je n'écorcherai pas le nom --- contrairement à
d'autres), Philippe (dont on sait où il se cache), Benjamin (pour son bon goût),
Mathias (parce qu'il a la classe), Guillaume (pour ses deals de café du 9-3, tu
me remettras 1kg de rouge t'as vu?), Aurélien (le type-classieux de
Rochechouart), Jérémie ($λ$-traître devant l'éternel).
Spéciale dédicace à Tarpuy, Iodure M., Yannick S., Mato, Alexandra S, Rebecca
B., Rick A., schmoyoho, les thèmes respectifs de Guile et de l'invité surprise
ainsi que toute l'équipe de Final Form Games.

% Potes
% TODO noms
% Famille

On raconte que la thèse c'est un ascenseur émotionnel. C'est cliché, mais c'est
vrai. Pour partager les moments agréables et faciliter les moments de doutes,
un grand merci à mes amis qui m'ont bien aidé. Et un peu charrié, aussi.

Merci aussi à ma famille qui m'a toujours permis de poursuivre mes projets.
D'ailleurs sans mes premières lignes de code sur l'Amstrad CPC 6128+ familial,
j'aurais sûrement tourné différemment!

% Anaïs

Enfin, merci à toi, Anaïs. Tu es sans doute celle qui a vu le plus les coulisses
de ce travail de longue haleine, jouant à la fois le rôle de confidente et
d'attachée de presse, en répondant "Bientôt!" tant de fois à la question "alors
Étienne, ça va? Il soutient quand?". Aujourd'hui je peux te le dire: ça se passe
le 10 Juillet et vous êtes tous invités!
