\section{Contexte}

Une description plus détaillée de l'implantation de Linux peut être trouvée dans
\cite{UnderstandingTheLinuxKernel}.

\section{Rôle d'un système d'exploitation}
\section{Espace noyau, espace utilisateur}

\todo{OS = kernel ?}

Le système d'exploitation est l'interface logicielle entre le matériel et les
programmes qui vont s'exécuter sur un ordinateur. Pour une description détaillée
de son rôle typique, le lecteur pourra se référer à \cite{tanenbaum}.

Néanmoins on peut citer les rôles suivants :

\begin{itemize}
\item gestion des processus : un système d'exploitation peut permettre
  d'exécuter plusieurs programme à la fois. Il faut alors orchestrer ces
  différents processus et les séparer en terme de temps et de ressources
  partagées.
\item
  gestion de la mémoire : chaque processus, en plus du noyau, doit disposer d'un
  espace mémoire différent. C'est-à-dire qu'un processus ne doit pas pouvoir
  interférer avec un autre.
\item
  gestion des périphériques : le noyau étant le seul code à s'exécuter en mode
  privilégié, c'est lui qui doit communiquer avec les périphériques matériels.
\item
  abstractions : le noyau fournit au programmes une interface unifiée,
  permettant de stocker des informations de la même manière sur un disque dur ou
  une clef USB (alors que l'accès se déroulera de manière très différente en
  pratique). C'est ici que la notion arbitraire de fichier sera définie, par exemple.
\end{itemize}

\section{Cas de Linux}

Sous Linux, on retrouve cette séparation entre espace noyau et espace
utilisateur. Cette section décrit l'implémentation de cette séparation dans le
contexte suivant :

\begin{itemize}
\item l'architecture Intel x86
\item un système 32 bits
\item le noyau Linux
\end{itemize}

Sous d'autres architectures et d'autres systèmes d'exploitation, des mécanismes
similaires existent, et ces travaux peuvent sans doute s'y appliquer.

L'isolation est réalisée par le matériel :

\section{Appels système}
