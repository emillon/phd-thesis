Dans le chapitre~\ref{cha:typbase}, nous avons vu comment ajouter un système de
types forts statiques à un langage impératif. Ici, nous étendons ce système afin
de lui ajouter des \emph{qualificateurs de type} qui décrivent l'origine des
données. Ils permettent de restreindre certaines opérations sensibles à des
expressions dont la valeur est sûre.

\section{Éditions et ajouts}

\gramlr{Qualificateurs}{
\begin{align*}
q  \gramisa & \textsc{Kernel} & \textrm{Donnée noyau (sûre)}
\\ \gramor  & \textsc{User}   & \textrm{Donnée utilisateur (non sûre)}
\end{align*}
}

\section{Propriété d'isolation mémoire}

Le déréférencement d'un pointeur dont la valeur est contrôlée par l'utilisateur
ne peut se faire qu'à travers une fonction qui vérifie la sûreté de celui-ci.
