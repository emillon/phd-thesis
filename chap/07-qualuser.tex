Dans le chapitre~\ref{cha:typbase}, nous avons vu comment ajouter un système de
types forts statiques à un langage impératif. Ici, nous étendons ce système afin
de lui ajouter des \emph{qualificateurs de type} qui décrivent l'origine des
données. Ils permettent de restreindre certaines opérations sensibles à des
expressions dont la valeur est sûre.

\section{Éditions et ajouts}

\gramlr{Qualificateurs}{
\begin{align*}
q  \gramisa & \textsc{Kernel} & \textrm{Donnée noyau (sûre)}
\\ \gramor  & \textsc{User}   & \textrm{Donnée utilisateur (non sûre)}
\end{align*}
}

Ensuite,

\gramlr{Types}{
\begin{align*}
τ  \gramisa & τ q* & \textrm{Pointeur qualifié}
\\ \gramor  & …    & \textrm{Reste inchangé}
\end{align*}
}

voire :

\gramlr{Environnements}{
\begin{align*}
Γ  \gramisa & ε
\\ \gramor  & Γ, x:τ~q
\end{align*}
}

Règle de sûreté du déréférencement

\begin{mathpar}
\irule{Lv-Deref-Kernel}{
  Γ ⊢ e : τ~\textsc{Kernel}*
}{
  Γ ⊢ *e : τ
}
\end{mathpar}

\section{Propriété d'isolation mémoire}

Le déréférencement d'un pointeur dont la valeur est contrôlée par l'utilisateur
ne peut se faire qu'à travers une fonction qui vérifie la sûreté de celui-ci.
