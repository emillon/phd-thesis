Après avoir décrit notre solution théorique dans la partie~\ref{part:lang}, nous
avons présenté ici notre démarche expérimentale. Dans le
chapitre~\ref{cha:implem}, nous avons détaillé l'implantation de notre
prototype. Pour ce faire, nous avons ajouté des étiquettes de type au langage
\newspeak et implanté un algorithme d'inférence de types. Ce prototype est
distribué sur~\link{penjili} sous le nom de \ptrtype.

Ensuite, le but du chapitre~\ref{cha:etudedecas} est d'appliquer notre analyse
(à l'aide de ce prototype) sur le noyau Linux. Après avoir décrit le
fonctionnement des appels système sur ce noyau, on présente deux bugs qui ont
touché respectivement un pilote de carte graphique et l'implantation d'un appel
système. Ils sont la manifestation d'un problème de pointeur utilisateur mal
déréférencé dans le noyau, ainsi que décrit dans le chapitre~\ref{cha:os}. En
lançant notre analyse sur le code présentant un problème, l'erreur est détectée.
Au contraire, en la lançant sur le code après application du correctif, aucune
erreur n'est trouvée.

En s'appuyant sur le langage \newspeak, on gagne beaucoup par rapport à d'autres
représentations intermédiaires. Le fait d'avoir un langage avec peu de
constructions permet de ne pas avoir à exprimer plusieurs fois la même règle
(par exemple, une fois sur la boucle \emph{for} et une autre sur la boucle
\emph{while}).

\label{page:ccl-npk-spk}

Un des inconvénients de notre système est que le modèle mémoire utilisé par
\newspeak est assez différent de celui de \langname (ainsi que décrit dans le
chapitre~\ref{cha:lang}). \newspeak est en effet prévu pour implanter des
analyses précises de valeur reposant sur l'interprétation abstraite, et
nécessite donc un modèle mémoire de plus bas niveau (où on peut créer des
valeurs à partir d'une suite d'octets, par exemple).

Le prototype d'implantation peut évoluer dans deux directions: d'une part, en
continuant à s'appuyer sur \newspeak, on peut réaliser des pré-analyses de
typage qui permettent de guider une analyse de valeurs plus précise, par exemple
en choisissant un domaine abstrait différent en fonction des types de données
rencontrés. D'autre part, il est possible de faire une implantation plus fidèle
à \langname, qui permette d'ajouter de nouvelles fonctionnalités plus éloignées
de C. Par exemple, un système de régions comme~\cite{jfp92} permettrait de
simplifier l'environnement d'exécution en enlevant l'opération de nettoyage
mémoire $\phx\cCleanup$. Le système de types peut également être enrichi, pour
ajouter par exemple du polymorphisme. Cela rapprocherait le langage source de
Rust. Le chapitre~\ref{cha:conclusion} présente quelques unes de ces extensions
possibles.

L'expérimentation, quant à elle, est pour le moment limitée, mais on peut
l'étendre à des domaines de plus en plus importants dans le noyau Linux. Tout
d'abord, le module graphique définit d'autres fonctions implantant des
\emph{ioctls}. Celles-ci reçoivent donc également des pointeurs utilisateur et
sont susceptibles d'être vulnérables à ce genre d'erreurs de programmation.
Ensuite, d'autres modules exposent une interface similaire, à commencer par les
autres pilotes de cartes graphiques. Ceux-ci sont également un terrain sur
lequel appliquer cette analyse.

De manière générale, toutes les interfaces du noyau manipulant des pointeurs
utilisateur gagnent à être analysées. Outre les implantations des \emph{ioctls}
dans chaque pilote et les appels système, les systèmes de fichiers manipulent
aussi de tels pointeurs via leurs opérations de lecture et d'écriture.

% vim: spelllang=fr
