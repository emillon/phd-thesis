\section*{Extrait PLAS}
Manipulating user-provided pointers in the kernel of an operating system can
lead to security flaws if done in an incautious manner. We present an efficient
system to detect and prevent this class of erroneous memory manipulation.

At the core of our approach is \langname, an imperative language that we equip
with a qualified type system, where two kinds of pointers are distinguished:
\emph{safe} pointers, whose value is statically proved to be controlled by the
kernel, and \emph{unsafe} ones, whose value comes from userspace through
run-time system calls. Dereferencing unsafe pointers is forbidden in a static
manner by the means of a strong type system.

A concrete case study is described based on a bug that affected a video driver
in the Linux kernel. We also explain a technique to automatically translate GNU
C code to our core language, which will enable us to analyze larger fractions of
the kernel in order to find similar vulnerabilities.

\section{Introduction}

Modern operating systems such as GNU/Linux are comprised of several independent
programs: the kernel, which handles the machine's hardware as well as trusted
core features; and user applications.

These programs run with different privilege levels: low-level operations can
only be executed by the kernel. They also have different views on the memory. If
a user program had access to the kernel's internal data structures, it could use
it to obtain information it is normally denied, or gain extra privileges (for
example getting access to the \texttt{root} account). This is why the kernel
stores its code and data in a place in memory that is not accessible to
unprivileged programs. On an Intel 32-bit system, it corresponds to addresses
higher than \texttt{0xc0000000}, or 3 GiB.

Because user-mode programs cannot run low-level instructions, they cannot do
anything interesting on their own. They have to use the kernel's facilities to
perform input/output or manipulate the operating system's abstractions (such as
creating a new process). To this end, they can communicate with the kernel
via the narrow interface of \emph{system calls}.

For example, every time a user process calls the \texttt{read()} function, an
architecture-specific mechanism invokes the kernel-mode function
\texttt{sys\_read()}. The user passes a file descriptor, a pointer and an amount
of bytes to fill and the kernel will actually perform input/output and copy the
resulting bytes on the user's buffer.

But if the user supplies a pointer whose value is in the kernel's reserved area,
this zone will be overwritten, potentially leading to a security breach. In a
sense, the kernel has been abused because it accessed the memory with its own
privileges instead of the originator's. This is known as the \emph{confused
deputy problem}\cite{hardy88confused}.

When implementing a system call, it is thus necessary to forbid direct
dereference of a pointer whose value can be controlled by the user. In case of a
system call with a pointer argument, one has to dynamically verify that the
address provided by the user lies within the process's memory space.

The object of this work is to detect the places where it is necessary to insert
dynamic checks. To this end, we augment the type system of a subset of the C
programming language with \emph{type qualifiers}: instead of having a single
pointer type, we add syntactic rules to distinguish between safe pointers that
can be dereferenced with the \texttt{*}, \texttt{->} and \texttt{[]} operators ;
and unsafe pointers that have to go through a special API provided by the
kernel, consisting most notably of the two functions \texttt{copy\_from\_user}
and \texttt{copy\_to\_user}.
