De plus en plus, les logiciels sont présents partout. Le plus petit appareil
électrique contient maintenant des composants matériels programmables
nécessitant des logiciels.

Autant certains de ces systèmes embarqués ont un rôle très délimité et ne
comportent que peu de code, certains comportent des millions de lignes de code.
En 2013, on pense bien sûr aux téléphones portables dont la fonctionnalité et la
complexité les rapprochent plus des ordinateurs que de la génération précédente
qu'ils remplacent. Par exemple, le système Android de Google est basé sur le noyau
Linux, destiné à la base aux micro-ordinateurs.

Le noyau de système d'exploitation est chargé de faire l'intermédiaire entre le
matériel (processeur, mémoire, périphériques, …) et les applications exécutées
sur celui-ci (par exemple un navigateur internet, une calculatrice ou un carnet
d'adresses). Il doit aussi garantir la sécurité et l'isolation de celles-ci.

En tant que tiers de confiance, le noyau a un certain nombre de responsabilités
et est le seul à avoir accès à certaines informations sensibles. Il est capital
de s'assurer qu'il est bien le seul à pouvoir y accéder. En particulier, il faut
pouvoir vérifier que les requêtes faites par l'utilisateur au noyau ne peuvent
pas détourner ce dernier et lui faire fuiter des informations confidentielles.

Le problème est que comme tous les logiciels, les noyaux de système
d'exploitation sont écrits par des humains, qui ne sont pas parfaits. Loin de
là: on estime qu'avant relecture, 100 lignes de code contiennent entre 0.5 et 10
erreurs de programmation en moyenne.
% TODO sourcer
Les activités de relecture et de
débogage ont beau prendre la majeure partie du temps de développement, elles ne
laissent pas un logiciel de qualité parfaite à la sortie. En effet,
même avec des tests les plus exhaustifs il n'est pas possible de simuler le code
d'une manière qui recouvre tous les cas d'utilisation et permette de s'assurer
de sa correction.

De plus, les tests ne permettent que de s'assurer qu'une erreur est présente.
S'assurer qu'un programme est correct est particulièrement délicat, et ne peut
pas être fait automatiquement dans le cas général.

Le typage statique permet d'associer à chaque fragment d'un programme, une
étiquette décrivant quel genre de valeur sera produite par son évaluation. Cette
analyse, faite à la compilation, permet de rejeter les programmes incorrects du
point de vue de cette catégorisation.

Le but de cette thèse est de montrer que les techniques de typage statique
permettent de vérifier que les pointeurs utilisateurs sont manipulés d'une
manière qui préserve l'intégrité du noyau d'un système d'exploitation.

\section{Rôle d'un système d'exploitation}

Pour une description plus détaillée des rôles d'un système d'exploitation ainsi
que plusieurs cas d'étude détaillés, on pourra se référer à \cite{tanenbaum}.

Un ordinateur est constitué de nombreux composants matériels : microprocesseur,
mémoire, et divers périphériques. Pourtant, au niveau de l'utilisateur, des
dizaines de logiciels permettent d'effectuer toutes sortes de calculs et de
communications. Le système d'exploitation permet de faire l'interface entre ces
niveaux d'abstraction.

Au cours de l'histoire des systèmes informatiques, la manière de les programmer
a beaucoup évolué. Au départ, les programmeurs avaient accès au matériel dans
son intégralité : toute la mémoire pouvait être accédée, toutes les instructions
pouvaient être utilisées.

Néanmoins c'est un peu restrictif, puisque cela ne permet qu'à une personne
d'interagir avec le système. Dans la seconde moitié des années 60, sont apparus
les premiers systèmes ``à temps partagé'', permettant à plusieurs utilisateurs
de travailler en même temps.

Permettre l'exécution de plusieurs programmes en même temps est une idée
révolutionnaire, mais elle n'est pas sans difficultés techniques : en effet les
ressources de la machine doivent être aussi partagées entre les utilisateurs et
les programmes. Par exemple, plusieurs programmes vont utiliser le processeur
les uns à la suite des autres (partage \emph{temporel}) ; et chaque programme
aura à sa disposition une partie de la mémoire principale, ou du disque dur
(partage \emph{spatial}).

Si deux programmes (ou plus) s'exécutent de manière concurrente sur le même
matériel, il faut s'assurer que l'un ne puisse pas écrire dans la mémoire de
l'autre, ou que les deux utilisent la carte réseau les uns à la suite des
autres. Ce sont des rôles du système d'exploitation.

Cela passe donc par une limitation des possibilités du programme : plutôt que de
permettre n'importe quel type d'instruction, il communique avec le système
d'exploitation. Celui-ci centralise donc les appels au matériel, ce qui permet
d'abstraire certaines opérations.

Par exemple, si un programme veut copier des données depuis un cédérom vers la
mémoire principale, il devra interroger le bus SATA, interroger le lecteur sur
la présence d'un disque dans le lecteur, activer le moteur, calculer le numéro
de trame des données sur le disque, demander la lecture, puis déclencher une
copie de la mémoire.

Si dans un autre cas il désire récupérer des données depuis une mémorette USB,
il devrait interroger le bus USB, rechercher le bon numéro de périphérique, le
bon numéro de canal dans celui-ci, lui appliquer une commande de lecture au bon
numéro de bloc, puis copier la mémoire.

Ces deux opérations, bien qu'elles aient le même but (copier de la mémoire
depuis un périphérique amovible), ne sont pas effectuées en pratique de la même
manière. C'est pourquoi le système d'exploitation fournit les notions de
fichier, lecteur, etc : le programmeur n'a plus qu'à utiliser des commandes de
haut niveau (``monter un lecteur'', ``ouvrir un fichier'', ``lire dans un
fichier'') et selon le type de lecteur, le système d'exploitation effectuera les
actions appropriées.

En résumé, un système d'exploitation est l'intermédiaire entre le logiciel et
le matériel, et en particulier assure les rôles suivants :

% TODO À affiner ou supprimer

\begin{itemize}
\item
  Gestion des processus : un système d'exploitation peut permettre
  d'exécuter plusieurs programmes à la fois. Il faut alors orchestrer
  ces différents processus et les séparer en terme de temps et de
  ressources partagées.
\item
  Gestion de la mémoire : chaque processus, en plus du noyau, doit
  disposer d'un espace mémoire différent. C'est-à-dire qu'un processus
  ne doit pas pouvoir interférer avec un autre.
\item
  Gestion des fichiers : les processus peuvent accéder à une hiérarchie de
  fichiers, indépendamment de la manière d'y accéder.
\item
  Gestion des périphériques : le noyau étant le seul code ayant des privilèges,
  c'est lui qui doit communiquer avec les périphériques matériels.
\item
  Abstractions : le noyau fournit aux programmes une interface unifiée,
  permettant de stocker des informations de la même manière sur un
  disque dur ou une clef USB (alors que l'accès se déroulera de manière
  très différente en pratique).
\end{itemize}


\section{Noyaux de systèmes d'exploitation}

Les programmes qui composent un système d'exploitation sont exécutés avec un
niveau de privilège différent, correspondant à un jeu d'instructions plus ou
moins important. Cette vérification est faite directement pas le processeur,
d'une manière matérielle. En plus de cette distinction, les programmes ont une
vue différente de la mémoire. Si un programme utilisateur avait accès aux
structures de données internes du noyau, il pourrait obtenir des informations
auquel il n'a pas normalement accès, ou obtenir des privilège supplémentaires
(par exemple en obtenant les privilèges de l'administrateur sur la machine).

C'est pourquoi les noyau stockent leur code et données dans un endroit en
mémoire qui n'est pas accessible aux programmes utilisateur. Sur un système
Intel 32 bits, cela correspond aux adresses supérieures à \texttt{0xc0000000},
ou 3 Gio.

% TODO trad

Because user-mode programs cannot run low-level instructions, they cannot do
anything interesting on their own. They have to use the kernel's facilities to
perform input/output or manipulate the operating system's abstractions (such as
creating a new process). To this end, they can communicate with the kernel
via the narrow interface of \emph{system calls}.

For example, every time a user process calls the \texttt{read()} function, an
architecture-specific mechanism invokes the kernel-mode function
\texttt{sys\_read()}. The user passes a file descriptor, a pointer and an amount
of bytes to fill and the kernel will actually perform input/output and copy the
resulting bytes on the user's buffer.

But if the user supplies a pointer whose value is in the kernel's reserved area,
this zone will be overwritten, potentially leading to a security breach. In a
sense, the kernel has been abused because it accessed the memory with its own
privileges instead of the originator's. This is known as the \emph{confused
deputy problem}\cite{hardy88confused}.

When implementing a system call, it is thus necessary to forbid direct
dereference of a pointer whose value can be controlled by the user. In case of a
system call with a pointer argument, one has to dynamically verify that the
address provided by the user lies within the process's memory space.

The object of this work is to detect the places where it is necessary to insert
dynamic checks. To this end, we augment the type system of a subset of the C
programming language with \emph{type qualifiers}: instead of having a single
pointer type, we add syntactic rules to distinguish between safe pointers that
can be dereferenced with the \texttt{*}, \texttt{->} and \texttt{[]} operators ;
and unsafe pointers that have to go through a special API provided by the
kernel, consisting most notably of the two functions \texttt{copy\_from\_user}
and \texttt{copy\_to\_user}.

\section{Systèmes de types}

\section{Plan de la thèse}

Cette thèse comporte trois parties.

\paragraph{La partie~\ref{part:ctx}} présente le contexte de ces travaux. Le
fonctionnement général d'un système d'exploitation y est détaillé, et les
problèmes de manipulation de pointeurs contrôlés par l'utilisateur y sont
introduits. On fait ensuite un tour d'horizon des techniques existantes
permettant de traiter ce problème.

\paragraph{La partie~\ref{part:lang}} décrit notre solution : \langname, un
langage impératif. On y décrit sa syntaxe, sa sémantique ainsi qu'un système de
types statiques. On l'étend ensuite pour capturer les problèmes d'adressage
mémoire présents dans les systèmes d'exploitation en ajoutant des pointeurs
contrôlés par l'utilisateur. Le système de types est également étendu. Pour
chacune de ces variantes, on établit la propriété de sûreté de typage reliant la
sémantique dynamique aux types statiques.

\paragraph{La partie~\ref{part:xp}} documente la démarche expérimentale associée
à ces travaux. L'implantation du système de types est décrite, afin que la
manière de transformer automatiquement du code C en \langname. Un cas d'étude
est déroulé, consistant d'un bug ayant touché le noyau Linux. Il est démontré
que le système de type capture précisément ce genre d'erreur de programmation.
Enfin, les possibilités d'extension tant théoriques qu'expérimentales sont
présentées.

%  - omniprésence
%  - complexité
%  - vérification
%  - taux de bugs après relecture
%  - vérification automatique obligatoire
%  - historique des techniques
%  - analyse non intrusive
%  - méthodes de développement du noyau
%  - syntaxe, sémantique, typage
