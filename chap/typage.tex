\section{Présentation et but}

Au plus bas niveau d'abstraction, un ordinateur ne manipule que des nombres
entiers \footnote{en allant plus loin on pourrait dire qu'il ne manipule que des
suites de bits} : en langage machine, il n'y a pas de distinction entre une
adresse et un nombre.

Pourtant il est clair que certaines opérations n'ont pas de sens : par exemple,
ajouter deux adresses, ou déréférencer le résultat d'une division sont des
comportements qu'on voudrait pouvoir empêcher.

En un mot, le but du typage est de classifier les objets et de restreindre les
opérations possibles selon la classe d'un objet : "ne pas ajouter des pommes et
des oranges".

Le modèle qui permet cette classification est appelé \emph{système de types} et
est en général constitué d'un ensemble de \emph{règles de typage}, comme "un
entier plus un entier égale un entier".

\section{Taxonomie}

La définition d'un langage de programmation introduit la plupart du temps celle
d'un système de types. Il y a donc de nombreux systèmes de types différents,
dont nous pouvons donner une classification sommaire.

\subsection{Dynamique, statique, mixte}

Il y a deux grandes familles de systèmes de types, selon quand se fait la
vérification de types. On peut en effet l'effectuer au moment de l'exécution, ou
au contraire prévenir les erreurs à l'exécution en la faisant au moment de la
compilation (ou avant l'interprétation).

\subsubsection{Typage dynamique}

La première est le typage \emph{dynamique}. Pour différencier les différents
types de données, on ajoute une étiquette à chaque valeur. Dans tout le
programme, on ne manipulera que des valeurs étiquettées :

\begin{itemize}
\item
  si on veut réaliser l'opération
  $(0x00000001, Int) + (0x0000f000, Int)$, on vérifie tout d'abord qu'on
  peut réaliser l'opération $+$ entre deux $Int$. Ensuite on réalise
  l'opération elle même, qu'on étiquette avec le type du résultat :
  $(0x0000f001, Int)$
\item
  si au contraire on tente d'ajouter deux adresses
  $(0x2e8d5a90, Addr) + (0x76a5e0ec, Addr)$, la vérification échoue et
  l'opération s'arrête avec une erreur.
\end{itemize}

Il existe plusieurs techniques pour signaler les erreurs de typage dynamiques :
arrêter l'exécution, lever une exception, convertir une opérande, utiliser une
valeur d'erreur, etc.

\subsubsection{Typage statique}

La seconde technique est le typage \emph{statique} : plutôt que de vérifier les
types sur les données, on les vérifie à l'arrêt, sur les expressions. Cela
implique par exemple que chaque variable doit contenir des valeurs d'un même
type tout au long de sa vie.

\todo{...}

\subsubsection{Typage hybride ("stanamique")}

Il est aussi possible de mélanger ces deux approches :

\begin{itemize}
\item
  à la compilation, essayer d'obtenir les types les plus précis
  possibles
\item
  pour les cas restants, insérer un test dynamique
\end{itemize}

\subsection{Fort, faible, sound}

Un autre critère intéressant est la "force" du typage : est-ce qu'il peut
induire des conversions de type implicites ?

Prenons l'exemple de l'addition entre un entier $2$ et un flottant $3.14$. On
peut le voir de plusieurs manières différentes :

\begin{itemize}
\item
  un entier est un nombre décimal comme les autres : $2 = 2.00$. Donc
  $2 + 3.14 = 2.00 + 3.14 = 5.14$.
\item
  $2$ est un entier, $3.14$ un flottant : ces deux types sont différents et
  l'opération est impossible.
\end{itemize}

\subsection{Polymorphisme}
\subsection{Expressivité, garanties, types dépendants}

\section{Exemples}

\subsection{Faible dynamique : Perl}
\subsection{Faible statique : C}
\subsection{Fort dynamique : Python}
\subsection{Fort statique : OCaml}
\subsection{Fort statique à effets typés : Haskell}
\subsection{Theorem prover : Coq}
