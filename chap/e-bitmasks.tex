Par exemple, en ouvrant un fichier on précise le mode du fichier (lecture,
écriture ou les deux) par les bits 1 et 2, s'il faut créer le fichier ou non
s'il n'existe pas par le bit 7, s'il dans ce cas il doit être effacé par le bit
8), etc. On obtient un mode en réalisant un ``ou'' bit à bit entre des
constantes. Ces deux utilisations du type \texttt{int} n'ont rien à voir ; il
faudrait donc empêcher d'utiliser un descripteur de fichier comme un mode, et
vice-versa. De même, aucun opérateur n'a de sens sur les descripteurs de
fichier, mais l'opérateur \texttt{\textbar{}} du ``ou'' bit à bit doit rester
possible pour les modes.
