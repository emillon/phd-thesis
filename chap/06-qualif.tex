\epigraph{A friend of mine in a compiler writing class produced a compiler with
         one error message "you lied to me when you told me this was a program".
        }{--- \textup{Pete Fenelon}}

% TODO[E] le chapeau est à reformuler complètement, c'est trop rapide à suivre,
% par contre c'est clair à partir de 6.1 mais ensuite il faudrait détailler

Dans le chapitre~\ref{cha:typbase}, nous avons vu comment ajouter un système de
types forts statiques à un langage impératif (défini dans le
chapitre~\ref{cha:lang}).

% TODO[E] 1è phrase à revoir

Ici, nous étendons l'expressivité de \langname avec un système d'annotations de
``souillure'' (\emph{tainting} en anglais). Un cas d'erreur est ajouté,
%TODO[E] trop tôt
lorsqu'on tente d'accéder à une valeur souillée. Avec cet ajout, la propriété de
progrès (théorème~\ref{thm:progres}) n'est donc plus valable.

% TODO[E] un peu péjoratif ?


Afin de retrouver cette adéquation entre la sémantique et le systme de typage,
ce dernier est étendu d'un système de \emph{qualificateurs de type} qui
décrivent l'origine des données. Ils permettent de restreindre certaines
opérations sensibles à des expressions dont la valeur est sûre.

% TODO[E] peu clair

La propriété de progrès est alors retrouvée (théorème~\ref{thm:progres-qual}).

\section{Extensions noyau pour \langname}

Jusqu'ici \langname, tel qu'il a été présenté dans le chapitre~\ref{cha:lang}
est un langage de programmation impératif généraliste. Aucune construction en
particulier n'est prévue pour implanter un système d'exploitation.

On ajoute donc la notion de valeur provenant de l'espace utilisateur (cf.
chapitre~\ref{cha:os}) en trois étapes (figure~\ref{fig:qualif-changes}) :

\begin{figure}%{{{

\begin{align*}
\gramdef{Expressions}{e}
  { … }{}
  { \eTaint{e} }{Expression souillée}
  { \uGet{lv}{e}}{ Lecture depuis l'espace utilisateur}
  { \uPut{e}{e}}{ Écriture dans l'espace utilisateur}
  {END}
\\
\gramdef{Contextes}{C}
  { … }{}
  { \eTaint{C}}{}
  {END}
\\
\gramdef{Chemins}{φ}
  { … }{}
  { \vTainted{φ} }{Valeur souillée}
  {END}
\\
\gramdef{Erreurs}{Ω}
  { … }{}
  { \serr{taint} }{Erreur de souillure}
  {END}
\end{align*}

\caption{Ajouts liés aux pointeurs utilisateurs}
\label{fig:qualif-changes}
% TODO définir les états et ms

\end{figure}%}}}

\begin{itemize}
\item
  tout d'abord, on ajoute une expression d'annotation sur les variables
  indiquant que celles-ci sont contrôlés par un utilisateur non privilégié,
  ainsi que des opérateurs de copie sûre.
\item
  ensuite, on étend l'ensemble des valeurs possibles pour les pointeurs
  à une valeur $\vTainted{φ}$ signifiant que l'objet pointé se situe en
  % TODO [E] ?
  espace utilisateur
\item
  enfin, on définit une nouvelle erreur $\serr{taint}$ produite par le
  déréférencement d'un pointeur ayant une telle valeur.
\end{itemize}

Pour adapter l'évaluation, plusieurs cas sont à rajouter. D'une part, la
présence de $\phx{\eTaint}$ dans une instruction consiste à ajouter un
$\phx{\vTainted}$ dans la valeur construite. Ceci ne peut être fait que dans le
cas où la valeur est un chemin $φ$, c'est-à-dire que la construction
$\phx{\eTaint}$ ne peut se faire que sur une expression de type pointeur.

% TODO passe globale c'est-a-dire & co (pas de ')

\begin{mathpar}
  \irule{Expr-Tainted}
    { }
    { \mm{m}{\eTaint{φ}}{m}{\vTainted{φ}} }
\end{mathpar}

D'autre part, une règle accède à la mémoire : \textsc{Exp-Lv}; pour
rappel :

\begin{mathpar}
  \semrule{Exp-Lv}
\end{mathpar}

Puisque la définition des chemins $φ$ a été changée, il est aussi nécessaire de
redéfinir la lentille $Φ$ utilisée ci-dessus (définition~\ref{def:acces-phi}).

% TODO [E] à détailler

On rajoute donc le cas :

\begin{align*}
Φ(\vTainted{φ}) =& \serr{taint}
\end{align*}

Pour accéder à ces valeurs, il faut utiliser les opérateurs $\phxx\uGet$ et
$\phxx\uPut$.

% TODO leur sémantique

\section{Insuffisance des types simples}

% TODO[E] c'est ok

Étant donné \langname augmenté de cette extension sémantique, on peut étendre
trivialement le système de types avec la règle suivante :

\begin{mathpar}
  \irule{Taint-Ignore}
    { Γ ⊢ e : t* }
    { Γ ⊢ \eTaint{e} : t* }
\end{mathpar}

Cette règle est compatible avec l'extension, sauf qu'elle introduit des termes
qui sont bien typables mais dont l'évaluation provoque une erreur autre que
$\serr{div}$, $\serr{array}$ ou $\serr{ptr}$, violant ainsi le
théorème~\ref{thm:progres}.

Par exemple, supposons que $x$ soit une variable globale entière, et posons $e =
* \eTaint {\& x}$.

$e$ est alors bien typée sous $Γ = x : \tInt$ :

\begin{mathpar}
  \irule{Lv-Deref}
    {
      \irule{Taint-Ignore}
        {
          \irule{Lv-Deref}
            {
              \irule{Lv-Var}
                { x:\tInt ∈ Γ }
                { Γ ⊢ x : \tInt }
            }
            { Γ ⊢ \& x : \tInt*}
        }
        { Γ ⊢ \eTaint {\& x} : \tInt*}
    }
    { Γ ⊢ * \eTaint {\& x} : \tInt}
\end{mathpar}

Posons alors $m = ([[x↦0]], [])$ (on a bien $\mcomp{Γ}{m}$). L'évaluation de $e$
sous $m$ provoque une erreur, comme le montre la dérivation suivante.

\begin{mathpar}
\inferrule*
  {
    \irule{Exp-Lv}
      {
        \inferrule*
          { }
          {m[*\vTainted{x}] = \serr{taint}}
      }
      { \mm{m}{* \eTaint {\& x}}{m}{\serr{taint}} }
    \\
    \irule{Eval-Err}
      { }
      {
        \msi{m}{\serr{taint}} → \serr{taint}
      }
  }
  {\msi{m}{* \eTaint {\& x}} → \serr{taint}}
\end{mathpar}

\section{Extensions du système de types}

On présente ici un système de types plus expressif permettant de capturer les
extensions de sémantique. \emph{In fine}, cela permettra de prouver le
théorème~\ref{thm:progres-qual} qui est l'équivalent du
théorème~\ref{thm:progres} mais pour le nouveau jugement de typage.

% TODO [E] ??

Définir un nouveau système de types revient à définir un nouveau jugement de
typage $\cdot ⊢_q \cdot : \cdot$, à partir d'un ensemble de règles. Pour la
plupart, les règles seront identiques, donc sauf mention contraire, les règles
portant sur $\cdot ⊢ \cdot : \cdot$ s'appliqueront aussi sur $\cdot ⊢_q \cdot :
\cdot$. Naturellement, la plupart des différences porteront sur le traitement des
pointeurs.

La différence principale est qu'à chaque pointeur, on ajoute un
\emph{qualificateur} qui représente \emph{qui} contrôle sa valeur
(section~\ref{sec:secu-syscalls}). Les deux qualificateurs sont :

\begin{itemize}
\item
  \qKernel : il s'applique aux pointeurs contrôlés par le noyau. Par exemple,
  prendre l'adresse d'un objet de la pile noyau donne un pointeur noyau.
\item
  \qUser : il s'applique aux pointeurs qui proviennent de l'espace utilisateur.
  Ces pointeurs proviennent toujours d'interfaces particulières, comme les
  appels système ou les paramètres de la fonction \texttt{ioctl}.
% TODO [E] tu comptes malloc ? &x ?
\end{itemize}

Cet ajout est précisé dans la figure~\ref{fig:qualif-changes-typ}.

\begin{figure}%{{{

\begin{align*}
\gramdef{Qualificateurs}{q}
  { \qKernel }{Donnée noyau (sûre)}
  { \qUser   }{Donnée utilisateur (non sûre)}
  {END}
\\
\gramdef{Types}{t}
  { … }{}
  { \msout{t~*} }{\textrm{Pointeur}}
  { t~q~* }{Pointeur qualifié}
  {END}
\end{align*}

\caption{Changements liés aux qualificateurs de types}
\label{fig:qualif-changes-typ}
\end{figure}%}}}

Au niveau du système de types, la principale restriction est que seuls les
pointeurs \qKernel peuvent être déréférencés de manière sûre :

\begin{mathpar}
\irule{Lv-Deref-Kernel}{
  Γ ⊢_q e : τ~\textsc{Kernel}*
}{
  Γ ⊢_q *e : τ
}
\end{mathpar}

% TODO [E] et que se passe-t'il quand on a une expression e de type τ K* que
% l'on cherche à déréférencer ?

L'opérateur $\eTaint{\cdot}$ transforme un pointeur selon la règle de souillure
suivante :

% TODO définir les stubs

\begin{mathpar}
\irule{Taint}
  { Γ ⊢_q e : t~q~* }
  { Γ ⊢_q \eTaint{x} : t~\qUser~* }
\end{mathpar}

Les opérateurs $\phxx\uGet$ et $\phxx\uPut$ sont typés de la manière suivante :

\begin{mathpar}
    \disprule{GetU}

    \disprule{PutU}
\end{mathpar}

% TODO adresse = kernel

% TODO parler de sous typage

La définition du typage sémantique doit aussi être étendue au cas $φ =
\vTainted{φ'}$. Les références mémoires sont ``nettoyées'' pour accéder à la
left-value encapsulée.

\begin{mathpar}
  \irule{S-Tainted}
    { m ⊧           φ'  : t~\qKernel~* }
    { m ⊧ \vTainted{φ'} : t~\qUser~* }
\end{mathpar}

\wip

\section{Sûreté du typage}

Le déréférencement d'un pointeur dont la valeur est contrôlée par l'utilisateur
ne peut se faire qu'à travers une fonction qui vérifie la sûreté de celui-ci.

\begin{theorem}[Progrès pour les types qualifiés]
  \label{thm:progres-qual}

  Supposons que $Γ ⊢_q e : t$. Soit $m$ un état mémoire tel que $\mcomp{Γ}{m}$.
  Alors l'un des cas suivant est vrai :

\begin{itemize}
  \item $∃ v ≠ Ω, e = v$
  \item $∃ (e', m'), \mcomp{Γ}{m'} ∧ \mm{m}{e}{m'}{e'}$
  \item $∃ Ω ∈ \{\serr{div},\serr{array},\serr{ptr}\}, \msi{m}{e} → Ω$
\end{itemize}
\end{theorem}

%\begin{proof}

  %% TODO

%\end{proof}

Et nous donnons un équivalent du théorème~\ref{thm:preservation}.

\begin{theorem}[Préservation pour les types qualifiés]

  Si une expression est typable et que son évaluation produit une valeur, alors
  cette valeur est du même type que l'expression.

  Si $Γ ⊢_q e : t$ et $e → v$ % TODO

  alors $Γ ⊢_q v : t$.

\end{theorem}

% vim: spelllang=fr
