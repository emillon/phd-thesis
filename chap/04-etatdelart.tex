L'analyse statique de programmes est un sujet de recherche actif depuis
l'apparition de la science informatique.

L'analyse la plus simple consiste à traiter un programme comme du texte, et à y
rechercher des motifs dangereux. Ainsi, utiliser des outils comme \texttt{grep}
permet parfois de trouver un grand nombre de vulnérabilités\cite{SpenderGrep}.

On peut continuer cette approche en recherchant des motifs mais en étant
sensible à la syntaxe et au flot de contrôle du programme. Cette notion de
\emph{semantic grep} est présente dans l'outil Coccinelle
\cite{coccinelle09,coccinelle11} \todo{lire coccinelle09} : on peut définir des
\emph{patches sémantiques} pour détecter ou modifier des constructions
particulières.

Dans le cas particulier des vulnérabilités liées à une mauvaise utilisation de
la mémoire, les développeurs du noyau Linux ont ajouté un système d'annotations
au code source. Un pointeur peut être décoré d'une annotation
\texttt{\_\_kernel} ou \texttt{\_\_user} selon s'il est sûr ou pas. Celle-ci
sont ignorées par le compilateur, mais un outil d'analyse statique ad-hoc nommé
Sparse \link{sparse} peut être utilisé pour détecter les cas les plus simples
d'erreurs.

Ce système d'annotations sur les types a été formalisé sous le nom de
\emph{qualificateurs de types} : chaque type peut être décoré d'un ensemble de
qualificateurs (à la manière de \texttt{const}), et des règles de typage
permettent d'établir des propriétés sur le programme. Ces analyses ont été
implantée dans l'outil CQual
\cite{pldi99,usenix01,pldi02,cquk-usenix04,toplas-quals}. \todo{lister les
applications}.

L'interprétation abstraite est une technique d'analyse générique qui permet de
simuler statiquement tous les comportements d'un programme Cousot
\cite{Cousot77,Cousot92-1}. Un exemple d'application est de calculer les bornes
de variations des variables pour s'assurer qu'aucun débordement de tableau n'est
possible. Cette technique est très puissante mais possède plusieurs
inconvénients. D'une part, pour réaliser une analyse interprocédurale il faut
partir d'un point en particulier du programme (comme la fonction \texttt{main}).
Cette hypothèse n'est pas facilement satisfaite dans un noyau de système
d'exploitation, qui possède de nombreux points d'entrée. D'autre part,
il est très difficile de faire passer à l'échelle un interpréteur abstrait
\cite{AstreeScale,coverityBillion}.

L'approche par typage, plus légère, est séduisante. Pour les différents enjeux
des systèmes de types statiques, on pourra se référer à~\cite{TAPL}.


Ce que nous voulons vérifier peut être vue comme une propriété de flot. Un
\emph{survey} des problèmes et techniques existantes peut être trouvé
dans~\cite{sm-jsac03}.

\wip{}

Interprétation Abstraite :
    widening \cite{granger},
    CGS \cite{cgs},
    Astrée : presentation\cite{Astree04,Astree05},

Politiques de sécurité par les types : \cite{lightweight-static-capabilities,LZ06a}

Divers : Contracts \cite{cssv}, Taint sequences \cite{mdv10}, Dynamic taint analysis \cite{oakland10}, perl

Frama-C ?

CCurred ?
