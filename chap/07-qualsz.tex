Dans ce chapitre, nous présentons une autre extension au système de types du
chapitre~\ref{cha:typbase}, similaire à celle du chapitre~\ref{cha:qualuser}.
Il s'agit cette fois-ci de détecter les pointeurs sur caractères (\texttt{char
*}) qui sont terminés par un caractère NUL et donc une chaîne C correcte. La
bibliothèque C propose quantité de fonctions manipulant ces chaînes et appeler
une fonction comme \texttt{strcpy} sur un pointeur quelconque est un problème de
sécurité que nous cherchons à détecter.

\section{But}

Le langage C ne fournit pas directement de type "chaîne de caractère". C'est au
programmeur de les gérer via des pointeurs sur caractère (\texttt{char *}).

En théorie le programmeur est libre de choisir une représentation : des chaînes
préfixées par la longueur, une structure contenant la taille et un pointeur vers
les données, ou encore une chaîne avec un terminateur comme 0.

Néanmoins c'est ce dernier style qui est le plus idiomatique : par exemple, les
littéraux de chaîne (\texttt{"comme ceci"}) ajoutent un octet nul à la fin. De
plus, le standard décrit dans la bibliothèque d'exécution de nombreuses
fonctions destinées à les manipuler --- c'est le fichier
\texttt{\textless{}string.h\textgreater{}} (\cite{AnsiC} section 7.21).

Ainsi la fonction \texttt{strcpy} a pour protoype :

\begin{Verbatim}
char *strcpy(char *dest, const char *src);
\end{Verbatim}

Elle réalise la copie de la chaîne pointée par \texttt{src} à l'endroit pointé
par \texttt{dest}. Pour détecter la fin de la chaîne, cette fonction parcourt la
mémoire jusqu'à trouver un caractère nul. Une implémentation naïve pourrait être:

\begin{Verbatim}
char *strcpy(char *dest, const char *src)
{
    int i;
    for(i=0;src[i]!=0;i++) {
      dest[i] = src[i];
    }
    return dest;
}
\end{Verbatim}

La copie n'est arrêtée que lorsqu'un 0 est lu. Autrement dit, si quelqu'un
contrôle la valeur pointée par \texttt{src}, il pourra écraser autant de données
qu'il le désire. On est dans le cas d'école du débordement de tampon sur la pile
tel que décrit dans \cite{SmashingTheStack}. Considérons la fonction suivante :

\begin{Verbatim}
void f(char *src)
{
    char buf[100];
    strcpy(buf, src);
}
\end{Verbatim}

Si le pointeur \texttt{src} pointe sur une chaîne de longueur supérieure à 100
(ou une zone mémoire qui n'est pas une chaîne et ne contient pas de 0), les
valeurs placées sur la pile juste avant \texttt{buf} (à une adresse supérieure)
seront écrasées. Avec les conventions d'appel habituelles, il s'agit de
l'adresse de retour de la fonction. Un attaquant pourra donc détourner le flot
d'exécution du programme.

Pour éviter ces cas de fonctions vulnérables, on peut introduire une distinction
entre les pointeurs \texttt{char *} classiques (représentant l'adresse d'un
caractère par exemple) et les pointeurs sur une chaîne terminée par un caractère
nul.

Dans certaines bases de code (la plus célèbre étant celle de Microsoft), une
convention syntaxique est utilisée : les pointeurs vers des chaînes terminées
par 0 ont un nom qui commence par \texttt{"sz"}, comme \texttt{"szUser"}. C'est
pourquoi nous appellerons ce qualificateur de type \texttt{sz}.

\section{Approche}

Cette propriété est un peu différente de la séparation entre espace utilisateur
et espace noyau modélisée dans le chapitre~\ref{cha:qualuser} : autant un pointeur
reste contrôlé par l'utilisateur (ou sûr) toute sa vie, autant le fait d'être
terminé par un octet nul dépend de l'ensemble de l'état mémoire. Il y a deux
problèmes principaux à considérer.

D'une part, l'\emph{aliasing} rend l'analyse difficile : si \texttt{p} et
\texttt{q} pointent tous les deux vers une même zone mémoire, le fait de
modifier l'un peut modifier l'autre. D'autre part, ce n'est pas parce qu'une
fonction maintient l'invariant de terminaison, qu'elle le maintient à chaque
instruction.

On peut résoudre en partie le problème d'\emph{aliasing} en étant très
conservateur, c'est à dire en sous-approximant l'ensemble des chaînes du
programme (on traitera une chaîne légitime comme une chaîne non terminée,
interdisant par excès de zèle les fonctions comme \texttt{strcpy}).

Le second problème est plus délicat puisqu'il casse l'hypothèse habituelle que
chaque variable conserve le même type au long de sa vie. Plusieurs techniques
sont possibles pour contourner ce problème : la première est d'être encore une
fois conservateur et d'interdire ces constructions (on ne pourrait alors
analyser que les programmes ne manipulant les chaînes qu'à travers les fonctions
de la bibliothèque standard). Une autre est d'insérer des annotations permettant
de s'affranchir localement du système de types. Enfin, il est possible
d'utiliser un système de types où les variables ont en plus d'un type, un
automate d'états possible dépendant de la position dans le programme : c'est le
concept de \emph{typestates}\cite{tse12-typestate}.

\section{Annotation de \texttt{string.h}}

\subsection{Fonctions de copie}

\begin{itemize}
\item memcpy
\item memmove
\item strcpy
\item strncpy
\end{itemize}

\subsection{Fonctions de concaténation}

\begin{itemize}
\item strcat
\item strncat
\end{itemize}

\subsection{Fonctions de comparaison}

\begin{itemize}
\item memcmp
\item strcmp
\item strncmp
\item strcoll
\item strxfrm
\end{itemize}

\subsection{Fonctions de recherche}

\begin{itemize}
\item memchr
\item strchr
\item strcspn
\item strpbrk
\item strrchr
\item strspn
\item strstr
\item strtok
\end{itemize}

\subsection{Fonctions diverses}

\begin{itemize}
\item memset
\item strerror
\item strlen
\end{itemize}

% vim: spelllang=fr
