\long\def\partintro{
Après avoir décrit notre solution dans la partie~\ref{part:lang}, on présente
ici son implantation.

Le chapitre~\ref{cha:implem} décrit l'implantation en elle-même: un prototype
d'analyseur de types, distribué avec le langage \newspeak sur~\link{penjili}. Il
s'agit d'un logiciel libre, distribué sous la license LGPL. La compilation
depuis C est réalisée par l'utilitaire \ctonewspeak. Celui-ci, tout comme le
langage \newspeak, proviennent d'EADS et sont antérieurs à ce projet, mais le
support de plusieurs extensions GNU C a été développé spécialement pour pouvoir
analyser le code du noyau Linux.

L'analyse en elle-même est implantée de la manière classique avec une variation
de l'algorithme W de Damas et Milner. Pour des raisons de simplicité et
d'efficacité, l'unification est faite en utilisant le partage de références
plutôt que des substitutions. L'algorithme d'inférence ne pose pas de problèmes
de performance.

Ensuite, dans le chapitre~\ref{cha:etudedecas}, on évalue cette implantation sur
le noyau Linux. On commence par décrire comment fonctionnennt les appels système
sous ce noyau, et comment le \emph{confused deputy problem} évoqué dans le
chapitre~\ref{cha:os} peut arriver dans ce contexte. Dans une deuxième partie,
on décrit le cas de deux \emph{bugs} dans le noyau Linux. On montre que, pour
chacun, les analyses précédentes permettent de distinguer statiquement le cas
incorrect du cas corrigé.

% TODO[E] bugs: à indiquer
}
