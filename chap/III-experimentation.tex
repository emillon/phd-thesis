\long\def\partintro{
Après avoir décrit notre solution dans la partie~\ref{part:lang}, on présente
ici son implantation.

Le chapitre~\ref{cha:implem} décrit l'implantation en elle-même: un prototype
d'analyseur de types, distribué avec le langage \newspeak sur
\url{http://penjili.org/}. Il s'agit d'un logiciel libre, distribué sous la
license LGPL. La compilation depuis C est réalisée par l'utilitaire
\ctonewspeak. Celui-ci, comme le langage \newspeak, est antérieur à ce projet,
mais le support de plusieurs extensions GNU C a été développé spécialement pour
pouvoir analyser le code du noyau Linux.

L'analyse en elle-même est implantée de la manière classique avec une variation
de l'algorithme W de Damas et Milner. Pour des raisons de simplicité et
d'efficacité, l'unification est faite en utilisant le partage de références
plutôt que des substitutions explicites. L'algorithme d'inférence ne pose pas de
problèmes de performance.

Ensuite, dans le chapitre~\ref{cha:etudedecas}, on applique cette implantation
au noyau Linux. On commence par décrire l'implantation des appels système sous
ce noyau, et les problèmes de sécurité qui peuvent arriver (ceci a déjà été
abordé dans le chapitre~\ref{cha:os}). Dans une deuxième partie, on décrit le le
cas d'un \emph{bug} de pilote graphique dans le noyau Linux. On montre que les
analyses précédentes permettent de distinguer statiquement entre le cas
incorrect et le cas corrigé.
}
