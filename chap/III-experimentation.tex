\long\def\partintro{
On décrit ici la démarche expérimentale liée à l'implémentation des analyses
décrites dans la partie~\ref{part:lang}.

Le chapitre~\ref{cha:implem} décrit l'implémentation en elle-même : comment le
code source C est compilé vers \langname, et comment les types du programme sont
vérifiés.

Ensuite, dans le chapitre~\ref{cha:etudedecas}, le cas d'un bogue de pilote
graphique dans le noyau Linux est étudié. On montre que les analyses précedentes
permettent de distinguer statiquement entre le cas incorrect et le cas corrigé.

Enfin, le chapitre~\ref{cha:conclusion} conclut : les limitations de cette
approche sont présentées, ainsi qu'un résumé des contributions de cet ouvrage.

% TODO bogue/bug
% TODO ouvrage ça le fait ?
}
