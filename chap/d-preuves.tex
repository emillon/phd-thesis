\wip{}

\section{Composition de lentilles}

\label{proof:compo-lens}

\begin{proof}
Pour prouver que $ℒ_1 \ggg ℒ_2 ∈ \setLens{A}{C}$, il suffit de prouver les trois
propriétés caractéristiques.

\paragraph{GetPut}%{{{

{

\def\xa{
  \lensPut{ℒ}{
    \lensGet{ℒ}{r}
    }{
      r
    }
  }


\def\xb{
  \lensPut{ℒ}
    {
      \lensGet{ℒ_2}{
        \lensGet{ℒ_1}{r}
      }
    }
    { r }
}

\def\xab{ définition de \lensGetX{ℒ} }

\def\xc{
  \lensPut{ℒ_1}
          {
            \lensPut{ℒ_2}
                    {
                      \lensGet{ℒ_2}
                        {
                          \lensGet{ℒ_1}{r}
                        }
                    }
                    { \lensGet{ℒ_1}
                              {r}
                    }
          }
          {r}
}

\def\xbc{ définition de \lensPutX{ℒ} }

\def\xd{
  \lensPut{ℒ_1}
          {
            \lensGet{ℒ_1}{r}
          }
          {r}
}

\def\xcd{ \textsc{GetPut} sur $ℒ_2$ }

\def\xe{r}

\def\xde{\textsc{GetPut} sur $ℒ_1$}

\begin{conteq}
    \xa{} \\
  = \xb{} & \xab{} \\
  = \xc{} & \xbc{} \\
  = \xd{} & \xcd{} \\
  = \xe{} & \xde{}
\end{conteq}

}

%}}}

\paragraph{PutGet}%{{{

\begin{conteq}
  \lensGet{ℒ}
    {
      \lensPut{ℒ}
        {a}
        {r}
    } \\
= \lensGet{ℒ_2}
    {
      \lensGet{ℒ_1}
        {
          \lensPut{ℒ}
            {a}
            {r}
        }
    }
& définition de \lensGetX{ℒ} \\
= \lensGet{ℒ_2}
    {
      \lensGet{ℒ_1}
        {
          \lensPut{ℒ_1}
            {
              \lensPut{ℒ_2}
                {a}
                { \lensGet{ℒ_1}{r} }
            }
            { r }
        }
    }
& définition de \lensPutX{ℒ} \\
= \lensGet{ℒ_2}
    {
      \lensPut{ℒ_2}
        {a}
        { \lensGet{ℒ_1}{r} }
    }
& \textsc{PutGet} sur $ℒ_1$ \\
= a
& \textsc{PutGet} sur $ℒ_2$ \\
\end{conteq}
%}}}

\paragraph{PutPut}%{{{

\begin{conteq}[onecolumn]
  \lensPut{ℒ}{a'}{
    \lensPut{ℒ}{a}{r}
  } \\
= \lensPut{ℒ}{a'}{
    \lensPut{ℒ_1}
      {
        \lensPut{ℒ_2}
          {a}
          { \lensGet{ℒ_1}{r} }
      }
      { r }
  }
& définition de \lensPutX{ℒ} \\
= \lensPut{ℒ_1}
    {
      \lensPut{ℒ_2}
        {a'}
        { \lensGet{ℒ_1}{
            \lensPut{ℒ_1}
              {
                \lensPut{ℒ_2}
                  {a}
                  { \lensGet{ℒ_1}{r} }
              }
              { r }
          }
        }
    }
    {
        \lensPut{ℒ_1}
          {
            \lensPut{ℒ_2}
              {a}
              { \lensGet{ℒ_1}{r} }
          }
          { r }
    }
& définition de \lensPutX{ℒ} \\
= \lensPut{ℒ_1}
    {
      \lensPut{ℒ_2}
        {a'}
        {
          \lensPut{ℒ_2}
            {a}
            { \lensGet{ℒ_1}{r} }
        }
    }
    {
        \lensPut{ℒ_1}
          {
            \lensPut{ℒ_2}
              {a}
              { \lensGet{ℒ_1}{r} }
          }
          { r }
    }
& \textsc{GetPut} sur $ℒ_1$ \\
= \lensPut{ℒ_1}
    {
      \lensPut{ℒ_2}
        {a'}
        { \lensGet{ℒ_1}{r} }
    }
    {
        \lensPut{ℒ_1}
          {
            \lensPut{ℒ_2}
              {a}
              { \lensGet{ℒ_1}{r} }
          }
          { r }
    }
& \textsc{PutPut} sur $ℒ_2$ \\
= \lensPut{ℒ_1}
    {
      \lensPut{ℒ_2}
        {a'}
        { \lensGet{ℒ_1}{r} }
    }
    { r
    }
& \textsc{PutPut} sur $ℒ_1$ \\
= \lensPut{ℒ}{a'}{r}
& définition de $\ggg$ \\
\end{conteq}
%}}}

\end{proof}

\section{Progrès}
\label{proof:progres}

On prouve de manière récursive mutuelle les
Théorèmes~\ref{thm:progres}
et~\ref{thm:progres-lv}.

% TODO mouais, à préciser mieux.

\begin{proof}

  On procède par induction sur la dérivation de $Γ ⊢ e : t$, et plus précisément
  sur la dernière règle utilisée.

  \paragraph{\textsc{Cst-Int}:} % {{{
$e$ est alors de la forme $n$, qui est une valeur.
%}}}
  \paragraph{\textsc{Cst-Float}:} % {{{
$e$ est alors de la forme $d$, qui est une valeur.
%}}}
  \paragraph{\textsc{Cst-Null}:} % {{{
$e$ est alors égale à $\eNull$, qui est une valeur.
%}}}
  \paragraph{\textsc{Cst-Unit}:}%{{{
$e$ est alors égale à \eUnit, qui est une valeur.
%}}}
\paragraph{\textsc{Lv-Var}:}%{{{


Puisque $(x, t) ∈ Γ$ et $\mcomp{Γ}{m}$,
on peut appliquer le Lemme~\ref{lemma:invsem}. La construction $a =
\mathrm{Lookup}(x, m)$ correspond bien à une adresse valide et on peut appliquer
\textsc{Phi-Var}.

% TODO mouais
%}}}
\paragraph{\textsc{Lv-Deref}:}%{{{

  Appliquons l'hypothèse de récurrence à $lv$ (vue en tant qu'expression).

\begin{itemize}
\item
  $lv = v$. Puisque $Γ ⊢ v : t*$, on déduit du
  Lemme~\ref{lemma:canon} que $v = φ$ ou $v = \eNull$.

  Dans le premier cas, la règle \textsc{Phi-Deref} s'applique:
  $\mms{e}{\widehat{*}φ}$.
  Dans le second, puisque $\msi{m}{*\eNull} → \serr{ptr}$, on a
  $\msi{m}{e} → \serr{ptr}$.

\item
  $\mm{m}{lv}{m'}{e'}$.
  De \textsc{Ctx} avec $C = *\ctxEmpty$, on obtient
  $\mm{m}{e}{m'}{*e'}$.

\item
  $\msi{m}{lv} → Ω$.
  En appliquant \textsc{Eval-Err} avec $C = *\ctxEmpty$, on obtient
  $\msi{m}{e} → Ω$.

\end{itemize}

% }}}
\paragraph{\textsc{Lv-Index}:} % %{{{

De même, on applique l'hypothèse de récurrence à $lv$.

\begin{itemize}
\item $lv = v$.

Comme $Γ ⊢ v : t[]$, on déduit du Lemme~\ref{lemma:canon} que
$v = \eArray{v_1; …; v_p}$.
Appliquons l'hypothèse de récurrence à $e$.

\begin{itemize}
\item $e = v'$. Puisque $Γ ⊢ e : \tInt$, on réapplique le
Lemme~\ref{lemma:canon} et $v' = n$.
D'après \textsc{Phi-Array}, $ \mms{lv[e]}{\eArray{v_1; …; v_p}\widehat{[n]}} $.
Deux cas sont à distinguer:
si $n ∈ [0;p-1]$, la partie droite vaut $v_{n+1}$ et donc
$\mm{m}{lv[e]}{m}{v_{n+1}}$.
Sinon elle vaut $\serr{array}$ et $\msi{m}{lv[e]} → \serr{array}$ par \textsc{Exp-Err}.

% TODO attention à l'off by one

\item $\mm{m}{e}{m'}{e'}$.
En appliquant \textsc{Ctx} avec $C = v[\ctxEmpty]$, on en déduit
\item $\mm{m}{lv[e]}{m'}{lv[e']}$.

\item $\msi{m}{e} → Ω$.
Avec \textsc{Eval-Err} sous ce même contexte,
$\msi{m}{lv[e]} → Ω$
\end{itemize}

\item $\mm{m}{lv}{m'}{e'}$.
On applique alors \textsc{Ctx} avec $C = \ctxEmpty[e]$, et
$\mm{m}{lv[e]}{m'}{e'[e]}$.

\item $\msi{m}{lv} → Ω$.
Toujours avec $C = \ctxEmpty[e]$, de \textsc{Eval-Err} il vient
$\msi{m}{lv[e]} → Ω$.

\end{itemize}
%}}}
\paragraph{\textsc{Lv-Field}:}%{{{

On applique l'hypothèse de récurrence du Théorème~\ref{thm:progres-lv} à $lv$.

\begin{itemize}

\item $lv = φ$
Alors \textsc{Phi-Struct} s'applique. Puisque $(l, t) ∈ S$, l'accès au champ $l$
ne provoque pas d'erreur $\serr{field}$. Donc $\mm{m}{e}{m}{φ[l]}$.

\item $\mm{m}{lv}{m'}{lv'}$
En appliquant \textsc{Ctx} avec $C = \ctxEmpty.l_S$, il vient
$\mm{m}{lv}{m'}{lv'}$.
% TODO Ctx-Lv en fait

\item $\msi{m}{lv} → Ω$
En appliquant \textsc{Eval-Err} avec $C = \ctxEmpty.l_S$, on a
$\msi{m}{lv} → Ω$.
% TODO Pas une sorte de Err-Lv ?

\end{itemize}

%}}}
  \paragraph{\textsc{Op-Int}:} % {{{

  Cela implique que $e = e_1~\opbin~e_2$. Par le Lemme~\ref{lemma:inversion}, on
  en déduit que $Γ ⊢ e_1 : \tInt$ et $Γ ⊢ e_2 : \tInt$.

  Appliquons l'hypothèse de récurrence sur $e_1$. Trois cas peuvent se produire.

\begin{itemize}

  \item $e_1 = v_1$. On a alors $\mm{m}{e_1}{m'}{v_1}$ avec $m' = m$.

    On applique l'hypothèse de récurrence à $e_2$.

      \begin{itemize}

        \item $e_2 = v_2$: alors $\mm{m'}{e_2}{m''}{v_2}$ avec $m'' = m$. On
          peut alors appliquer \textsc{Exp-BinOp}, sauf dans le cas d'une
          division par zéro ($ \opbin ∈ \{ / ; \% ; /. \} $ et
          $ v_2 = 0 $) où alors $v_1~\widehat{\opbin}~v_2 = \serr{div}$. Dans ce cas, on a
          alors par \textsc{Exp-Err} $\msi{m}{e} → \serr{div}$.

        \item $∃ (e'_2, m''), \mm{m'}{e_2}{m''}{e'_2}$.

          En appliquant \textsc{Ctx} avec $C = \ctxOp{v_1}{\ctxEmpty}$, on
          en déduit $\mm{m'}{v_1~\opbin~e_2}{m''}{v_1~\opbin~e'_2}$ soit
          $\mm{m}{e}{m''}{v_1~\opbin~e'_2}$.

        \item $\msi{m'}{e_2} → Ω$.
          De \textsc{Eval-Err} avec $C = \ctxOp{v_1}{\ctxEmpty}$
          vient alors $\msi{m}{e} → Ω$.

      \end{itemize}

  \item $∃(e_1', m'), \mm{m}{e_1}{m'}{e'_1}$.
    En appliquant \textsc{Ctx} avec $C = \ctxOp{\ctxEmpty}{e_2}$, on obtient
    $\mm{m}{e_1~\opbin~e_2}{m'}{e'_1~\opbin~e_2}$, ou
    $\mm{m}{e}{m'}{e'_1~\opbin~e_2}$.

  \item $\msi{m}{e_1} → Ω$.
    D'après \textsc{Eval-Err} avec $C = \ctxOp{\ctxEmpty}{e_2}$, on a
    $\msi{m}{e} → Ω$.

\end{itemize}

% }}}
\paragraph{\textsc{Op-Float}:} % {{{
Ce cas est similaire à \textsc{Op-Int}.
% TODO quid du lemme d'inversion (premiere ligne de Op-Int)?
%}}}
\paragraph{\textsc{Op-Eq}:} %{{{
Ce cas est similaire à \textsc{Op-Int}.
% TODO expand un peu
% TODO quid du lemme d'inversion (premiere ligne de Op-Int)?
%}}}
\paragraph{\textsc{Op-Comparable}:} %{{{
Ce cas est similaire à \textsc{Op-Int}.
% TODO expand un peu
% TODO quid du lemme d'inversion (premiere ligne de Op-Int)?
% TODO n'existe plus?
%}}}
\paragraph{\textsc{Unop-Plus-Int}:} % {{{

Alors $e = +~e_1$. En appliquant l'hypothèse d'induction sur $e_1$:

\begin{itemize}
\item
  soit $e_1 = v_1$. Alors en appliquant \textsc{Exp-UnOp},
  $\mm{m}{+~v_1}{m}{\widehat{+}~v_1}$, c'est-à-dire en posant $v =
  \widehat{+}~v_1$, $\mm{m}{e}{m}{v}$.
% TODO écrire la règle
\item
  soit $∃ e'_1, m', \mm{m}{e_1}{m'}{e'_1}$. Alors en appliquant \textsc{Ctx}
avec $C = +~\ctxEmpty$, on obtient $\mm{m}{e}{m'}{e'_1}$.
\item
  soit $\msi{m}{e_1} → Ω$.
  De \textsc{Eval-Err} avec $C = +~\ctxEmpty$ il vient$\msi{m}{e} → Ω$.
\end{itemize}

% }}}
\paragraph{\textsc{Unop-Plus-Float}:} % {{{
Ce cas est similaire à \textsc{Unop-Plus-Int}.
% }}}
\paragraph{\textsc{Unop-Minus-Int}:} % {{{
Ce cas est similaire à \textsc{Unop-Plus-Int}.
% }}}
\paragraph{\textsc{Unop-Minus-Float}:} % {{{
Ce cas est similaire à \textsc{Unop-Plus-Int}.
% }}}
\paragraph{\textsc{Unop-Not}:}%{{{
Ce cas est similaire à \textsc{Unop-Plus-Int}.
%}}}
\paragraph{\textsc{Ptr-Arith}:} % %{{{

Le schéma est similaire au cas \textsc{Op-Int}. Le seul cas intéressant arrive
lorsque $e_1$ et $e_2$ sont des valeurs. D'après le Lemme~\ref{lemma:canon}:

\begin{itemize}
\item $e_1 = \eNull$ ou $e_1 = φ$
\item $e_2 = n$
\end{itemize}

D'après \textsc{Exp-Binop}, $\mms{e}{e_1~\widehat{\opbin}~n}$.

On se réfère ensuite à la définition de $\widehat{\opbin}$
(page~\pageref{page:def-arith-ptr-error}): si $e_1$ est de la forme $φ[m]$,
alors $e_1~\widehat{\opbin}~n = φ[m+n]$. Donc $\mms{e}{φ[m+n]}$.

Dans les autres cas ($e_1 = \eNull$ ou $e_1 = φ$ avec $φ$ pas de la forme
$φ'[m]$), on a $e_1~\widehat{\opbin}~n = \serr{ptr}$. Donc d'après
\textsc{Exp-Err}, $\msi{m}{e} → \serr{ptr}$.

%}}}
\paragraph{\textsc{Addr}:} % {{{ TODO

\[ \left( \disprule{Addr} \right) \]

On applique l'hypothèse de récurrence du Théorème~\ref{thm:progres-lv} à $lv$.

Les cas d'évaluation et d'erreur sont traités en appliquant respectivement
\textsc{Ctx} et \textsc{Eval-Err} avec $C = \&\ctxEmpty$. On détaille le cas où
$lv =  φ$.

% TODO même remarques sur les règles que Lv-Field

\[ \left( \semrule{Exp-AddrOf} \right) \]

% TODO expliquer un peu mieux

%}}}
\paragraph{\textsc{Set}:} % {{{ TODO

\[ \left( \disprule{Set} \right) \]

%}}}
\paragraph{\textsc{Array}:} % {{{

On va appliquer l'hypothèse de récurrence à $e_1$, puis si $e_1 = v_1$, on
l'applique à $e_2$, etc. Alors on se retrouve dans un des cas suivants:

\begin{itemize}
\item $∃ p ∈ [1;n],e'_p,m: e_1 = v_1, …, e_{p-1} = v_{p-1}, \mm{m}{e_p}{m'}{e'_p}$.
  Alors on peut appliquer \textsc{Ctx} avec
  $C = [v_1, …, v_{p-1}, \ctxEmpty, e_{p+1}, …, e_n]$.
\item $∃ p ∈ [1;n],Ω :     e_1 = v_1, …, e_{p-1} = v_{p-1}, \msi{m}{e_p} → Ω$.
  Dans ce cas \textsc{Eval-Err} est applicable avec ce même $C$.
\item $e_1 = v_1, …, e_n = v_n$.
  Alors on peut appliquer \textsc{Exp-Array} en construisant un tableau.
\end{itemize}

%}}}
\paragraph{\textsc{Struct}:} % {{{

Le schéma de preuve est similaire au cas \textsc{Array}.
En cas de pas d'évaluation ou d'erreur, on utilise le contexte
$C = \eStruct{l_1: v_1, …, l_{p-1}: v_{p-1}, \ctxEmpty, l_{p+1}: e_{p+1}, …, l_n:
e_n}$; et dans le cas où toutes les expressions sont évaluées, on applique
\textsc{Exp-Struct}.

%}}}
\paragraph{\textsc{Call}:} % {{{

On commence par appliquer l'hypothèse de récurrence à $e$. Dans le cas d'un pas
d'évaluation ou d'erreur, on applique respectivement \textsc{Ctx} ou
\textsc{Eval-Err} avec $C = \ctxEmpty (e_1, …, e_n)$.
Reste le cas où $e$ est une valeur: d'après le Lemme~\ref{lemma:canon}, $e$ est
de la forme $f = \eFun{\vec{a}}{i}$.

Ensuite, appliquons le même schéma que pour \textsc{Array}.
En cas de pas d'évaluation ou d'erreur, on utilise
\textsc{Ctx} ou \textsc{Eval-Err} avec
$C = f (v_1, …, v_{p-1}, \ctxEmpty, e_{p+1}, …, e_n)$.
Le seul cas restant est celui où l'expression considérée a pour forme
$f (v_1, …, v_n)$
avec
$f = \eFun{\vec{a}}{i}$
:
\textsc{Exp-Call} s'applique alors.

% TODO cohérence mémoire

%}}}
\paragraph{\textsc{Fun}:} % {{{

Ce cas est direct: la règle \textsc{Exp-Fun} s'applique.

%}}}
% TODO vérifier qu'il n'y a pas d'autre règle
% TODO général: que désigne e. Ex avec lv[e]
% TODO prouver aussi mcomp
\end{proof}

\section{Préservation}
\label{proof:preservation}

(Théorème~\ref{thm:preservation})


\begin{proof}

On procède par induction sur la forme de l'expression $e$.

\paragraph{ Constante }             $c$ % {{{ TODO
On détaille par exemple le cas d'une constante entière.

D'une part, $Γ ⊢ e : t$ donc d'après le Lemme~\ref{lemma:canon}, $t = \tInt$.

D'autre part, la seule règle d'évaluation possible est \textsc{Exp-Cst} qui
évalue en une constante entière.

% TODO check si c'est vrai aussi pour Null
% TODO ajouter un typage des valeurs?

% }}}
\paragraph{ Accès mémoire }         $ lv $ % {{{ TODO

% }}}
\paragraph{ Opération unaire }      $ \opun~e $ % {{{ TODO


% }}}
\paragraph{ Opération binaire }     $ e~\opbin~e $ % {{{ TODO

% }}}
\paragraph{ Pointeur }              $ \& lv $ % {{{ TODO

% }}}
\paragraph{ Affectation }           $ lv ← e $ % {{{ TODO

% }}}
\paragraph{ Structure }             $ \eStruct{ l_1 : e_1 ; ; l_n : e_n } $ % {{{ TODO

% }}}
\paragraph{ Tableau }               $ \eArray{e_1 ;…; e_n} $ % {{{ TODO

% }}}
\paragraph{ Fonction }              $ f $ % {{{ TODO

% }}}
\paragraph{ Appel de fonction }     $ e (e_1, …, e_n) $ % {{{ TODO

% }}}

\end{proof}

%\section*{TODO}

%\begin{itemize}
%\item
  %titres dans les chapeaux buggés
%\end{itemize}

\section{Progrès pour les types qualifiés}
\label{proof:progres-qualif}

(Théorème~\ref{thm:progres-qual})

\section{Préservation pour les types qualifiés}
\label{proof:preservation-qualif}

(Théorème~\ref{thm:preservation-qualif})

  %% TODO
