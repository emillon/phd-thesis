On définit ici un langage impératif.

\section{Syntaxe}

\begin{figure}

  \gramlr{Constantes}{
    \begin{align*}
      c \gramisa & i & \textrm{Entier}
      \\ \gramor & f & \textrm{Flottant}
    \end{align*}
  }

  \gramlr{Expressions}{
    \begin{align*}
      e \gramisa & c               & \textrm{Constante}
      \\ \gramor & x               & \textrm{Variable}
      \\ \gramor & \opsymb~e       & \textrm{Opération unaire}
      \\ \gramor & e~\opsymb~e     & \textrm{Opération binaire}
      \\ \gramor & \& x            & \textrm{Pointeur}
      \\ \gramor & * e             & \textrm{Déréférencement}
      \\ \gramor & f (e_1, …, e_n) & \textrm{Appel de fonction}
    \end{align*}
  }

  \gramlr{Instructions}{
    \begin{align*}
      i \gramisa & \iPass        & \textrm{Instruction vide}
      \\ \gramor & i;i           & \textrm{Séquence}
      \\ \gramor & x ← e         & \textrm{Affectation}
      \\ \gramor & \iIf{e}{i}{i} & \textrm{Alternative}
      \\ \gramor & \iWhile{e}{i} & \textrm{Boucle}
      \\ \gramor & \iReturn{e}   & \textrm{Retour de fonction}
    \end{align*}
  }

  \gramlr{Fonctions}{
    \begin{align*}
      f \gramisa & (i, (x_1, …, x_n)) & \textrm{Corps et arguments}
    \end{align*}
  }

  \caption{Syntaxe}
  \label{fig:stx}
\end{figure}

\section{État mémoire}

\begin{figure}

  \gramlr{Valeurs}{
    \begin{align*}
      v \gramisa & c & \textrm{Constante}
      \\ \gramor & a & \textrm{Adresse}
    \end{align*}
  }

  \gramlr{Adresses}{
    \begin{align*}
      a \gramisa & (n, x) & \textrm{Variable locale}
      \\ \gramor & x      & \textrm{Variable globale}
    \end{align*}
  }

  \gramlr{Pile}{
    \begin{align*}
      s \gramisa & []                                   & \textrm{Pile vide}
      \\ \gramor & \{ x_1 ↦ v_1 ; … ; x_n ↦ v_n \} :: s & \textrm{Ajout d'un cadre}
    \end{align*}
  }

  \gramlr{État mémoire}{
    \begin{align*}
      m \gramisa & (s, \{ x_1 ↦ v_1 ; … ; x_n ↦ v_n \}) & \textrm{Pile, globales}
    \end{align*}
  }

  \caption{Interpréteur}
  \label{fig:interp}
\end{figure}

\[
  (x, v) ∈ m
\]

\[
  m[x ↦ v]
\]


\[
  \mathrm{Push}(s, (x_1, v_1), …, (x_n, v_n)) = \{ x_1 ↦ v_1 ; … ; x_n ↦ v_n \}~::~s
\]

\[
  \mathrm{Pop}(f~::~s) = s
\]

+ extension aux $m$.

\begin{definition}{Recherche de variable}

  \[
    \textrm{VarLookup} (x, (s, g)) =
      \begin{cases}
        x      & \mbox{si } x ∈ g \\
      (0, x)   & \mbox{si } x ∉ g \mbox{ et } s = f::[]
                 \mbox{ avec } x ∈ f \\
      (n+1, x) & \mbox{si } x ∉ g \mbox{ et } s = f::s'
                 \mbox{ avec } \textrm{VarLookup} (x, (s, g)) = (n, x)
      \end{cases}
  \]

\end{definition}

\section{Expressions}

Réduction (grand pas) :

\[
  (m, e) → (m', v)
\]

Constante :

\begin{mathpar}
  \irule{Exp-Cst}
    { }
    {(m, c) → (m, \widehat{c})}
\end{mathpar}

Cas variable :

\begin{mathpar}
  \irule{Exp-Var}
  {(x, v) ∈ m}
  {(m, x) → (m, v)}
\end{mathpar}

Cas op :

\begin{mathpar}
  \irule{Exp-UnOp}
  {(m, e) → (m', v) \\ v' = \widehat{\opsymb}~v}
  {(m, \opsymb~e) → (m', v') }
\end{mathpar}

À chaque opérateur $\opsymb$, correspond un opérateur sémantique
$\widehat{\opsymb}$ qui agit sur les valeurs. Par exemple, l'opérateur
$\widehat{+}$ est l'addition classique entre entiers.

\begin{mathpar}
  \irule{Exp-BinOp}
  {(m, e_1) → (m', v_1) \\
   (m', e_2) → (m'', v_2) \\
   v' = v_1~\widehat{\opsymb}~v_2
  }
  {
    (m, e_1~\opsymb~e_2) → (m'', v')
  }
\end{mathpar}

Adresse :

\begin{mathpar}
  \irule{Exp-AddrOf}
    {a = \textrm{VarLookup} (x, m)}
    {(m, \&x) → (m, a)}
\end{mathpar}

Déréférencement :

\begin{mathpar}
  \irule{Exp-AddrOf}
    { (m, e) → (m', a) \\
      (a, v) ∈ m'
    }
    {
      (m, *e) → (m', v)
    }
\end{mathpar}

Appel de fonction :

\[
  \irule{Call}
  {(m  , e_1) → (m_1, v_1) \\
   (m_1, e_2) → (m_2, v_2) \\
   … \\
   (m_{n-1}, e_n) → (m_n, v_n) \\
   f = (i, arg_1, …, arg_n) \\
   m' = \mathrm{Push}(m_n, (arg_1, v_1), …, (arg_n, v_n)) \\
   (m', i) → (m'', \iReturn{v}) \\
   m''' = \mathrm{Pop}(m'')
  }
  {(m, f(e_1, …, e_n)) → (m''', v)}
\]

\section{Instructions}

(attention au pas)

\[
  \irule{Set}
  {
    (m, e) → (m', v) \\
    m'' = m' [ x ↦ v ]
  }
  {(m, x ← e) → (m'', pass)}
\]

\begin{itemize}
\item
  ajout des adresses
\item
  declaration
\item
  champs de records
\end{itemize}
