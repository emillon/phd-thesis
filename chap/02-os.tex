Le système d'exploitation est le programme qui permet à un système informatique
d'exécuter d'autre programmes. Son rôle est donc capital et ses responsabilités
multiples. Dans ce chapitre, nous allons voir quel est son rôle, et comment il
peut être implanté. Pour ce faire, nous étudierons l'exemple d'une architecture
Intel 32 bits, et d'un noyau Linux 2.6.

Pour une description plus détaillée des rôles d'un système d'exploitation ainsi
que plusieurs cas d'étude détaillés, on pourra se référer à \cite{tanenbaum}.

\section{Rôle d'un système d'exploitation}

Un ordinateur est constitué de nombreux composants matériels : microprocesseur,
mémoire, et divers périphériques. Pourtant, au niveau de l'utilisateur, des
dizaines de logiciels permettent d'effectuer toutes sortes de calculs et de
communications. Le système d'exploitation permet de faire l'interface entre ces
niveaux d'abstraction.

Au cours de l'histoire des systèmes informatiques, la manière de les programmer
a beaucoup évolué. Au départ, les programmeurs avaient accès au matériel dans
son intégralité : toute la mémoire pouvait être accédée, toutes les instructions
pouvaient être utilisées.

Néanmoins c'est un peu restrictif, puisque cela ne permet qu'à une personne
d'interagir avec le système. Dans la seconde moitié des années 60, sont apparus
les premiers systèmes ``à temps partagé'', permettant à plusieurs utilisateurs
de travailler en même temps.

Permettre l'exécution de plusieurs programmes en même temps est une idée
révolutionnaire, mais elle n'est pas sans difficultés techniques : en effet les
ressources de la machine doivent être aussi partagées entre les utilisateurs et
les programmes. Par exemple, plusieurs programmes vont utiliser le processeur
les uns à la suite des autres (partage \emph{temporel}) ; et chaque programme
aura à sa disposition une partie de la mémoire principale, ou du disque dur
(partage \emph{spatial}).

Si deux programmes (ou plus) s'exécutent de manière concurrente sur le même
matériel, il faut s'assurer que l'un ne puisse pas écrire dans la mémoire de
l'autre, ou que les deux utilisent la carte réseau les uns à la suite des
autres. Ce sont des rôles du système d'exploitation.

Cela passe donc par une limitation des possibilités du programme : plutôt que de
permettre n'importe quel type d'instruction, il communique avec le système
d'exploitation. Celui-ci centralise donc les appels au matériel, ce qui permet
d'abstraire certaines opérations.

Par exemple, si un programme veut copier des données depuis un cédérom vers la
mémoire principale, il devra interroger le bus SATA, interroger le lecteur sur
la présence d'un disque dans le lecteur, activer le moteur, calculer le numéro
de trame des données sur le disque, demander la lecture, puis déclencher une
copie de la mémoire.

Si dans un autre cas il désire récupérer des données depuis une mémorette USB,
il devrait interroger le bus USB, rechercher le bon numéro de périphérique, le
bon numéro de canal dans celui-ci, lui appliquer une commande de lecture au bon
numéro de bloc, puis copier la mémoire.

Ces deux opérations, bien qu'elles aient le même but (copier de la mémoire
depuis un périphérique amovible), ne sont pas effectuées en pratique de la même
manière. C'est pourquoi le système d'exploitation fournit les notions de
fichier, lecteur, etc : le programmeur n'a plus qu'à utiliser des commandes de
haut niveau (``monter un lecteur'', ``ouvrir un fichier'', ``lire dans un
fichier'') et selon le type de lecteur, le système d'exploitation effectuera les
actions appropriées.

En résumé, un système d'exploitation est l'intermédiaire entre le logiciel et
le matériel, et en particulier assure les rôles suivants :

\todo{À affiner ou supprimer}

\begin{itemize}
\item
  Gestion des processus : un système d'exploitation peut permettre
  d'exécuter plusieurs programmes à la fois. Il faut alors orchestrer
  ces différents processus et les séparer en terme de temps et de
  ressources partagées.
\item
  Gestion de la mémoire : chaque processus, en plus du noyau, doit
  disposer d'un espace mémoire différent. C'est-à-dire qu'un processus
  ne doit pas pouvoir interférer avec un autre.
\item
  Gestion des fichiers : les processus peuvent accéder à une hiérarchie de
  fichiers, indépendamment de la manière d'y accéder.
\item
  Gestion des périphériques : le noyau étant le seul code ayant des privilèges,
  c'est lui qui doit communiquer avec les périphériques matériels.
\item
  Abstractions : le noyau fournit aux programmes une interface unifiée,
  permettant de stocker des informations de la même manière sur un
  disque dur ou une clef USB (alors que l'accès se déroulera de manière
  très différente en pratique).
\end{itemize}

\section{Architecture Intel}

L'implantation d'un système d'exploitation est très proche du matériel sur
lequel il s'exécute. Pour étudier une implantation en particulier, voyons ce que
permet le matériel lui-même.

Dans cette section nous décrivons le fonctionnement d'un processeur utilisant
une architecture Intel 32 bits. Les exemples de code seront écrits en syntaxe
AT\&T, celle que comprend l'assembleur GNU.

La référence pour la description de l'assembleur Intel est la documentation du
constructeur \cite{intelsys} ; une bonne explication de l'agencement dans la
pile peut aussi être trouvée dans \cite{SmashingTheStack}.

\subsection{Assembleur}

Pour faire des calculs, le processeur est composé de registres, qui sont des
petites zones de mémoire interne, et peut accéder à la mémoire principale.

La mémoire principale contient divers types des données :

\begin{itemize}
\item le code des programmes à exécuter
\item les données à disposition des programmes
\item la pile d'appels
\end{itemize}

La pile d'appels est une zone de mémoire qui est notamment utilisée pour tenir
une trace des calculs en cours. Par exemple, c'est ici que seront stockées les
données propres à chaque fonction appelée : paramètres, adresse de retour et
variables locales. La pile est manipulée par un pointeur de pile (\emph{stack
pointer}), qui est l'adresse du ``haut de la pile''. On peut la manipuler en
empilant des données (les placer au niveau du pointeur de pile et déplacer celui
si) ou dépilant des données (déplacer le pointeur de pile dans l'autre sens et
retourner la valeur présente à cet endroit).

L'état du processeur est défini par la valeur de ses registres, qui sont des
petites zones de mémoire interne (quelques bits chacun). Par exemple, la valeur
du pointeur de pile est stockée dans \esp. Le registre \ebp, couplé à \esp sert
à adresser les variables locales et paramètres d'une fonction, comme ce sera
expliqué dans la section~\ref{sec:convappel}.

L'adresse de l'instruction courante est accessible dans le registre \eip.

En plus de ces registres spéciaux, le processeur possède de nombreux registres
génériques, qui peuvent être utilisés pour réaliser des calculs intermédiaires.
Ils sont nommés \eax, \ebx, \ecx, \edx, \esi et \edi. Ils peuvent être utilisés
pour n'importe quel type d'opération, mais certains sont spécialisés : par
exemple il est plus efficace d'utiliser \eax en accumulateur, ou \ecx en
compteur.

Les calculs sont décrits sous forme d'une suite d'instructions. Chaque
instruction est composée d'un mnémonique et d'une liste d'opérandes. Les
mnémoniques (\asminstr{mov}, \asminstr{call}, \asminstr{sub}, etc) définissent
un type d'opération à appliquer sur les opérandes. L'instruction peut aussi être
précédée d'une étiquette, qui correspondra à son adresse.

\begin{center}
  \shorthandoff{!}
  \begin{tikzpicture}
    [bbr/.style={bigbrace, decoration={mirror, amplitude=2mm}}
    ,asmbox/.style={minimum height=1cm}
    ,dway/.style={dashed}
    ]

    \node[asmbox]              (lbl) {\asminstr{lbl:}};
    \node[asmbox,right of=lbl] (mov) {\asminstr{mov}};
    \node[asmbox,right of=mov] (op1) {\asminstr{\$4,}};
    \node[asmbox,right of=op1] (op2) {\asminstr{\%ebx}};

    \draw[bbr] (lbl.south west) -- (lbl.south east);
    \draw[bbr] (mov.south west) -- (mov.south east);
    \draw[bbr] (op1.south west) -- (op2.south east);

    \coordinate (ops) at ($ (op1.south west)!.5!(op2.south east) $);

    \node[below left of=lbl, xshift=-1cm, yshift=-5mm]   (tlbl) {Étiquette};
    \node[below of=tlbl, node distance=7mm] (tmov) {Mnémonique};
    \node[below of=tmov, node distance=7mm] (tops) {Opérandes};

    \draw[dway] (tlbl) -| ($ (lbl.south) + (0, -0.3) $);
    \draw[dway] (tmov) -| ($ (mov.south) + (0, -0.3) $);
    \draw[dway] (tops) -| ($ (ops.south) + (0, -0.3) $);

  \end{tikzpicture}
  \shorthandon{!}
\end{center}

Ces opérandes peuvent être de plusieurs types :

\begin{itemize}
\item
  un nombre, noté \$4
\item
  le nom d'un registre, noté \texttt{\%eax}
\item
  une opérande mémoire, c'est à dire le contenu de la mémoire à une
  adresse effective. Cette adresse effective peut être exprimée de
  plusieurs manières :
  \begin{itemize}
  \item
    directement : \texttt{addr}
  \item
    indirectement : \texttt{(\%ecx)}. L'adresse effective est le contenu
    du registre.
  \item
    ``base + déplacement'' : \texttt{4(\%ecx)}. L'adresse effective est
    le contenu du registre plus le déplacement (4 ici).
  \end{itemize}
\end{itemize}

En pratique il y a des modes d'adressage plus complexes, et toutes les
combinaisons ne sont pas possibles, mais ceux-ci suffiront à décrire les
exemples suivants :

\begin{itemize}

\item \texttt{mov src, dst} copie le contenu de \texttt{src} dans \texttt{dst}.

\item \texttt{add src, dst} calcule la somme des contenus de \texttt{src} et
  \texttt{dst} et place ce résultat dans \texttt{dst}.

\item \texttt{push src} place \texttt{src} sur la pile, c'est à dire que cette
  instruction enlève au pointeur de pile \esp la taille de \texttt{src}, puis
  place \texttt{src} à l'adresse mémoire de la nouvelle valeur \esp.

\item \texttt{pop src} réalise l'opération inverse : elle charge le contenu de
  la mémoire à l'adresse \esp dans \texttt{src} puis incrémente \esp de la
  taille correspondante.

\item \texttt{jmp addr} saute à l'adresse \texttt{addr} : c'est l'équivalent de
  \texttt{mov addr, \%eip}.

\item \texttt{call addr} sert aux appels de fonction : cela revient à
  \texttt{push \%eip} puis \texttt{jmp addr}.

\item \texttt{ret} sert à revenir d'une fonction : c'est l'équivalent de
  \texttt{pop \%eip}.

\end{itemize}

\subsection{Fonctions et conventions d'appel}
\label{sec:convappel}

\begin{figure} % {{{ fig:stackframe
\begin{tikzpicture}
[ stack/.style={draw,shape=rectangle,minimum height=1cm,minimum width=5cm}
, bigbrace/.style={decorate,decoration={brace,amplitude=3mm}}
, >=latex
]

  \path (0,-8.5)   node [stack, minimum height=2cm] (gloc) {Locales de g}
  -- ++(0,1.5) node [stack] (gebp) {Cadre précédent}
  -- ++(0,1)   node [stack] (gret) {Retour de g}
  -- ++(0,1.5) node [stack, minimum height=2cm] (gpar) {Arguments de g}
  -- ++(0,1.5) node [stack] (floc) {Locales de f}
  -- ++(0,1)   node [stack] (febp) {Cadre précédent}
  -- ++(0,1)   node [stack] (fret) {Retour de f}
  -- ++(0,1)   node [stack] (fpar) {Arguments de f};

  \draw [->] (gebp) -- ++(4, 0) |- (febp.355);
  \draw [->] (febp.5) -- ++(1.5, 0) -- ++(0,4);

  \draw [dashed] (fpar.north west) -- ++ (0,2);
  \draw [dashed] (fpar.north east) -- ++ (0,2);
  \draw [dashed] (gloc.south east) -- ++ (0,-2);
  \draw [dashed] (gloc.south west) -- ++ (0,-2);

  \node at (5, 1) (ox) {};

  \node [text width=3cm] at (-0.8, 2)     {Adresses hautes (sens de \asminstr{pop})};
  \node [text width=3cm] at (-0.8, -10.5) {Adresses basses (sens de \asminstr{push})};

  \node [text width=3cm] at (-6, -3) (asmtext) {\tt
    call f\\
    ... \\
    f:  ... \\
    pushl \$2 \\
    pushl \$4 \\
    call g \\
    addl \$8, \%esp \\
    leave \\
    ret \\
    ... \\
    g: push \%ebp \\
    mov \%esp, \%ebp \\
    sub \$8, \%esp \\
    ... \\
    leave \\
    ret
    };

  \draw [->] (gret) -- ++ (-3, 0) |- ($ (asmtext.north west) + (2.5, -3) $);
  \draw [->] (fret) -- ++ (-3, 0) |- ($ (asmtext.north west) + (2.5, -0.5) $);

  \draw [<-] (gebp.west) -- ++ (-0.5,0) node [xshift=-4mm] {\ebp};
  \draw [<-] ($ (gloc.west) + (0,-0.5) $) -- ++ (-0.5,0) node [xshift=-4mm] {\esp};

  \draw [<-] ($ (asmtext.north west) + (2.5, -6.7) $) -- ++ (0.5,0) node [xshift=3mm] {\eip};

  \draw [bigbrace]
        ($ (gpar.north east) + (3, -0.1) $)
     -- node [auto,midway,xshift=3mm] {Cadre de g}
        ($ (gloc.south east) + (3, 0) $);

  \draw [bigbrace]
        ($ (fpar.north east) + (3, 0) $)
     -- node [auto,midway,xshift=3mm] {Cadre de f}
        ($ (floc.south east) + (3, 0.1) $);

\end{tikzpicture}

\caption[Cadres de pile]{ Cadres de pile. }
\label{fig:stackframe}
\end{figure} % }}}

\todo{Mettre des vrais nombres plutôt que du symbolique}

Dans le langage d'assemblage, il n'y a pas de notion de fonction ; mais
\asminstr{call} et \asminstr{ret} permettent de sauvegarder et de restaurer une
adresse de retour, ce qui permet de faire un saut et revenir à l'adresse initiale.
Ce système permet déjà de créer des procédures, c'est-à-dire des fonctions sans
arguments ni valeur de retour.

Pour gérer ceux-ci, il faut que les deux morceaux (appelant et appelé) se
mettent d'accord sur une convention d'appel commune. La convention utilisée sous
GNU/Linux est appelée \emph{cdecl} et possède les caractéristiques suivantes :

\begin{itemize}
\item la valeur de retour d'une fonction est stockée dans \eax
\item \eax, \ecx et \edx peuvent être écrasés sans avoir à les sauvegarder
\item les arguments sont placés sur la pile (et enlevés) par l'appelant. Les
  paramètres sont empilés de droite à gauche.
\end{itemize}

Pour accéder à ses paramètres, une fonction peut donc utiliser un adressage
relatif à \esp. Cela peut fonctionner, mais cela complique les choses si elle
contient aussi des variables locales. En effet, les variables locales sont
placées sur la pile, au dessus des (c'est à dire, empilées après) paramètres,
augmentant le décalage.

Pour simplifier, la pile est organisée en cadres logiques : chaque cadre
correspond à un niveau dans la pile d'appels de fonctions. Si f appelle g, qui
appelle h, il y aura dans l'ordre sur la pile le cadre de f, celui de g puis
celui de h.

Ces cadres sont chainés à l'aide du registre \ebp : à tout moment, \ebp contient
l'adresse du cadre de l'appelant.

Prenons exemple sur la figure~\ref{fig:stackframe} : pour appeler
\texttt{g(4,2)}, \texttt{f} empile les arguments de droite à gauche.
L'instruction \asminstr{call g} empile l'adresse de l'instruction suivante sur
la pile puis saute au début de \texttt{g}.

Au début de la fonction, les trois instructions permettent de sauvegarder
l'ancienne valeur de \ebp, faire pointer \ebp à une endroit fixe dans le cadre
de pile, puis allouer 8 octets de mémoire pour les variables locales.

Dans le corps de la fonction \texttt{g}, on peut donc se référer aux variables
locales par \texttt{-4(\%ebp)}, \texttt{-8(\%ebp)}, etc, et aux arguments par
  \texttt{8(\%ebp)}, \texttt{12(\%ebp)}, etc.

À la fin de la fonction, l'instruction \asminstr{leave} est équivalente à
\texttt{mov \%ebp, \%esp} puis \texttt{pop \%ebp} et permet de défaire le cadre
de pile, laissant l'adresse de retour en haut de pile. Le \asminstr{ret} final
la dépile et y saute.

\subsection{Tâches, niveaux de privilèges}
\label{sec:taches}

Sans mécanisme particulier, le processeur exécuterait uniquement une suite
d'instruction à la fois. Pour lui permettre d'exécuter plusieurs tâches, un
système de partage du temps existe.

À des intervalles de temps réguliers, le système est programmé pour recevoir une
interruption. C'est une condition exceptionnelle (au même titre qu'une division
par zéro) qui fait sauter automatiquement le processeur dans une routine de
traitement d'interruption. À cet endroit le code peut sauvegarder les registres
et restaurer un autre ensemble de registres, ce qui permet d'exécuter plusieurs
tâches de manière entrelacée. Si l'alternance est assez rapide, cela peut donner
l'illusion que les programmes s'exécutent en parallèle. Comme l'interruption
peut survenir à tout moment, on parle de multitâche préemptif.

En plus de cet ordonnancement de processus, l'architecture Intel permet
d'affecter des niveaux de privilège à ces tâches, en restreignant le type
d'instructions exécutables, ou en donnant un accès limité à la mémoire aux
tâches de niveaux moins élevés.

\begin{figure}
\begin{Verbatim}
+-------------------------------------+
|   Ring 3                            |
|                                     |
|       Programmes utilisateur        |
|                                     |
|                                     |
|   ^ | appels système                |
+---|-v-------------------------------+
|                                     |
|   Ring 0                            |
|                                     |
|       Noyau                         |
|                                     |
|   ^ | instructions privilégiées     |
+---|-v-------------------------------+

   Hardware
\end{Verbatim}

\begin{tikzpicture}

\begin{scope}

  %\path[clip] (0, 0.8) rectangle (5, -0.8);

  \foreach \x/\c in { 4/20, 3/30, 2/40, 1/50, 0/60 }
  \draw[fill=black!\c] ($ (-2,0) + (\x,0) $) circle (2 cm);

\end{scope}

%\draw [<-] (3.5, 0) -- ++(0,-2) node {Applications};
%\draw [<-] (0, 0) -- ++(0,-2) node {Noyau};

\end{tikzpicture}

\caption[Les différents \emph{rings}]{
  Les différents \emph{rings}. Seul le \ring{0} a accès au hardware
  via des instructions privilégiées. Pour accéder aux fonctionnalités du noyau,
  les programmes utilisateur doivent passer par des appels système.}
\label{fig:rings}
\end{figure}

Il y a 4 niveaux de privilèges (nommés aussi \emph{rings} -
figure~\ref{fig:rings}) : le \ring{0} est le plus privilégié, le \ring{3} le
moins privilégié. Dans l'exemple précédent, on pourrait isoler l'ordonnanceur de
processus en le faisant s'exécuter en \ring{0} alors que les autres tâches
seraient en \ring{3}.

\subsection{Mémoire virtuelle}

À partir du moment où plusieurs processus s'exécutent de manière concurrente, un
problème d'isolation se pose : si un processus peut lire dans la mémoire d'un
autre, des informations peuvent fuiter ; et s'il peut y écrire, il peut en
détourner l'exécution.

Le mécanisme de mémoire virtuelle permet de donner à deux tâches une vue
différente de la mémoire : c'est à dire que vue de tâches différentes, une
adresse contiendra une valeur différente.

\begin{figure}
\begin{tikzpicture}
[ bit/.style={ draw
             , shape=rectangle
             , minimum width=4mm
             , minimum height=4mm
             , node distance=4mm
             }
, pstuff/.style={ draw
                , shape=rectangle
                , node distance=1cm
                , text width=2cm
                , text centered
                }
, >=latex
]

\node [bit] (b31) {};

% generate this ?
\node [bit, right of=b31] (b30) {}; \node [bit, right of=b30] (b29) {}; \node [bit, right of=b29] (b28) {};
\node [bit, right of=b28] (b27) {}; \node [bit, right of=b27] (b26) {}; \node [bit, right of=b26] (b25) {};
\node [bit, right of=b25] (b24) {}; \node [bit, right of=b24] (b23) {}; \node [bit, right of=b23] (b22) {};
\node [bit, right of=b22] (b21) {}; \node [bit, right of=b21] (b20) {}; \node [bit, right of=b20] (b19) {};
\node [bit, right of=b19] (b18) {}; \node [bit, right of=b18] (b17) {}; \node [bit, right of=b17] (b16) {};
\node [bit, right of=b16] (b15) {}; \node [bit, right of=b15] (b14) {}; \node [bit, right of=b14] (b13) {};
\node [bit, right of=b13] (b12) {}; \node [bit, right of=b12] (b11) {}; \node [bit, right of=b11] (b10) {};
\node [bit, right of=b10] (b9) {}; \node [bit, right of=b9] (b8) {}; \node [bit, right of=b8] (b7) {};
\node [bit, right of=b7] (b6) {}; \node [bit, right of=b6] (b5) {}; \node [bit, right of=b5] (b4) {};
\node [bit, right of=b4] (b3) {}; \node [bit, right of=b3] (b2) {}; \node [bit, right of=b2] (b1) {};
\node [bit, right of=b1] (b0) {};

\foreach \b in {0,11,12,21,22,31}
    \node [above of=b\b, node distance=5mm] {\small \b};

\newcommand{\mbrace}[3]{
    \draw [bigbrace]                      ($ (b#1.south east) + (-0.1,-0.3) $)
    -- node [midway, yshift=-2mm] (#3) {} ($ (b#2.south west) +  (0.1,-0.3) $);
} % mbrace

\mbrace{0}{11}{bpp};
\mbrace{12}{21}{bpt};
\mbrace{22}{31}{bpd};

\node[pstuff, below of=bpd] (pd) {Répertoire de pages};
\node[pstuff, below of=bpt] (pt) {Table de pages};
\node[pstuff, below of=bpp] (pp) {Page de 4 Kio};

\node[left of=pd, node distance=2cm] (cr) {\crtrois};

\draw [->] (bpp) -- (pp);
\draw [->] (bpt) -- (pt);
\draw [->] (bpd) -- (pd);

\draw [->] (cr) -- (pd);
\draw [->] (pd) -- (pt);
\draw [->] (pt) -- (pp);
\draw [->] (pp) -- ++ (3,0);

\end{tikzpicture}

\caption{Implantation de la mémoire virtuelle}
\label{fig:pagetables}
\end{figure}

Ce mécanisme est contrôlé par valeur du registre \crtrois : les 10 premiers bits
d'une adresse virtuelle sont un index dans le répertoire de pages qui commence à
l'adresse contenue dans \crtrois. À cet index, se trouve l'adresse d'une table
de pages. Les 10 bits suivants de l'adresse sont un index dans cette page,
donnant l'adresse d'une page de 4 kibioctets (figure~\ref{fig:pagetables}).

\begin{figure} % fig:memoire-virtuelle {{{
\centering
\begin{tikzpicture}
  [every node/.style={node distance=2cm}]

  \node (u1) {};

  \node[right of=u1] (phy) {};

  \node[right of=phy] (u2) {};

  \draw (u1) rectangle ++(1,-6);
  \draw (phy) rectangle ++(1,-6);
  \draw (u2) rectangle ++(1,-6);
\end{tikzpicture}
\caption{Mécanisme de mémoire virtuelle.}
\label{fig:memoire-virtuelle}
\end{figure} % }}}

\todo{Faire cette figure}

En ce qui concerne la mémoire, les différentes tâches ont une vision différente
de la mémoire physique : c'est-à-dire que deux tâches lisant à une même adresse
peuvent avoir un résultat différent. C'est le concept de mémoire virtuelle
(fig~\ref{fig:memoire-virtuelle}).

\todo{Redite qui n'apporte pas plus d'explication}

\section{Cas de Linux}

Dans cette section, nous allons voir comment ces mécanismes sont implantés dans
le noyau Linux. Une description plus détaillée pourra être trouvée dans
\cite{UnderstandingTheLinuxKernel}, ou pour le cas de la mémoire virtuelle,
\cite{LinuxVMM}.

Deux rings sont utilisés : en \ring{0}, le code noyau et en \ring{3}, le code
utilisateur.

Une notion de tâche similaire à celle décrite dans la section~\ref{sec:taches}
existe : elles s'exécutent l'une après l'autre, le changement s'effectuant sur
interruptions.

Pour faire appel aux services du noyau, le code utilisateur doit faire appel à
des appels systèmes, qui sont des fonctions exécutées par le noyau. Chaque tâche
doit donc avoir deux piles : une pile ``utilisateur'' qui sert pour
l'application elle-même, et une pile ``noyau'' qui sert aux appels système.

\begin{figure} % fig:memmap {{{
\centering
\fbox{
  \begin{tikzpicture}
  [scale=0.7
  ,user/.style={fill=black!20}
  ,kernel/.style={fill=black!70}
  ]

  % Memory zone
  %
  % #1 - start
  % #2 - end
  % #3 - color
  \newcommand{\mzone}[3]{
    \path[#3] (#1,0) rectangle (#2,1);
  }

  % Address label
  %
  % #1 - x position
  % #2 - text
  \newcommand{\alabel}[2]{
    \path (#1,1) -- ++(0,0.3) node [pos=1] {\small \tt #2};

  }

  % exec
  \mzone{0.5}{1}{user}

  % lib
  \mzone{2.5}{3.2}{user}

  % stack
  \mzone{3.7}{4}{user}

  % stack
  \mzone{5}{5.5}{user}

  % kernel
  \mzone{6}{8}{kernel}

  % contour
  \draw (0,0) rectangle (8,1);

  \alabel{0}{0}
  \alabel{6}{3 Go}
  \alabel{8}{4 Go}

\end{tikzpicture}

}

\caption[Espace d'adressage d'un processus]{L'espace d'adressage d'un processus.
En gris clair, les zones accessibles à tous les niveaux de privilèges : code du
programme, bibliothèques, tas, pile. En gris foncé, la mémoire du noyau,
réservée au mode privilégié.}

\label{fig:memmap}
\end{figure}
% }}}

Grâce à la mémoire virtuelle, chaque processus possède sa propre vue de la
mémoire dans son espace d'adressage (figure~\ref{fig:memmap}), et donc chacun
un ensemble de tables de pages et une valeur de \crtrois associée. Au moment de
changer le processus en cours, l'ordonnanceur charge donc le \crtrois du nouveau
processus.

Les adresses basses (inférieures à \texttt{PAGE\_OFFSET} = 3 Gio =
\texttt{0xc0000000}) sont réservées à l'utilisateur. On y trouvera par exemple :

\begin{itemize}
\item le code du programme
\item les données du programmes (variables globales)
\item la pile utilisateur
\item le tas (mémoire allouée par \texttt{malloc} et fonctions similaires)
\item les bibliothèques partagées
\end{itemize}

Au dessus de \texttt{PAGE\_OFFSET}, se trouve la mémoire réservée au noyau.
Cette zone contient le code du noyau, les piles noyau des processus, etc.

\subsection{Appels système}
\label{sec:impl-syscall}

\todo{clarifier encore tout ça}

Les programmes utilisateur s'exécutant en \ring{3}, ils ne peuvent pas
contenir d'instructions privilégiées, et donc ne peuvent pas accéder directement
au matériel (c'était le but !). Pour qu'ils puissent interagir avec le système
(afficher une sortie, écrire sur le disque...), le mécanisme des appels système
est nécessaire. Il s'agit d'une interface de haut niveau entre les \emph{rings}
3 et 0. Du point de vue du programmeur, il s'agit d'un ensemble de fonctions C
``magiques'' qui font appel au système d'exploitation pour effectuer des
opérations.

Voyons ce qui se passe derrière la magie apparente. Une explication plus
détaillée est disponible dans la documentation fournie par Intel
\cite{intelsys}.

\subsubsection{Dans la bibliothèque C}

Il y a bien une fonction \texttt{getpid} présente dans la bibliothèque C du
système. C'est la fonction qui est directement appelée par le programme. Cette
fonction commence par placer le numéro de l'appel système (noté
\texttt{\_\_NR\_getpid}, valant 20 ici) dans \eax, puis les arguments éventuels
dans les registres (\ebx, \ecx, \edx, \esi puis \edi). Une interruption
logicielle est ensuite déclenchée (\texttt{int 0x80}).

\subsubsection{Dans la routine de traitement d'interruption}

Étant donné la configuration du processeur\footnote{Il est impropre de dire que
le processeur est configuré --- tout dépend uniquement de l'état de certains
registres, ici la \emph{Global Descriptor Table} et les \emph{Interrupt
Descriptor Tables}.}, elle sera traitée en \ring{0}, à un point d'entrée
prédéfini (\texttt{arch/x86/kernel/entry\_32.S}, \texttt{ENTRY(system\_call)}).

\insertcode{entry-syscall.s}

L'exécution reprend donc en \ring{0}, avec dans \esp le pointeur de pile noyau
du processus. Les valeurs des registres ont été préservées, la macro
\texttt{SAVE\_ALL} les place sur la pile. Ensuite, à l'étiquette
\texttt{syscall\_call}, le numéro d'appel système (toujours dans \eax) sert
d'index dans le tableau de fonctions \texttt{sys\_call\_table}.


\subsubsection{Dans l'implantation de l'appel système}

Puisque les arguments sont en place sur la pile, comme dans le cas d'un appel de
fonction ``classique'', la convention d'appel \emph{cdecl} est respectée. La
fonction implantant l'appel système, nommée \texttt{sys\_getpid}, peut donc être
écrite en C.

On trouve cette fonction dans \texttt{kernel/timer.c} :

\insertcode{syscall-definition.c}

La macro \texttt{SYSCALL_DEFINE0} nomme la fonction \texttt{sys\_getpid}, et
définit entre autres des points d'entrée pour les fonctionnalités de débogage du
noyau. À la fin de la fonction, la valeur de retour est placée dans \eax,
conformément à la convention \emph{cdecl}.

\subsubsection{Retour vers le ring 3}

Au retour de la fonction, la valeur de retour est placée à la place de \eax là
où les registres ont été sauvegardés sur la pile noyau
(\texttt{PT\_EFLAGS(\%esp)}). L'instruction \texttt{iret} (derrière la macro
\texttt{INTERRUPT\_RETURN}) permet de restaurer les registres et de repasser en
mode utilisateur, juste après l'interruption. La fonction de la bibliothèque C
peut alors retourner au programme appelant.

\section{Sécurité des appels système}
\label{sec:secu-syscalls}

On a vu que les appels systèmes permettent aux programmes utilisateur d'accéder
au services du noyau. Ils forment donc une interface particulièrement sensible aux
problèmes de sécurité.

Comme pour toutes les interfaces, on peut être plus ou moins fin. D'un
côté, une interface pas assez fine serait trop restrictive et ne permettrait pas
d'implémenter tout type de logiciel. De l'autre, une interface trop laxiste
(``écrire dans tel registre matériel'') empêche toute isolation. Il faut
donc trouver la bonne granularité.

Nous allons présenter ici une difficulté liée à la manipulation de mémoire au
sein de certains types d'appels système.

Il y a deux grands types d'appels systèmes : d'une part, ceux qui renvoient un
simple nombre, comme \texttt{getpid} qui renvoie le numéro du processus
appelant.

\insertcode{appel-syscall-getpid.c}

Ici, pas de difficulté particulière : la communication entre le \ring{0} et le
\ring{3} est faite uniquement à travers les registres, comme décrit dans la
section~\ref{sec:impl-syscall}.

\begin{figure}
\insertcode{appel-syscall-gettimeofday.c}
\caption{Appel de gettimeofday}
\label{fig:appel-gettimeofday}
\end{figure}

Mais la plupart des appels systèmes communiquent de l'information de manière
indirecte, à travers un pointeur. L'appellant alloue une zone mémoire dans son
espace d'adressage et passe un pointeur à l'appel système. Ce mécanisme est
utilisé par exemple par la fonction \texttt{gettimeofday}
(figure~\ref{fig:appel-gettimeofday}).

% Fig memory zones (needed ?)
% Memory zone
%
% #1 - start
% #2 - end
% #3 - color
\newcommand{\mzone}[3]{
  \path[#3] (#1,0) rectangle (#2,1);
}

% Address label
%
% #1 - x position
% #2 - text
\newcommand{\alabel}[2]{
  \path (#1,1) -- ++(0,0.3) node [pos=1] {\small \ttfamily #2};

}

% Pointer
%
% #1 - start address
% #2 - label
\newcommand{\ptr}[2]{
  \path [draw,open triangle 60-] (#1, -0.2) -- ++ (0,-0.5) node [below] {#2};
}

\newcommand{\memoryframe}[1]{%
\begin{tikzpicture}
  [user/.style={fill=green!40}
  ,kernel/.style={fill=red!40}
  ]

  % exec
  \mzone{0.5}{1}{user}

  % lib
  \mzone{2.5}{3.2}{user}

  % stack
  \mzone{3.7}{4}{user}

  % stack
  \mzone{5}{5.5}{user}

  % kernel
  \mzone{6}{8}{kernel}

  % contour
  \draw (0,0) rectangle (8,1);

  \alabel{0}{0}
  \alabel{6}{3 Go}
  \alabel{8}{4 Go}

  % User struct
  \ifthenelse{#1=1}{\mzone{2.6}{3.1}{draw, fill=green}}{}
  \ifthenelse{#1=1}{\ptr{2.6}{pu}}{}

  \ifthenelse{#1=2}{\mzone{7}{7.5}{draw, fill=red}}{}
  \ifthenelse{#1>1}{\ptr{7}{pk}}{}

\end{tikzpicture}
} % End of newcommand \memoryframe

 % Fig mzones {{{
\begin{figure}
    \centering
    \subfloat[][]{
        \label{fig:mzone1}
        \memoryframe{1}
    }

    \subfloat[][]{
        \label{fig:mzone2}
        \memoryframe{2}
    }

    \subfloat[][]{
        \label{fig:mzone3}
        \memoryframe{3}
    }

    \caption{Zones mémoire}
    \label{fig:mzones}
\end{figure} % }}}

Considérons une implémentation naïve de cet appel système qui écrirait
directement à l'adresse pointée. La figure~\ref{fig:mzone1} présente ce qui se
passe lorsque le pointeur fourni est dans l'espace d'adressage du processus :
c'est le cas d'utilisation normal et l'écriture est donc possible.

Si l'utilisateur passe un pointeur dont la valeur est supérieure à
\texttt{0xc0000000} (figure~\ref{fig:mzone2}), que se passe-t'il ? Comme le
déréférencement est fait dans le code du noyau, il est également fait en
\ring{0}, et va pouvoir être réalisé sans erreur : l'écriture se fait et
potentiellement une structure importante du noyau est écrasée.

Un utilisateur malicieux peut donc utiliser cet appel système pour écrire à
n'importe quelle adresse dans l'espace d'adressage du noyau. Ce problème vient
du fait que l'appel système utilise les privilèges du noyau au lieu de celui qui
contrôle la valeur des paramètres sensibles. Celà s'appelle le \emph{Confused
Deputy Problem}\cite{hardy88confused}.

La bonne solution est de tester dynamiquement la valeur du pointeur : si la
valeur du pointeur est supérieure à \texttt{0xc0000000}, il faut indiquer une
erreur avant d'écrire (figure~\ref{fig:mzone3}. Sinon, cela ne veut pas dire
que le déréférencement se fera sans erreur, mais au moins le noyau est protégé.

\begin{figure}
  \insertcode{implem-gettime.c}
  \caption{Implantation de l'appel système \texttt{gettimeofday}}
  \label{fig:implem-gettime}
\end{figure}

Dans le noyau, un ensemble de fonctions permet d'effectuer des copies sûres. La
fonction \texttt{access\_ok} réalise le test décrit précédemment. Les fonctions
\texttt{copy\_from\_user} et \texttt{copy\_to\_user} réalisent une copie de la
mémoire après avoir fait ce test. Ainsi, l'implantation correcte de l'appel
système \texttt{gettimeofday} fait appel à celle-ci
(figure~\ref{fig:implem-gettime}).

Pour préserver la sécurité du noyau, il est donc nécessaire de vérifier la
valeur de tous les pointeurs dont la valeur est contrôlée par l'utilisateur.
Cette conclusion est assez contraignante, puisqu'il existe de nombreux endroits
dans le noyau où des données proviennent de l'utilisateur. Il est donc
raisonnable de vouloir vérifier automatiquement et statiquement l'absence de
tels défauts.

% vim: spelllang=fr
