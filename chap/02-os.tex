Le système d'exploitation est le programme qui permet à un système informatique
d'exécuter d'autres programmes. Son rôle est donc capital et ses responsabilités
multiples. Dans ce chapitre, nous allons voir à quoi il sert, et comment il peut
être implanté. Pour ce faire, nous présentons ici de quoi est constitué un
sustème Intel 32 bits et ce dont on se sert pour y implanter un système
d'exploitation.

\section{Architecture physique}

Un système informatique est principalement constitué d'un processeur (ou CPU
pour \emph{Central Processing Unit}), de mémoire principale (ou RAM pour
\emph{Random Access Memory}), et de divers périphériques.

Le processeur est constitué de divers registres internes qui permettent
d'encoder l'état dans lequel il se trouve : quelle est l'instruction courante
(registre \eip), quelle est la hauteur de la pile système (registre \esp), etc.
Son fonctionnement peut être vu de la manière la plus simple qui soit comme le
cycle :

\begin{itemize}
\item
  charger depuis la mémoire la prochaine instruction
\item
  \emph{(optionnel)} charger depuis la mémoire les données référencées
  par l'instruction
\item
  effectuer l'instruction
\item
  \emph{(optionnel)} stocker en la mémoire les données modifiées
\end{itemize}

Les instructions sont constituées d'un mnémonique (indicant quelle opération
faire) et d'un ensemble d'opérandes. La signification des opérandes dépend du
mnémonique, mais en général elles permettent de désigner les sources et la
destination. Les opérandes peuvent avoir plusieurs formes : une valeur
immédiate (\$4), un nom de registre (\texttt{\%eax}) ou une référence à la
mémoire (directement : \texttt{addr} ou indirectement : \texttt{(\%ecx)}
\footnote{
  Cela consiste à interpréter le contenu du regitre \ecx comme une adresse
  mémoire.
}
). On décrit les mnémoniques les plus utilisés, permettant de compiler un cœur
de langage impératif:

\begin{itemize}

\item \texttt{mov src, dst} copie le contenu de \texttt{src} dans \texttt{dst}.

\item \texttt{add src, dst} calcule la somme des contenus de \texttt{src} et
  \texttt{dst} et place ce résultat dans \texttt{dst}.

\item \texttt{push src} place \texttt{src} sur la pile, c'est-à-dire que cette
  instruction enlève au pointeur de pile \esp la taille de \texttt{src}, puis
  place \texttt{src} à l'adresse mémoire de la nouvelle valeur \esp.

\item \texttt{pop src} réalise l'opération inverse : elle charge le contenu de
  la mémoire à l'adresse \esp dans \texttt{src} puis incrémente \esp de la
  taille correspondante.

\item \texttt{jmp addr} saute à l'adresse \texttt{addr} : c'est l'équivalent de
  \texttt{mov addr, \%eip}.

\item \texttt{call addr} sert aux appels de fonction : cela revient à
  \texttt{push \%eip} puis \texttt{jmp addr}.

\item \texttt{ret} sert à revenir d'une fonction : c'est l'équivalent de
  \texttt{pop \%eip}.

\end{itemize}

Certaines de ces instructions font référence à la pile par le biais du registre
\esp. Cette zone mémoire n'est pas gérée de manière particulière. Elle permet de
gérer la pile des appels de fonction en cours grâce à la manière dont
\texttt{jmp} et \texttt{ret} fonctionnent. Elle sert aussi à stocker les
variables locales des fonctions.

À l'aide de ces quelques instructions on peut implanter des algorithmes
impératifs. Mais pour faire quelque chose de visible, comme afficher à l'écran
ou envoyer un paquet sur le réseau, cela ne suffit pas : il faut parler au reste
du matériel.

Pour ceci, il y a deux techniques principales. D'une part, certains
périphériques sont dits \emph{memory-mapped} : ils sont associés à un espace
mémoire particulier, qui ne permet pas de stocker des informations mais de lire
ou d'écrire des données dans le périphérique. Par exemple, écrire à l'adresse
$0xB8000$ permet d'écrire des caractères à l'écran. L'autre système principal
est l'utilisation des ports d'entrée/sortie. Cela correspond à des instructions
spéciales \texttt{in \%ax, port} et \texttt{out port, \%ax} où \texttt{port}
est un numéro qui correspond à un périphérique particulier. Par exemple, en
écrivant dans le port $0x60$ on peut contrôler l'état des indicateurs lumineux
du clavier PS/2.

\section{Tâches et niveaux de privilèges}
\label{sec:taches}

\subsection*{Alternance des tâches}

Sans mécanisme particulier, le processeur exécuterait uniquement une suite
d'instruction à la fois. Pour lui permettre d'exécuter plusieurs tâches, un
système de partage du temps existe.

À des intervalles de temps réguliers, le système est programmé pour recevoir une
interruption. C'est une condition exceptionnelle (au même titre qu'une division
par zéro) qui fait sauter automatiquement le processeur dans une routine de
traitement d'interruption. À cet endroit le code peut sauvegarder les registres
et restaurer un autre ensemble de registres, ce qui permet d'exécuter plusieurs
tâches de manière entrelacée. Si l'alternance est assez rapide, cela peut donner
l'illusion que les programmes s'exécutent en parallèle. Comme l'interruption
peut survenir à tout moment, on parle de multitâche préemptif.

En plus de cet ordonnancement de processus, l'architecture Intel permet
d'affecter des niveaux de privilège à ces tâches, en restreignant le type
d'instructions exécutables, ou en donnant un accès limité à la mémoire aux
tâches de niveaux moins élevés.

Il y a 4 niveaux de privilèges (nommés aussi \emph{rings}): le \ring{0} est le
plus privilégié, le \ring{3} le moins privilégié. Dans l'exemple précédent, on
pourrait isoler l'ordonnanceur de processus en le faisant s'exécuter en \ring{0}
alors que les autres tâches seraient en \ring{3}.

\subsection*{Mémoire virtuelle}

À partir du moment où plusieurs processus s'exécutent de manière concurrente, un
problème d'isolation se pose: si un processus peut lire dans la mémoire d'un
autre, des informations peuvent fuiter; et s'il peut y écrire, il peut en
détourner l'exécution.

Le mécanisme de mémoire virtuelle permet de donner à deux tâches une vue
différente de la mémoire: c'est-à-dire que vue de tâches différentes, une
adresse contiendra une valeur différente (figure~\ref{fig:memoire-virtuelle}).

\begin{figure}[h] % fig:memoire-virtuelle {{{
\centering
\begin{tikzpicture}
  [memlayout/.style={draw
                    ,shape=rectangle
                    ,minimum height=5cm
                    ,minimum width=1cm
                    }
  ,node distance=2.5cm
  ]

  \node[memlayout]            (A) {};
  \node[memlayout,right of=A] (B) {};
  \node[memlayout,right of=B] (C) {};

  \def\putlabel#1#2{
    \path (#1.north) -- ++(0,3mm) node {#2};
  }

  \putlabel{A}{Processus 1}
  \putlabel{B}{Mémoire}
  \putlabel{C}{Processus 2}

  \def\mmap#1#2#3#4#5{
    \begin{scope}
      [mzone/.style={draw
                    ,shape=rectangle
                    ,fill=black!20
                    ,minimum height=#5 mm
                    ,minimum width=1cm
                    }
      ,corr/.style={dashed}
      ]
      \node[mzone] (L) at ($ (#1.north) - (0, #2) $) {};
      \node[mzone] (R) at ($ (#3.north) - (0, #4) $) {};

      \draw[corr] (L.north east) -- (R.north west);
      \draw[corr] (L.south east) -- (R.south west);
    \end{scope}
  }

  \mmap{A}{1.5}{B}{2.5}{4}
  \mmap{A}{4.5}{B}{4}{3}
  \mmap{B}{1}{C}{1}{2}
  \mmap{B}{3.2}{C}{2.7}{4}
\end{tikzpicture}
\caption{Mécanisme de mémoire virtuelle.}
\label{fig:memoire-virtuelle}
\end{figure} % }}}

\label{page:mem-virt}
Ce mécanisme est contrôlé par valeur du registre \crtrois: les 10 premiers bits
d'une adresse virtuelle sont un index dans le répertoire de pages qui commence à
l'adresse contenue dans \crtrois. À cet index, se trouve l'adresse d'une table
de pages. Les 10 bits suivants de l'adresse sont un index dans cette page,
donnant l'adresse d'une page de 4 kio (figure~\ref{fig:pagetables}).
Plus de détails sur l'utilisation de ce mécanisme seront donnés dans la
section~\ref{sec:linux-sys}.

\begin{figure}[h]
\begin{tikzpicture}
[ bit/.style={ draw
             , shape=rectangle
             , minimum width=4mm
             , minimum height=4mm
             , node distance=4mm
             }
, pstuff/.style={ draw
                , shape=rectangle
                , node distance=1cm
                , text width=2cm
                , text centered
                }
, >=latex
]

\node [bit] (b31) {};

% generate this ?
\node [bit, right of=b31] (b30) {}; \node [bit, right of=b30] (b29) {}; \node [bit, right of=b29] (b28) {};
\node [bit, right of=b28] (b27) {}; \node [bit, right of=b27] (b26) {}; \node [bit, right of=b26] (b25) {};
\node [bit, right of=b25] (b24) {}; \node [bit, right of=b24] (b23) {}; \node [bit, right of=b23] (b22) {};
\node [bit, right of=b22] (b21) {}; \node [bit, right of=b21] (b20) {}; \node [bit, right of=b20] (b19) {};
\node [bit, right of=b19] (b18) {}; \node [bit, right of=b18] (b17) {}; \node [bit, right of=b17] (b16) {};
\node [bit, right of=b16] (b15) {}; \node [bit, right of=b15] (b14) {}; \node [bit, right of=b14] (b13) {};
\node [bit, right of=b13] (b12) {}; \node [bit, right of=b12] (b11) {}; \node [bit, right of=b11] (b10) {};
\node [bit, right of=b10] (b9) {}; \node [bit, right of=b9] (b8) {}; \node [bit, right of=b8] (b7) {};
\node [bit, right of=b7] (b6) {}; \node [bit, right of=b6] (b5) {}; \node [bit, right of=b5] (b4) {};
\node [bit, right of=b4] (b3) {}; \node [bit, right of=b3] (b2) {}; \node [bit, right of=b2] (b1) {};
\node [bit, right of=b1] (b0) {};

\foreach \b in {0,11,12,21,22,31}
    \node [above of=b\b, node distance=5mm] {\small \b};

\newcommand{\mbrace}[3]{
    \draw [bigbrace]                      ($ (b#1.south east) + (-0.1,-0.3) $)
    -- node [midway, yshift=-2mm] (#3) {} ($ (b#2.south west) +  (0.1,-0.3) $);
} % mbrace

\mbrace{0}{11}{bpp};
\mbrace{12}{21}{bpt};
\mbrace{22}{31}{bpd};

\node[pstuff, below of=bpd] (pd) {Répertoire de pages};
\node[pstuff, below of=bpt] (pt) {Table de pages};
\node[pstuff, below of=bpp] (pp) {Page de 4 Kio};

\node[left of=pd, node distance=2cm] (cr) {\crtrois};

\draw [->] (bpp) -- (pp);
\draw [->] (bpt) -- (pt);
\draw [->] (bpd) -- (pd);

\draw [->] (cr) -- (pd);
\draw [->] (pd) -- (pt);
\draw [->] (pt) -- (pp);
\draw [->] (pp) -- ++ (3,0);

\end{tikzpicture}

\caption{Implantation de la mémoire virtuelle}
\label{fig:pagetables}
\end{figure}

\section{Appels système}

Avec une telle isolation, tout le code qui est exécuté en \ring{3} a une
expressivité limitée. Il ne peut pas contenir d'instructions privilégiées comme
\asminstr{in} ou \asminstr{out}, ni faire référence à des périphériques mappés
en mémoire. C'est en effet au noyau d'accéder au matériel, et pas au code
utilisateur.

Il est donc nécessaire d'appeller une routine du noyau depuis le code
utilisateur. C'est le but des appels système. Cela consiste à coupler une
fonction du \ring{3} à une fonction du \ring{0}: en appellant la fonction non
privilégiée, le flot d'exécution se retrouve dans le noyau avec les bons
privilèges.

Bien sûr, il n'est pas possible de faire directement un \asminstr{call} puisque
cela consisterait à faire un saut vers une zone plus privilégiée. Il y a
plusieurs manières d'implanter ce mécanisme. Nous décrivons ici la technique
historique à l'aide d'interruptions.

Le processeur peut répondre à des interruptions, qui sont des évenements
extérieurs. Cela permet d'écrire du code asynchrone. Par exemple, une fois qu'un
long transfert mémoire est terminé, une interruption est reçue. D'autres
interruptions dites logicielles peuvent arriver lorsqu'une erreur se produit.
Par exemple, diviser par zéro provoque l'interruption 0, et tenter d'exécuter
une instruction privilégiée provoque l'interruption 14. On peut aussi provoquer
une interruption par une instruction \asminstr{int} dédiée.

Une table globale définit, pour chaque numéro d'interruption : quelle est la
routine à appeller pour la traiter, avec quel niveau de privilège, ainsi que le
niveau de privilège requis pour pouvoir déclencher celle-ci avec l'instruction
\asminstr{int}.

Il est donc possible de créer une interruption purement logicielle (on utilise
en général le numéro 128 soit \hexa{80}), déclenchable en \ring{3} et traitée en
\ring{0}. Les registres sont préservés, donc on peut les utiliser pour passer un
numéro d'appel système (par exemple 3 pour \texttt{read()} et 5 pour
\texttt{open()}) et leurs arguments.

\section{Le \emph{Confused Deputy Problem}}
\label{sec:secu-syscalls}

On a vu que les appels systèmes permettent aux programmes utilisateur d'accéder
au services du noyau. Ils forment donc une interface particulièrement sensible aux
problèmes de sécurité.

Comme pour toutes les interfaces, on peut être plus ou moins fin. D'un
côté, une interface pas assez fine serait trop restrictive et ne permettrait pas
d'implanter tout type de logiciel. De l'autre, une interface trop laxiste
(«écrire dans tel registre matériel») empêche toute isolation. Il faut
donc trouver la bonne granularité.

Nous allons présenter ici une difficulté liée à la manipulation de mémoire au
sein de certains types d'appels système.

Il y a deux grands types d'appels systèmes: d'une part, ceux qui renvoient un
simple nombre, comme \texttt{getpid} qui renvoie le numéro du processus
appelant.

\insertcode{appel-syscall-getpid.c}

Ici, pas de difficulté particulière: la communication entre le \ring{0} et le
\ring{3} est faite uniquement à travers les registres, comme décrit dans la
section~\ref{sec:impl-syscall}.

Mais la plupart des appels systèmes communiquent de l'information de manière
indirecte, à travers un pointeur. L'appellant alloue une zone mémoire dans son
espace d'adressage et passe un pointeur à l'appel système. Ce mécanisme est
utilisé par exemple par la fonction \texttt{gettimeofday}
(figure~\ref{fig:appel-gettimeofday}).

\begin{figure}[h]
\insertcode{appel-syscall-gettimeofday.c}
\caption{Appel de gettimeofday}
\label{fig:appel-gettimeofday}
\end{figure}

Considérons une implantation naïve de cet appel système qui écrirait
directement à l'adresse pointée. Si le pointeur fourni est dans l'espace
d'adressage du processus, on est dans le cas d'utilisation normal et l'écriture
est donc possible.

Si l'utilisateur passe un pointeur dont la valeur correspond à la mémoire
réservée au noyau, que se passe-t-il? Comme le déréférencement est fait dans le
code du noyau, il est également fait en \ring{0}, et va pouvoir être réalisé
sans erreur: l'écriture se fait et potentiellement une structure importante du
noyau est écrasée.

Un utilisateur malveillant peut donc utiliser cet appel système pour écrire à
n'importe quelle adresse dans l'espace d'adressage du noyau. Ce problème vient
du fait que l'appel système utilise les privilèges du noyau au lieu de celui qui
contrôle la valeur des paramètres sensibles. Cela s'appelle le \emph{Confused
Deputy Problem}\cite{hardy88confused}.

La bonne solution est de tester dynamiquement la valeur du pointeur: s'il pointe
en espace noyau, il faut indiquer une erreur plutôt que d'écrire. Sinon, il peut
toujours y avoir une erreur, mais au moins le noyau est protégé.

\begin{figure}
\insertcode{implem-gettime.c}
\caption{Implantation de l'appel système \texttt{gettimeofday}}
\label{fig:implem-gettime}
\end{figure}

Dans le noyau, un ensemble de fonctions permet d'effectuer des copies sûres. La
fonction \texttt{access\_ok} réalise le test décrit précédemment. Les fonctions
\texttt{copy\_from\_user} et \nolinkurl{copy_to_user} réalisent une copie de la
mémoire après avoir fait ce test. Ainsi, l'implantation correcte de l'appel
système \texttt{gettimeofday} fait appel à celle-ci
(figure~\ref{fig:implem-gettime}).

Pour préserver la sécurité du noyau, il est donc nécessaire de vérifier la
valeur de tous les pointeurs dont la valeur est contrôlée par l'utilisateur.
Cette conclusion est assez contraignante, puisqu'il existe de nombreux endroits
dans le noyau où des données proviennent de l'utilisateur. Il est donc
raisonnable de vouloir vérifier automatiquement et statiquement l'absence de
tels défauts.

%\clearpage

%\section{Architecture et assembleur Intel}

%L'implantation d'un système d'exploitation est très proche du matériel sur
%lequel il s'exécute. Pour étudier une implantation en particulier, voyons ce que
%permet le matériel lui-même.

%Dans cette section nous décrivons le fonctionnement d'un processeur utilisant
%une architecture Intel 32 bits. Les exemples de code seront écrits en syntaxe
%AT\&T, celle que comprend l'assembleur GNU.\@

%La référence pour la description de l'assembleur Intel est la documentation du
%constructeur~\cite{intelsys}; une bonne explication de l'agencement dans la
%pile peut aussi être trouvée dans~\cite{SmashingTheStack}.

%Pour faire des calculs, le processeur est composé de registres, qui sont des
%petites zones de mémoire interne, et peut accéder à la mémoire principale.

%La mémoire principale contient divers types des données:

%\begin{itemize}
%\item le code des programmes à exécuter
%\item les données à disposition des programmes
%\item la pile d'appels
%\end{itemize}

%La pile d'appels est une zone de mémoire qui est notamment utilisée pour tenir
%une trace des calculs en cours. Par exemple, c'est ici que seront stockées les
%données propres à chaque fonction appelée: paramètres, adresse de retour et
%variables locales. La pile est manipulée par un pointeur de pile (\emph{stack
%pointer}), qui est l'adresse du «haut de la pile». On peut la manipuler en
%empilant des données (les placer au niveau du pointeur de pile et déplacer celui
%si) ou dépilant des données (déplacer le pointeur de pile dans l'autre sens et
%retourner la valeur présente à cet endroit).

%L'état du processeur est défini par la valeur de ses registres, qui sont des
%petites zones de mémoire interne (quelques bits chacun). Par exemple, la valeur
%du pointeur de pile est stockée dans \esp{}. Le registre \ebp{}, couplé à \esp{}
%sert à adresser les variables locales et paramètres d'une fonction, comme ce
%sera expliqué dans la section~\ref{sec:convappel}.

%L'adresse de l'instruction courante est accessible dans le registre \eip.

%En plus de ces registres spéciaux, le processeur possède de nombreux registres
%génériques, qui peuvent être utilisés pour réaliser des calculs intermédiaires.
%Ils sont nommés \eax{}, \ebx{}, \ecx{}, \edx{}, \esi{} et \edi{}. Ils peuvent
%être utilisés pour n'importe quel type d'opération, mais certains sont
%spécialisés: par exemple il est plus efficace d'utiliser \eax{} en accumulateur,
%ou \ecx{} en compteur.

%Les calculs sont décrits sous forme d'une suite d'instructions. Chaque
%instruction est composée d'un mnémonique et d'une liste d'opérandes. Les
%mnémoniques (\asminstr{mov}, \asminstr{call}, \asminstr{sub}, etc) définissent
%un type d'opération à appliquer sur les opérandes. L'instruction peut aussi être
%précédée d'une étiquette, qui correspondra à son adresse.

%\begin{center}
  %\shorthandoff{!}
  %\begin{tikzpicture}
    %[bbr/.style={bigbrace, decoration={mirror, amplitude=2mm}}
    %,asmbox/.style={minimum height=1cm}
    %,dway/.style={dashed}
    %]

    %\node[asmbox]              (lbl) {\asminstr{lbl:}};
    %\node[asmbox,right of=lbl] (mov) {\asminstr{mov}};
    %\node[asmbox,right of=mov] (op1) {\asminstr{\$4,}};
    %\node[asmbox,right of=op1] (op2) {\asminstr{\%ebx}};

    %\draw[bbr] (lbl.south west) -- (lbl.south east);
    %\draw[bbr] (mov.south west) -- (mov.south east);
    %\draw[bbr] (op1.south west) -- (op2.south east);

    %\coordinate (ops) at ($ (op1.south west)!.5!(op2.south east) $);

    %\node[below left of=lbl, xshift=-1cm, yshift=-5mm]   (tlbl) {Étiquette};
    %\node[below of=tlbl, node distance=7mm] (tmov) {Mnémonique};
    %\node[below of=tmov, node distance=7mm] (tops) {Opérandes};

    %\draw[dway] (tlbl) -| ($ (lbl.south) + (0, -0.3) $);
    %\draw[dway] (tmov) -| ($ (mov.south) + (0, -0.3) $);
    %\draw[dway] (tops) -| ($ (ops.south) + (0, -0.3) $);

  %\end{tikzpicture}
  %\shorthandon{!}
%\end{center}

%Ces opérandes peuvent être de plusieurs types:

%\begin{itemize}
%\item
%\item
  %une opérande mémoire, c'est-à-dire le contenu de la mémoire à une
  %adresse effective. Cette adresse effective peut être exprimée de
  %plusieurs manières:
  %\begin{itemize}
  %\item
    %directement: \texttt{addr}
  %\item
    %indirectement: \texttt{(\%ecx)}. L'adresse effective est le contenu
    %du registre.
  %\item
    %«base + déplacement»: \texttt{4(\%ecx)}. L'adresse effective est
    %le contenu du registre plus le déplacement (4 ici).
  %\end{itemize}
%\end{itemize}

%En pratique il y a des modes d'adressage plus complexes, et toutes les
%combinaisons ne sont pas possibles, mais ceux-ci suffiront à décrire les
%exemples suivants:

%\section{Fonctions et conventions d'appel}
%\label{sec:convappel}

%\begin{figure} % {{{ fig:stackframe
%\begin{tikzpicture}
[ stack/.style={draw,shape=rectangle,minimum height=1cm,minimum width=5cm}
, bigbrace/.style={decorate,decoration={brace,amplitude=3mm}}
, >=latex
]

  \path (0,-8.5)   node [stack, minimum height=2cm] (gloc) {Locales de g}
  -- ++(0,1.5) node [stack] (gebp) {Cadre précédent}
  -- ++(0,1)   node [stack] (gret) {Retour de g}
  -- ++(0,1.5) node [stack, minimum height=2cm] (gpar) {Arguments de g}
  -- ++(0,1.5) node [stack] (floc) {Locales de f}
  -- ++(0,1)   node [stack] (febp) {Cadre précédent}
  -- ++(0,1)   node [stack] (fret) {Retour de f}
  -- ++(0,1)   node [stack] (fpar) {Arguments de f};

  \draw [->] (gebp) -- ++(4, 0) |- (febp.355);
  \draw [->] (febp.5) -- ++(1.5, 0) -- ++(0,4);

  \draw [dashed] (fpar.north west) -- ++ (0,2);
  \draw [dashed] (fpar.north east) -- ++ (0,2);
  \draw [dashed] (gloc.south east) -- ++ (0,-2);
  \draw [dashed] (gloc.south west) -- ++ (0,-2);

  \node at (5, 1) (ox) {};

  \node [text width=3cm] at (-0.8, 2)     {Adresses hautes (sens de \asminstr{pop})};
  \node [text width=3cm] at (-0.8, -10.5) {Adresses basses (sens de \asminstr{push})};

  \node [text width=3cm] at (-6, -3) (asmtext) {\tt
    call f\\
    ... \\
    f:  ... \\
    pushl \$2 \\
    pushl \$4 \\
    call g \\
    addl \$8, \%esp \\
    leave \\
    ret \\
    ... \\
    g: push \%ebp \\
    mov \%esp, \%ebp \\
    sub \$8, \%esp \\
    ... \\
    leave \\
    ret
    };

  \draw [->] (gret) -- ++ (-3, 0) |- ($ (asmtext.north west) + (2.5, -3) $);
  \draw [->] (fret) -- ++ (-3, 0) |- ($ (asmtext.north west) + (2.5, -0.5) $);

  \draw [<-] (gebp.west) -- ++ (-0.5,0) node [xshift=-4mm] {\ebp};
  \draw [<-] ($ (gloc.west) + (0,-0.5) $) -- ++ (-0.5,0) node [xshift=-4mm] {\esp};

  \draw [<-] ($ (asmtext.north west) + (2.5, -6.7) $) -- ++ (0.5,0) node [xshift=3mm] {\eip};

  \draw [bigbrace]
        ($ (gpar.north east) + (3, -0.1) $)
     -- node [auto,midway,xshift=3mm] {Cadre de g}
        ($ (gloc.south east) + (3, 0) $);

  \draw [bigbrace]
        ($ (fpar.north east) + (3, 0) $)
     -- node [auto,midway,xshift=3mm] {Cadre de f}
        ($ (floc.south east) + (3, 0.1) $);

\end{tikzpicture}

%\caption[Cadres de pile]{ Cadres de pile. }
%\label{fig:stackframe}
%\end{figure} % }}}

%Dans le langage d'assemblage, il n'y a pas de notion de fonction; mais
%\asminstr{call} et \asminstr{ret} permettent de sauvegarder et de restaurer une
%adresse de retour, ce qui permet de faire un saut et revenir à l'adresse
%initiale. Ce système permet déjà de créer des procédures, c'est-à-dire des
%fonctions sans arguments ni valeur de retour.

%Pour gérer ceux-ci, il faut que les deux morceaux (appelant et appelé) se
%mettent d'accord sur une convention d'appel commune. La convention utilisée sous
%GNU/Linux est appelée \emph{cdecl} et possède les caractéristiques suivantes:

%\begin{itemize}
%\item la valeur de retour d'une fonction est stockée dans \eax{}
%\item \eax{}, \ecx{} et \edx{} peuvent être écrasés sans avoir à les sauvegarder
%\item les arguments sont placés sur la pile (et enlevés) par l'appelant. Les
  %paramètres sont empilés de droite à gauche.
%\end{itemize}

%Pour accéder à ses paramètres, une fonction peut donc utiliser un adressage
%relatif à \esp. Cela peut fonctionner, mais cela complique les choses si elle
%contient aussi des variables locales. En effet, les variables locales sont
%placées sur la pile, au dessus des (c'est-à-dire, empilées après) paramètres,
%augmentant le décalage.

%Pour simplifier, la pile est organisée en cadres logiques: chaque cadre
%correspond à un niveau dans la pile d'appels de fonctions. Si $f$ appelle $g$,
%qui appelle $h$, il y aura dans l'ordre sur la pile le cadre de $f$, celui de
%$g$ puis celui de $h$.

%Ces cadres sont chainés à l'aide du registre \ebp{}: à tout moment, \ebp{}
%contient l'adresse du cadre de l'appelant.

%Prenons exemple sur la figure~\ref{fig:stackframe}: pour appeler
%\verb!g(4,2)!, \texttt{f} empile les arguments de droite à gauche.
%L'instruction \asminstr{call g} empile l'adresse de l'instruction suivante sur
%la pile puis saute au début de \texttt{g}.

%Au début de la fonction, les trois instructions permettent de sauvegarder
%l'ancienne valeur de \ebp{}, faire pointer \ebp{} à une endroit fixe dans le
%cadre de pile, puis allouer 8 octets de mémoire pour les variables locales.

%Dans le corps de la fonction \texttt{g}, on peut donc se référer aux variables
%locales par \texttt{-4(\%ebp)}, \texttt{-8(\%ebp)}, etc, et aux arguments par
  %\texttt{8(\%ebp)}, \texttt{12(\%ebp)}, etc.

%À la fin de la fonction, l'instruction \asminstr{leave} est équivalente à
%\texttt{mov \%ebp, \%esp} puis \texttt{pop \%ebp} et permet de défaire le cadre
%de pile, laissant l'adresse de retour en haut de pile. Le \asminstr{ret} final
%la dépile et y saute.


% vim: spelllang=fr
