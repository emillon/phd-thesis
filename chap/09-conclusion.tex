\epigraph{
Many C programmers
believe that "strong typing" just means pounding extra hard on the keyboard.
}{Peter van den Linden}

\section{Limitations}

Un seul type d'entier est supporté.

Notre système ne supporte pas certains cas de transtypage (\emph{casts}).


\section{Travaux futurs}

\subsection{Transtypage}

Les casts entre types entiers ne posent pas de problèmes. Ceux entre types
pointeurs ne posent pas de problème non plus puisqu'au moment de la traduction
en \langname on efface les étiquettes de types sur les pointeurs.

Mais les casts entre entiers et pointeurs posent problème. Si on autorise les
casts, la seule manière sensible de les typer est la suivante.

\begin{mathpar}
\irule{PtrInt-Bad}
  { Γ ⊢ e : t~q~*}
  { Γ ⊢ (\tInt)~e : \tInt }

\irule{IntPtr-Bad}
  { Γ ⊢ e : \tInt }
  { Γ ⊢ (\textsc{Ptr})~e : t~q~*}
\end{mathpar}

Cela pose problème car il est alors possible de créer une fonction qui convertit
n'importe quel type pointeur en n'importe quel autre type pointeur :

\[
  ⊢ \eFun{p}{ \iReturn{(\textsc{Ptr})~(\tInt)~p} }
  : (t_a~q_a~*) → t_b~q_b~*
\]

% TODO et à l'exec?
% TODO on peut faire un dynamic check ou un type UInt ou un qualif sur tous les
% types

% TODO le modèle mémoire de Newspeak est bcp plus bas niveau que celui de
% Safespeak, ex types de deref, tailles dans Lv etc

L'opération problématique est en fait l'opérateur $(\tInt)$ de conversion d'un
pointeur vers un entier, car l'entier résultant ne pose pas de problème de
sûreté ni de sécurité. Pour lui donner une sémantique, on peut supposer que
l'environnement d'exécution est paramétré par une fonction de placement en
mémoire $\addrInt : Φ → \tInt$ qui à un chemin associe un entier.
On étend alors la sémantique par la règle suivante.

\[
  \irule{CastInt}
    { }
    { \mms{(\tInt)~\widehat{\&}φ}{\addrInt(φ)} }
\]

% TODO nommer les ensembles
% TODO mettre la syntaxe en annexe

Quant au transtypage dans le sens inverse, il est plus compliqué à traiter.
À l'exécution il est en effet nécessaire d'avoir à disposition une fonction
$\interpPtr{t~q~*} : \tInt → t~q~*$ qui permette d'interpréter un entier comme
pointeur qualifié $q$ vers un objet de type $t$ de manière qu'on ajoute la règle
suivante.

\[
  \irule{CastPtr}
    { }
    { \mms{(\textsc{Ptr})~\widehat{n}}{\interpPtr{t~q~*}(\widehat{n})} }
\]

% TODO les widehat...

L'exécution de l'analogue de $\addrInt$ demanderait de créer un pointeur
depuis un entier.

% TODO il n'y a pas de t q en mémoire, doncil faut juste un interp : Int -> Φ
% TODO problème car ça peut correspondre à plusieurs pointeurs
% TODO problème de layout mémoire : ça marche qd tout est boxé mais pas si c'est
% unboxé

La solution la plus sûre est donc d'interdire totalement ces conversions.

\subsection{Analyse du noyau Linux}

Ici nous avons présenté l'analyse expérimentale d'une fonction problématique
d'une interface programmation particulière. Mais le principe de l'analyse est
applicable à toutes les fonctions de traitement d'\texttt{ioctl}s des pilotes
KMS, et même à toutes les fonctions faisant partie des différentes interfaces
recevant un pointeur de l'espace utilisateur.

\subsection{Autres types abstraits}

Notre approche est basée sur le fait de rendre abstrait un type de C et
d'y interdire certaines opérations : ici, on marque un certain type de pointeur
comme ``utilisateur'' et on interdit l'opérateur \texttt{*} dessus.

Le langage C n'ayant pas de types abstraits
\footnote{
  La seule abstraction possible est lorsqu'on manipule une structure par
  pointeur. Il n'est alors pas nécessaire de connaître sa définition totale.
  L'idiome est alors de placer uniquement une déclaration en avance
  (\texttt{struct s;}) dans l'en-tête (\texttt{.h}) et de renseigner
  la définition complète dans l'implantation (\texttt{.c}).
}
, on ne peut pas séparer la réprésentation d'un type (par exemple : entier signé
de 32 bits) des opérations qui y sont attachées.

Dans de nombreuses interface, on emploie des types entiers qui servent
d'étiquette. Par exemple, les descripteurs de fichiers renvoyés par la fonction
\texttt{open()} et passés au fonctions \texttt{read()} et \texttt{write()} ont
le type \texttt{int}. Le langage autorise donc par exemple de multiplier entre
eux deux descripteurs de fichiers, ce qui ne correspond pas à une opération
concrète.

On peut aussi distinguer plusieurs types abstraits entre eux. Par exemple, si un
encodeur vidéo manipule des numéros de \emph{frame} et des identifiants de codec
tous les deux entiers, on peut interdire d'utiliser avec un identifiant de codec
une fonction prenant en paramètre un numéro de \emph{frame}.

% TODO check : les footnotes commençent par une majuscule
