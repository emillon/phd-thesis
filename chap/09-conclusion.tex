On présente ici un résumé des travaux présentés, en commençant par un bilan des
contributions réalisées. On réalise ensuite un tour des aspects posant problème,
ou traités de manière incomplète, en évoquant les travaux possibles pour
enrichir l'expressivité de ce système.

\section{Contributions}

Cette thèse comporte 4 contributions principales.

\paragraph{Un langage impératif bien typé}

Le système de types de C est trop rudimentaire pour permettre d'obtenir des
garanties sur l'exécution des programmes bien typés. En interdisant certaines
constructions dangereuses et en annotant certaines autres, nous avons isolé un
langage impératif bien typable, \langname, pour lequel on peut définir un
système de types sûr.

\paragraph{Une sémantique basée sur les lentilles}

Une des particularités de \langname est qu'il utilise un état mémoire structuré,
modélisant les cadres de piles présents dans le langage. Pour décrire la
sémantique des accès mémoire, nous utilisons le concept de lentilles issues de
la programmation fonctionnelle et des systèmes de bases de données. Cela permet
de définir de manière déclarative la modification en profondeur de valeurs dans
la mémoire, sans avoir à distinguer le cas de la lecture et celui de l'écriture.

\paragraph{Un système de types abstraits}

En partant de ce système de types, on a décrit une extension permettant de créer
des pointeurs pour lesquels l'opération de déréférencement est restreinte à
certaines fonctions. Dans le contexte d'un noyau de système d'exploitation,
cette restriction permet de vérifier statiquement qu'à aucun moment le noyau ne
déréférence un pointeur dont la valeur est contrôlée par l'espace utilisateur,
évitant ainsi un problème de sécurité. Cette approche peut s'étendre à d'autres
classes de problèmes comme par exemple éviter l'utilisation de certaines
opérations sur les types entiers lorsqu'ils sont utilisés comme identificateurs
ou masque de bits.

\paragraph{Un prototype d'analyseur statique}

Les analyses de typage ici décrites ont été implantées sous forme d'un prototype
d'analyseur statique distribué avec le langage \newspeak, développé par EADS. Le
choix de \newspeak pour l'implantation demande d'adapter les règles de typage,
mais il permet de réutiliser un traducteur existant et à l'entreprise de
profiter des résultats. Ce prototype permet d'une part de vérifier la propriété
d'isolation des appels système sur du code C existant, et d'autre part fournit
une base saine pour implanter d'autres analyses de typage sur le langage
\newspeak. Ce prototype a été utilisé pour confirmer l'existence de deux bugs
dans le noyau Linux, ce qui permet de valider l'approche: il est possible de
vérifier du code de production à l'aide de techniques de typage. Des travaux
d'expérimentation sont en cours afin d'analyser de plus grandes parties du
noyau.

\section{Différences avec C}

\langname a été construit pour pouvoir ajouter un système de types à un langage
proche de C. Ces deux langages diffèrent donc sur certains points. On détaille
ici ces différences et, selon les cas, comment les combler ou pourquoi cela est
impossible de manière inhérente.

\paragraph{Types numériques}

En C, on dispose de plusieurs types entiers, pouvant avoir plusieurs tailles et
être signés ou non signés, ainsi que des types flottants qui diffèrent par leur
taille. Au contraire, en \langname{} on ne conserve qu'un seul type d'entier et
un seul type de flottant. La raison pour cela est que nous ne nous intéressons
pas du tout aux problématiques de sémantique arithmétique: les débordements,
dénormalisations, etc, sont supposés ne pas arriver.

Il est possible d'étendre le système de types de \langname{} pour ajouter tous
ces nouveaux types. La traduction depuis \newspeak insère déjà des opérateurs de
transtypage pour lesquels il est facile de donner une sémantique (pouvant lever
une erreur en cas de débordement, comme en Ada) et un typage. Les littéraux
numériques peuvent poser problème, puisqu'ils deviennent alors polymorphes. Une
solution peut être de leur donner le plus grand type entier et d'insérer un
opérateur de transtypage à chaque littéral. Haskell utilise une solution
similaire: les littéraux entiers ont le type de précision arbitraire
\texttt{Integer} et sont convertis dans le bon type en appelant la fonction
\texttt{fromInteger} du type synthétisé à partir de l'environnement.

\paragraph{Transtypage et unions}

Puisque l'approche retenue est basée sur le typage statique, il est impossible
de capturer de nombreuses constructions qui sont permises, ou même idiomatiques,
en C:\@ les unions, les conversions de types (explicites ou implicites) et le
\emph{type punning} (défini ci-dessous). Les deux premières sont équivalentes.
Bien qu'on puisse remplacer chaque conversion explicite d'un type $t_1$ vers un
type $t_2$ par l'appel à une fonction $\mathrm{cast}_{t_1,t_2}$, on ajoute alors
un \enquote{trou} dans le système de types. Cette fonction devrait en effet être
typée $(t_1) → t_2$, autrement dit le type \enquote{maudit} $α → β$ de
\texttt{Obj.magic} en OCaml ou \texttt{unsafeCoerce} en Haskell.

Le \emph{type punning} consiste à modifier directement la suite de bits de
certaines données pour la manipuler d'une manière efficace. Par exemple, il est
commun de définir un ensemble de macros pour accéder à la mantisse et à
l'exposant de flottants IEEE754. Ceci peut être fait avec des unions ou des
masques de bits.

Dans de tels cas, le typage statique est bien sûr impossible. Pour traiter ces
cas, il faudrait encapsuler la manipulation dans une fonction et y ajouter une
information de type explicite, comme $\texttt{float\_exponent} : (\tFloat) →
\tInt$.

Pour ces conversions de types, on distingue en fait plusieurs cas: les
conversions entre types numériques, entre types pointeurs, ou entre un type
entier et un type pointeur.

\added{débordements}

Le premier ne pose pas de problème: il est toujours possible de donner une
sémantique à une conversion entre deux types numériques, quitte à détecter les
cas où il faut signaler une erreur à l'exécution (comme en cas de débordement).

\added{parenthèse user/kernel}

Le deuxième non plus n'est pas un problème en soi: une conversion entre deux
types pointeurs revient à convertir entre les types pointés (il faut bien
sûr interdire les conversions entre pointeurs noyau et utilisateur).

Le vrai problème provient des conversions entre entiers et pointeurs, qui sont
des données fondamentalement différentes. Le même problème se pose d'ailleurs si
on cherche à convertir une fonction en entier ou en pointeur, même si les
raisons valables pour faire cela sont moins nombreuses.
Si on s'en tient aux conversions entre entiers et pointeurs, une manière naïve
de typer ces opérations est:

\begin{mathpar}
\irule{PtrInt-Bad}
  { Γ ⊢ e : \ptrK{t}}
  { Γ ⊢ (\tInt)~e : \tInt }

\irule{IntPtr-Bad}
  { Γ ⊢ e : \tInt }
  { Γ ⊢ (\textsc{Ptr})~e : \ptrK{t}}
\end{mathpar}

Tout d'abord, cela pose problème car il est alors possible de créer une fonction
pouvant convertir n'importe quel type pointeur en n'importe quel autre type
pointeur:

\[
  ⊢ ~ \eFun{p}{ \iReturn{(\textsc{Ptr})~(\tInt)~p} } : ~ (\ptrK{t_a}) → \ptrK{t_b}
\]

Si on crée une variable du type $t_a$, prend son adresse, la convertit à l'aide
de cette fonction, puis déréférence le résultat, on obtient une valeur du type
$t_b$ (remarquons que ce genre d'opération est tout à fait possible en C).

Outre ce problème de typage, il faudrait pouvoir donner une sémantique à ces
opérations. Convertir un pointeur en entier revient à spécifier l'environnement
d'exécution, c'est-à-dire qu'il faut une fonction de placement en mémoire
beaucoup plus précise que notre modèle mémoire actuel. Celle-ci dépend de
beaucoup de paramètres: dans quel sens croit la pile, quelle est la taille des
types, etc.

La conversion dans le sens inverse, d'entier vers pointeur, est encore plus
complexe. Entre autres, cela suppose qu'on puisse retrouver la taille des
valeurs à partir de leur adresse. Dans de nombreux langages, on résout ce
problème en stockant la taille de chaque valeur avec elle.

Mais cela fait s'éloigner du modèle mémoire de C, où le déréférencement porte
sur une adresse mais également sur une taille (portée implicitement par le type
du pointeur). Le langage \newspeak{} conserve d'ailleurs cette distinction, que
nous avons éliminée dans \langname. Il y a une incompatibilité entre ces deux
approches : dans le cas de C (et de \newspeak), on laisse le programmeur gérer
l'organisation de la mémoire alors qu'avec \langname ces choix sont faits par le
langage. En contrepartie, cela permet d'avoir d'assurer la sûreté du typage.

\paragraph{Environnement d'exécution}

La sémantique opérationnelle utilise un environnement \linebreak d'exécution
pour certains cas. Contrairement à C, les débordements de tampon et les
déréférencements de pointeurs sont vérifiés dynamiquement. Mais ce n'est pas une
caractéristique cruciale de cette approche: en effet, si on suppose que les
programmes que l'on analyse ne comportent pas de telles erreurs de
programmation, on peut désactiver ces vérifications et le reste des propriétés
est toujours valable.

On repose sur l'environnement d'exécution à un endroit plus problématique. À la
sortie de chaque portée (au retour d'une fonction et après la portée d'une
variable locale déclarée), on parcourt la mémoire à la recherche des pointeurs
référençant les variables qui ne sont plus valides. Supprimer ce test rend
l'analyse incorrecte, car il est alors possible de faire référence à une
variable avec un type différent.

Si on peut avoir une garantie statique que les adresses des variables locales ne
seront plus accessibles au retour d'une fonction, alors on peut supprimer cette
étape de nettoyage. Cette garantie peut être obtenue avec une analyse statique
préalable. Par exemple les régions~\cite{jfp92} peuvent être utilisées à cet
effet: en plus de donner un type à chaque expression, on calcule
statiquement la zone mémoire dans laquelle cette valeur sera allouée. Cela
correspond à un ramasse-miette réalisé statiquement.

% attention, on peut finir avec une seule région (pire des cas)

\paragraph{Flot de contrôle}

Dans le langage C, en plus des boucles et de l'alternative, on peut sauter d'une
instruction à l'autre au sein d'une fonction à l'aide de la construction
\texttt{goto}. Pour pouvoir traiter ces cas, il est possible de transformer ces
sauts d'un programme vers des simples boucles. Cette réécriture peut être
coûteuse puisqu'elle peut introduire des variables booléennes et dupliquer du
code. En pratique, c'est d'ailleurs ce qui est fait dans l'implantation puisque
cette transformation est réalisée par \ctonewspeak.

Dans le noyau Linux, il est courant d'utiliser les sauts pour factoriser la
libération de ressources à la fin d'une fonction. Il est d'ailleurs possible
d'utiliser l'outil Coccinelle pour donner cette forme à du code utilisant un
autre style de structures de contrôle~\cite{lctes11}. On peut imaginer qu'il est
possible de l'utiliser pour faire la conversion inverse.

En plus de ces sauts locaux, le langage C contient une manière de sauvegarder un
état d'exécution et d'y sauter, même entre deux fonctions: ce sont
respectivement les constructions \texttt{setjmp} et \texttt{longjmp}. Elles sont
très puissantes puisqu'elles permettent d'exprimer de nouvelles structures de
contrôle. Il s'agit de formes légères de continuations où la pile reste commune.
Cette fonctionnalité peut servir par exemple à implanter des exceptions ou des
coroutines.

Avec l'interprète du chapitre~\ref{cha:lang}, il n'est pas possible de donner
une sémantique à ces constructions. Une des manières de faire est de modifier
les états de l'interprète : au lieu de retenir l'instruction à évaluer avec
$\msi{m}{i}$, on retient la continuation complète : $\msi{m}{k}$.
Pour faciliter ce changement, on peut tout d'abord passer à une sémantique
monadique (ainsi qu'évoqué dans la conclusion de la partie~\ref{part:lang},
page~\pageref{page:ccl-II-monad}) puis ajouter les continuations à la monade
sous-jacente.

En pratique, il est rare de trouver ces constructions plus avancées dans du code
noyau ou embarqué, donc ce manque n'a pas beaucoup d'impact. De plus, cela
permet une présentation plus simple et accessible.

\paragraph{Allocation dynamique}

La plupart des programmes, et le noyau Linux en particulier, utilisent la notion
d'allocation dynamique de mémoire. C'est une manière de créer dynamiquement une
zone de mémoire qui restera accessible après l'exécution de la fonction
courante. Cette mémoire pourra être libérée à l'aide d'une fonction dédiée. Dans
l'espace utilisateur, les programmes peuvent utiliser les fonctions
\verb!malloc()!, \verb!calloc()! et \verb!realloc()! pour allouer des zones de
mémoire et \verb!free()! pour les libérer. Dans le noyau Linux, ces fonctions
existent sous la forme de \verb!kmalloc()!, \verb!kfree()!, etc. Une explication
détaillée de ces mécanismes peut être trouvée dans~\cite{LinuxVMM}.

Ces fonctions manipulent les données en tant que zones mémoires opaques, en
renvoyant un pointeur vers une zone mémoire d'un nombre d'octets donnés. Cela
présuppose un modèle mémoire de plus bas niveau. Pour se rapprocher de la
sémantique de \langname, une manière de faire est de définir un opérateur de
plus haut niveau prenant une expression et retournant l'adresse d'une cellule
mémoire contenant cette valeur (la taille de chaque valeur fait partie de
celle-ci), ou \eNull si l'allocation échoue. Le typage est alors direct (on
suppose que $\textsc{Free}(e)$ est une instruction:

\begin{mathpar}
  \irule{New}
    { Γ ⊢ e : t }
    { Γ ⊢ \textsc{New}(e) : \ptrK{t} }

  \irule{Free}
    { Γ ⊢ e : \ptrK{t} }
    { Γ ⊢ \textsc{Free}(e) }
\end{mathpar}

En ce qui concerne l'exécution, on peut ajouter une troisième composante aux
états mémoire: $m = (s, g, h)$ où $h$ est une liste d'association entre des
identifiants uniques et des valeurs. Chaque allocation dynamique crée une
nouvelle clef entière et met à jour $h$. La libération de mémoire est en
revanche problématique parce qu'il faut faire confiance au programmeur pour ne
pas accéder aux zones mémoires libérées, ni libérer deux fois la même zone
mémoire. \added{double free} Il est aussi possible d'obtenir cette garantie avec
une analyse préalable. Par exemple, il est possible ici encore d'utiliser une
analyse basée sur les régions pour vérifier l'absence de pointeurs
fous~\cite{ifm10}.

%\paragraph{Types structures}

%L'accès à un type structure est délicat car on voudrait à partir d'une
%expression de la forme $x.l$ en déduire une information sur les types de $x$ et
%de $x.l$. La solution que nous avons choisi consiste à s'assurer que chaque $l$
%est associé à un seul type de structure $S$. On suppose qu'une passe de
%prétraitement a annoté le code pour transformer $x.l$ en $x.l_S$ où $S$ est la
%définition complète de la structure. Cela n'est pas satisfaisant pour deux
%raisons: d'une part, les types doivent apparaître dans le code du programme, ce
%qui annule l'avantage de l'inférence de types. D'autre part, cela veut dire
%qu'on ne peut pas exprimer de types structures contenant un pointeur vers une
%valeur du même type (par exemple, pour une liste chaînée). Un moyen de
%contourner cette limitation est d'enrichir le langage des types pour autoriser
%les structures récursives, à l'aide d'un opérateur de point fixe.

%Il existe d'autres solutions à ce problème, sans ajouter l'annotation totale de
%type. On ne connaît alors à l'accès qu'un seul nom de champ. La première est
%d'employer du sous-typage, en déclarant qu'un type structure contenant moins de
%champ est moins général qu'un autre. Au fur et à mesure de l'inférence, on
%obtient alors un type de plus en plus précis. En revanche dans ce cas là on perd
%la décidabilité de l'inférence. Une autre manière de faire est d'employer le
%polymorphisme de rangée: plutôt que de cacher les champs inconnus derrière le
%sous-typage, on en fait une variable $ρ$ qui pourra s'unifier avec d'autres
%champs. Par exemple, partant de $x.l ← 0$ on déduit que $x.l$ a pour type
%$\tInt$ et que $x$ a pour type $\{ l : \tInt ; ρ \} $. En pratique,
%l'implantation décrite dans le chapitre~\ref{cha:implem} utilise une technique
%similaire pour inférer les types des structures à partir du code \newspeak.

%\paragraph{Assembleur}

%Comme la plupart des outils d'analyse de code C, il est impossible de traiter
%l'assembleur en ligne qui peut se trouver entre deux instructions. Dans le cas
%de Linux, le code est fait pour être portable, et les parties dépendantes d'une
%certaine architecture (et donc le code assembleur) sont isolées explicitement.
%On peut alors si nécessaire ajouter une annotation de type sur une fonction dont
%l'implantation est faite en assembleur, mais au sein de cette implantation on ne
%peut bien sûr rien dire. De plus, le corps de cette fonction n'est pas
%couverte par la sûreté du typage: après son exécution on ne peut plus rien
%garantir sur le programme.

%\paragraph{Analyse du noyau Linux}

%Ici nous avons présenté l'analyse expérimentale d'une fonction problématique
%d'une interface programmation particulière. Mais le principe de l'analyse est
%applicable à toutes les fonctions de traitement d'\texttt{ioctl}s des pilotes
%KMS, et même à toutes les fonctions faisant partie des différentes interfaces
%recevant un pointeur de l'espace utilisateur.

%\paragraph{Autres types abstraits}

%Notre approche est basée sur le fait de rendre abstrait un type de C et d'y
%interdire certaines opérations: ici, on marque un certain type de pointeur
%comme \enquote{utilisateur} et on interdit l'opérateur \texttt{*} dessus.

%Le langage C n'ayant pas de types abstraits, on ne peut pas séparer la
%représentation d'un type (par exemple: entier signé de 32 bits) des opérations
%qui y sont attachées. La seule abstraction possible est lorsqu'on manipule une
%structure par pointeur. Il n'est alors pas nécessaire de connaître sa définition
%totale. L'idiome est alors de placer uniquement une déclaration en avance
%(\texttt{struct s;}) dans l'en-tête (\texttt{.h}) et de renseigner la définition
%complète dans l'implantation (\texttt{.c}). Cette technique de \emph{pointeurs
%opaques} n'est pas applicable aux autres types.

%Dans de nombreuses interfaces, on emploie des types entiers qui servent
%d'étiquettes. Par exemple, les descripteurs de fichiers renvoyés par la fonction
%\verb!open()! et passés au fonctions \verb!read()! et \verb!write()! ont
%le type \texttt{int}.

\section{Perspectives}

L'importance des logiciels grandit par deux effets: d'une part, ils sont
présents dans de plus en plus d'appareils et, d'autre part, leur taille est de
plus en plus importante. En une journée, entre les appareils dédiés au calcul, à
la communication, au multimédia et au transport, on est facilement exposé au
fonctionnement de plus d'une dizaine de millions de lignes de code. Il donc
primordial de vérifier que ces logiciels ne peuvent pas être détournés de leur
utilisation prévue. Dans le cas de logiciels avioniques ou militaires, les
conséquences peuvent en effet être catastrophiques. C'est dans ce contexte
industriel que ce travail a été motivé et réalisé.

Au cœur de la plupart de ces systèmes informatiques se trouve un noyau qui
abstrait les détails du matériel pour fournir aux programmes des abstractions
sûres, permettant de protéger les données sensibles contenues dans ce système.
Puisqu'une simple erreur de programmation peut briser cette isolation, on voit
pourquoi la vérification est si importante.

Dans ce but, les systèmes de types sont des outils bien connus de programmeurs.
Même dans les langages peu typés comme C, les compilateurs aident de plus en
plus les programmeurs à trouver des erreurs de programmation. De très nombreuses
analyses peuvent être faites rien qu'en classant les expressions selon le genre
de valeurs qu'elles créent à l'exécution --- c'est la définition que donne
Benjamin C. Pierce d'un système de types~\cite{TAPL}.

L'utilisation d'un système de types comme analyseur statique léger est donc
efficace. Pour des propriétés qui ne dépendent pas de la valeur des expressions,
mais uniquement de leur forme, c'est d'ailleurs la solution à préférer. En
effet, nous avons montré qu'elle est simple à mettre en œuvre et rapide à
exécuter.

On peut se poser la question suivante: pourquoi a-t-on besoin d'une analyse
statique dédiée, plutôt que de passer par le langage C lui-même? Le problème
vient du fait que celui-ci considère que les types définissent une
représentation en mémoire sans guère plus d'information. On peut définir de
nouveaux noms pour un type, mais l'ancien et le nouveau sont alors compatibles.
En un mot il est impossible de distinguer le rôle d'un type (son intention) de
sa représentation (son extension).

Ici, nous avons proposé une solution au problème de pointeurs utilisateur en
introduisant un type ayant la même représentation que les pointeurs classiques,
mais pour lequel l'ensemble des opérations est différent: c'est un type opaque.
Cela suffit déjà à détecter des erreurs de programmation.

Si on ajoutait cette construction au langage, on pourrait définir de nouveaux
types partageant la représentation d'un type C existant, mais qui ne soit pas
compatible avec le type d'origine. Avec cette fonctionnalité dans un langage,
non seulement on peut détecter d'autres classes de problèmes, mais surtout on
laisse le programmeur définir de nouvelles analyses lui-même en modélisant les
problèmes concrets par des types.

% vim: spelllang=fr
