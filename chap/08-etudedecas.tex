Le noyau Linux, abordé dans le chapitre~\ref{cha:os}, est un noyau de système
d'exploitation développé depuis le début des années 90 et \enquote{figure de
proue} du mouvement \emph{open-source}. Au départ écrit par Linus Torvalds sur
son ordinateur personnel, il a été porté au fil des années sur de nombreuses
architectures et s'est enrichi de nombreux pilotes de périphériques. Dans la
version 3.13.1 (2014), son code source comporte 12 millions de lignes de code
(en grande majorité du C) dont 58\% de pilotes.

Même si le noyau est monolithique (la majeure partie des traitements s'effectue
au sein d'un même fichier objet), les sous-systèmes sont indépendants. C'est ce
qui permet d'écrire des pilotes de périphériques et des modules.

Ces pilotes manipulent des données provenant de l'utilisateur, notamment par
pointeur. Comme on l'a vu, cela peut poser des problèmes de sécurité si on
déréférence ces pointeurs sans vérification.

Dans ce chapitre, on met en œuvre sur le noyau Linux le système de types décrit
dans le chapitre~\ref{cha:qualifs}, ou plus précisément l'outil \texttt{ptrtype}
du chapitre~\ref{cha:implem}.

Pour montrer que le système de types capture cette propriété et que
l'implantation est utilisable, on étudie les cas de deux bugs qui ont touché le
noyau Linux. À chaque fois, dans une routine correspondant à un appel système,
un pointeur utilisateur est déréférencé directement, pouvant provoquer une fuite
d'informations confidentielles dans le noyau.

On commence par décrire les difficultés rencontrées pour analyser le code du
noyau Linux. On décrit ensuite l'implantation du mécanisme d'appels système dans
ce noyau, et en quoi cela peut poser des problèmes. On détaille enfin les bugs
étudié, et comment les adapter pour traiter le code en question.

\section{GNU C}
\label{sec:gnuc}

Linux est écrit dans le langage C, mais pas dans la version qui correspond à la
norme. Il utilise le dialecte GNU C qui est celui que supporte GCC. Une première
difficulté pour traiter le code du noyau est donc de le compiler.

Pour traduire ce dialecte, il a été nécessaire d'adapter \ctonewspeak. La
principale particularité est la notation \verb!__attribute__((...))! qui peut
décorer les déclarations de fonctions, de variables ou de types. De nouvelles
fonctionnalités sont aussi présentes.

Par exemple, il est possible de manipuler des étiquettes de première classe: si
\enquote{\texttt{lbl:}} est présent avant une instruction, on peut capturer
l'adresse de celle-ci avec \texttt{void *p = \&\&lbl} et y sauter indirectement
avec \texttt{goto *p}.

Une autre fonctionnalité est le concept d'instruction-expression:
\texttt{(\{bloc\})} est une expression, dont la valeur est celle de la dernière
expression évaluée lors de \texttt{bloc}.

Les attributs, quant à eux, rentrent dans trois catégories:

\begin{itemize}
  \item les annotations de compilation; par exemple, \texttt{used} désactive
  l'avertissement \enquote{cette variable n'est pas utilisée}.

  \item les optimisations; par exemple, les objets marqués \texttt{hot} sont
    groupés de telle manière qu'ils se retrouvent en cache ensemble.

  \item les annotations de bas niveau; par exemple, \verb!aligned(n)!
    spécifie qu'un objet doit être aligné sur au moins \texttt{n} bits.
\end{itemize}

Dans notre cas, toutes ces annotations peuvent être ignorées.

\section{Configuration}

Pour que le code noyau soit compilable, il est nécessaire de définir certaines
macros. En particulier, le système de configuration de Linux utilise des macros
nommées \texttt{CONFIG\_*} pour inclure ou non certaines fonctionnalités. Il a
donc fallu faire un choix; nous avons choisi la configuration par défaut. Pour
analyser des morceaux plus importants du noyau, il faudrait définir un fichier
de configuration plus important.

\section{Appels système sous Linux}
\label{sec:linux-sys}

Dans cette section, nous allons voir comment ces mécanismes sont implantés dans
le noyau Linux. Une description plus détaillée pourra être trouvée
dans~\cite{UnderstandingTheLinuxKernel}, ou, pour le cas de la mémoire
virtuelle, dans~\cite{LinuxVMM}.

Deux rings sont utilisés: en \ring{0}, le code noyau et, en \ring{3}, le code
utilisateur.

Une notion de tâche similaire à celle décrite dans la section~\ref{sec:taches}
existe: elles s'exécutent l'une après l'autre, le changement s'effectuant sur
interruptions.

Pour faire appel aux services du noyau, le code utilisateur doit faire appel à
des appels système, qui sont des fonctions exécutées par le noyau. Chaque tâche
doit donc avoir deux piles: une pile \enquote{utilisateur}, qui sert pour
l'application elle-même, et une pile \enquote{noyau}, qui sert aux appels
système.

\begin{figure} % fig:memmap {{{
\centering
\begin{tikzpicture}
  [scale=0.7
  ,user/.style={fill=black!20}
  ,kernel/.style={fill=black!70}
  ]

  % Memory zone
  %
  % #1 - start
  % #2 - end
  % #3 - color
  \newcommand{\mzone}[3]{
    \path[#3] (#1,0) rectangle (#2,1);
  }

  % Address label
  %
  % #1 - x position
  % #2 - text
  \newcommand{\alabel}[2]{
    \path (#1,1) -- ++(0,0.3) node [pos=1] {\small \tt #2};

  }

  % exec
  \mzone{0.5}{1}{user}

  % lib
  \mzone{2.5}{3.2}{user}

  % stack
  \mzone{3.7}{4}{user}

  % stack
  \mzone{5}{5.5}{user}

  % kernel
  \mzone{6}{8}{kernel}

  % contour
  \draw (0,0) rectangle (8,1);

  \alabel{0}{0}
  \alabel{6}{3 Go}
  \alabel{8}{4 Go}

\end{tikzpicture}


\caption[Espace d'adressage d'un processus]{L'espace d'adressage d'un processus.
En gris clair, les zones accessibles à tous les niveaux de privilèges: code du
programme, bibliothèques, tas, pile. En gris foncé, la mémoire du noyau,
réservée au mode privilégié.}

\label{fig:memmap}
\end{figure}
% }}}

Grâce à la mémoire virtuelle, chaque processus possède sa propre vue de la
mémoire dans son espace d'adressage (figure~\ref{fig:memmap}), et donc chacun
gère un ensemble de tables de pages et une valeur de \crtrois{} associée (ce
mécanisme a été abordé page~\pageref{page:mem-virt}). Au moment de changer le
processus en cours, l'ordonnanceur charge donc le \crtrois{} du nouveau
processus.

Les adresses basses (inférieures à \texttt{PAGE\_OFFSET} = 3 Gio =
\hexa{c0000000}) sont réservées à l'utilisateur. On y trouvera par exemple:
le code du programme,
les données du programme (variables globales),
la pile utilisateur,
le tas (mémoire allouée par \texttt{malloc} et fonctions similaires), ou encore
les bibliothèques partagées.

Au dessus de \texttt{PAGE\_OFFSET}, se trouve la mémoire réservée au noyau.
Cette zone contient le code du noyau, les piles noyau des processus, etc.

\label{sec:impl-syscall}

Les programmes utilisateur s'exécutant en \ring{3}, ils ne peuvent pas contenir
d'instructions privilégiées, et donc ne peuvent pas accéder directement au
matériel. Pour que ces programmes puissent interagir avec le système (afficher
une sortie, écrire sur le disque…) le mécanisme des appels système est
nécessaire. Il s'agit d'une interface de haut niveau entre les \emph{rings} 3 et
0. Du point de vue du programmeur, il s'agit d'un ensemble de fonctions C
\enquote{magiques} qui font appel au système d'exploitation pour effectuer des
opérations.

Voyons ce qui se passe derrière la magie apparente. Une explication plus
détaillée est disponible dans la documentation fournie par
Intel~\cite{intelsys}.

\paragraph{Dans la bibliothèque C}

Il y a bien une fonction \texttt{getpid} présente dans la bibliothèque C du
système. C'est la fonction qui est directement appelée par le programme. Cette
fonction commence par placer le numéro de l'appel système (noté
\texttt{\_\_NR\_getpid}, valant 20 ici) dans \eax, puis les arguments éventuels
dans les registres (\ebx, \ecx, \edx, \esi{} puis \edi). Une interruption
logicielle est ensuite déclenchée (\verb!int 0x80!).

\paragraph{Dans la routine de traitement d'interruption}

Étant donné la configuration du processeur\footnote{Il est impropre de dire que
le processeur est configuré --- tout dépend uniquement de l'état de certains
registres, ici la \emph{Global Descriptor Table} et les \emph{Interrupt
Descriptor Tables}.}, elle sera traitée en \ring{0}, à un point d'entrée
prédéfini (dans \verb!arch/x86/kernel/entry_32.S!,
figure~\ref{fig:entry-syscall})

\begin{figure}[h]
    \insertcode{entry-syscall.s}
    \caption{Point d'entrée des appels système}
    \label{fig:entry-syscall}
\end{figure}

L'exécution reprend donc en \ring{0}, avec dans \esp{} le pointeur de pile noyau
du processus. Les valeurs des registres ont été préservées; la macro
\texttt{SAVE\_ALL} les place sur la pile. Ensuite, à l'étiquette
\texttt{syscall\_call}, le numéro d'appel système (toujours dans \eax) sert
d'index dans le tableau de fonctions \texttt{sys\_call\_table}.


\paragraph{Dans l'implantation de l'appel système}

Puisque les arguments sont en place sur la pile, comme dans le cas d'un appel de
fonction \enquote{classique}, la convention d'appel \emph{cdecl} est respectée.
La fonction implantant l'appel système, nommée \texttt{sys\_getpid}, peut donc
être écrite en C. On la trouve dans \texttt{kernel/timer.c}
(figure~\ref{fig:syscall-def}).

\begin{figure}[h]
    \insertcode{syscall-definition.c}
    \caption{Fonction de définition d'un appel système}
    \label{fig:syscall-def}
\end{figure}

La macro \texttt{SYSCALL\_DEFINE0} nomme la fonction \texttt{sys\_getpid}, et
définit entre autres des points d'entrée pour les fonctionnalités de débogage du
noyau. À la fin de la fonction, la valeur de retour est placée dans \eax,
conformément à la convention \emph{cdecl}.

\paragraph{Retour vers le ring 3}

Au retour de la fonction, la valeur de retour est placée à la place de \eax{} là
où les registres ont été sauvegardés sur la pile noyau
(\texttt{PT\_EFLAGS(\%esp)}). % chktex 36
L'instruction \texttt{iret} (derrière la macro
\texttt{INTERRUPT\_RETURN}) permet de restaurer les registres et de repasser en
mode utilisateur, juste après l'interruption. La fonction de la bibliothèque C
peut alors retourner au programme appelant.

\section{Risques}

Ainsi que décrit dans la section~\ref{sec:secu-syscalls}, cela peut poser un
problème de manipuler des pointeurs contrôlés par l'utilisateur au sein d'une
routine de traitement d'appel système.

Si le déréférencement est fait sans vérification, un utilisateur mal intentionné
peut forger un pointeur vers le noyau (en déterminant des adresses valides dans
l'espace noyau entre \hexa{c0000000} et \hexa{ffffffff}). En provoquant
une lecture sur ce pointeur, des informations confidentielles peuvent fuiter ;
et, en forçant une écriture, il est possible d'augmenter ses privilèges, par
exemple en devenant super-utilisateur (\emph{root}). En pratique, il n'est pas
toujours possible d'accéder à la mémoire. La mémoire utilisateur peut par
exemple avoir été placée en zone d'échange sur le disque, ou \emph{swap}. À ce
moment là, l'erreur provoquera tout de même un déni de service. Plus de détails
sur ce mécanisme, et le fonctionnement de la mémoire virtuelle dans Linux,
peuvent être trouvés dans~\cite{userspaceaccess}.

% TODO déni: à ref dans l'exemple 2

\section{Premier exemple de bug : pilote Radeon KMS}

On décrit le cas d'un pilote vidéo qui contenait un bug de pointeur utilisateur.
Il est répertorié sur \url{http://freedesktop.org} en tant que bug \#29340.

Pour changer de mode graphique, les pilotes de GPU peuvent supporter le
\emph{Kernel Mode Setting} (KMS).

Pour configurer un périphérique, l'utilisateur communique avec le pilote noyau
avec le mécanisme d'\emph{ioctls} (pour \emph{Input/Output Control}).
Ils sont similaires à des appels système, mais spécifiques à un périphérique
particulier. Le transfert de contrôle est similaire à ce qui a été décrit dans
la section précédente: les applications utilisateurs appellent la fonction
\texttt{ioctl()} de la bibliothèque standard, qui provoque une interruption.
Celle-ci est traitée par la fonction \texttt{sys\_ioctl()} qui appelle la
routine de traitement dans le bon pilote de périphérique.

Les fonctions implantant un \emph{ioctl} sont donc vulnérables à la
même classe d'attaques que les appels système, et donc doivent être écrits avec
une attention particulière.

Le code de la figure~\ref{fig:kms-buggy} est présent dans le pilote KMS pour les
GPU AMD Radeon.

\begin{figure}[h]
\insertcode{kms-buggy.c}
\caption{Code de la fonction \texttt{radeon\_info\_ioctl}}
\label{fig:kms-buggy}
\end{figure}

On peut voir que l'argument \texttt{data} est converti en un \texttt{struct
drm\_radeon\_info *}. Un pointeur \texttt{value\_ptr} est extrait de son champ
\texttt{value}, et finalement ce pointeur est déréferencé.

Cependant, l'argument \texttt{data} est un pointeur vers une structure (allouée
en espace noyau) du type donné dans la figure~\ref{fig:kms-info}, dont les
champs proviennent d'un appel utilisateur de \verb!ioctl()!.

\begin{figure}[h]
\insertcode{kms-info.c}
\caption{Définition de \texttt{struct drm\_radeon\_info}}
\label{fig:kms-info}
\end{figure}

Pour mettre ce problème en évidence, nous avons annoté la fonction
\texttt{radeon\_info\_ioctl} de telle manière que son second paramètre soit un
pointeur noyau vers une structure contenant un champ contrôlé par l'utilisateur,
\texttt{value}.

L'intégralité de ce code peut être trouvée en annexe~\ref{cha:code-noyau}.

\begin{figure}
  \insertcode{radeon-git-mini.diff}

  \caption[Patch résolvant le problème de pointeur utilisateur.]
      {Patch résolvant le problème de pointeur utilisateur.
       La ligne précédée par un signe \texttt{-} est supprimée et remplacée
       par les lignes précédées par un signe \texttt{+}.
      }
\label{fig:linux-patch}
\end{figure}

La bonne manière de faire a été publiée avec le numéro de \emph{commit}
\texttt{d8ab3557} (figure~\ref{fig:linux-patch}) (\texttt{DRM\_COPY\_FROM\_USER}
étant une simple macro pour \texttt{copy\_from\_user}). Dans ce cas, on
n'obtient pas d'erreur de typage.

\section{Second exemple : \texttt{ptrace} sur architecture Blackfin}

Le noyau Linux peut s'exécuter sur l'architecture Blackfin, qui est spécialisée
dans le traitement du signal. Le problème de manipulation des pointeurs
utilisateur auquel nous nous intéressons peut également s'y produire.

En particulier nous nous intéressons à l'appel système \texttt{ptrace}. Il
permet à un processus d'accéder à la mémoire et de contrôler l'exécution d'un
autre processus, par exemple à des fins de débogage. Ainsi,
\texttt{ptrace(PTRACE\_PEEKDATA, p, addr)} renvoie la valeur du mot mémoire à
l'adresse \texttt{addr} dans l'espace d'adressage du processus \texttt{p}.

Comme pour la plupart des appels système, la fonction \texttt{ptrace} est
dépendante de l'architecture. Le deuxième exemple que nous présentons concerne
l'implantation de celle-ci pour les processeurs Blackfin,
figure~\ref{fig:ptrace-blackfin}.

Dans d'anciennes versions de Linux\footnote{Jusqu'à la version 2.6.28 --- ce bug
a été corrigé dans le \emph{commit} \texttt{7786ce} en remplaçant l'appel à
\texttt{memcpy} par un appel à \texttt{copy\_from\_user\_page}}, cette fonction
appelle \texttt{memcpy} au lieu de \nolinkurl{copy\_from\_user} pour lire dans
la mémoire du processus. La ligne problématique est préfixée par \texttt{/*
=\textgreater{} */}. En théorie, si un utilisateur passe un pointeur noyau à la
fonction \texttt{ptrace}, il pourra lire des données du noyau. L'appel
\texttt{ptrace (PTRACE\_PEEKDATA, p, addr)} permet ainsi non seulement de lire
les variables du processus \texttt{p} si \texttt{addr} est une adresse dans
l'espace utilisateur (ce qui est le comportement attendu), mais aussi de lire
dans l'espace noyau si \texttt{addr} y pointe (ce qui est un bug de sécurité).

\begin{figure}
    \insertcode{blackfin-ptrace.c}
    \caption{Implantation de \texttt{ptrace} sur architecture Blackfin}
    \label{fig:ptrace-blackfin}
\end{figure}

On peut repérer ce bug par simple relecture pour commencer. On commence par
remarquer que l'argument \texttt{addr}, malgré son type \texttt{long}, est en
réalité un \texttt{void *} provenant directement de l'espace utilisateur. C'est
en effet le même argument \texttt{addr} de l'appel système \texttt{ptrace}.
Cet argument correspond à l'adresse à lire dans l'espace mémoire du processus.
Comme il est passé à \texttt{memcpy}, aucune vérification n'est faite avant la
copie. La valeur pointée par \texttt{addr} sera copiée, même si elle est en
espace noyau.

En annotant correctement les types, on peut donc détecter ce bug: le type
correct de \texttt{addr} est \ptrU{\tInt}, et celui de \texttt{memcpy} est
$(\ptrK{\tInt}, \ptrK{\tInt}, \tInt) → \ptrK{\tInt}$. Il est donc impossible de
lui passer cet argument. Remarquons que le type de \texttt{memcpy} en C utilise
des pointeurs de type \texttt{void *}. Pour les traiter correctement on pourrait
utiliser du polymorphisme, mais dans ce cas précis utiliser le type \ptrK{\tInt}
est suffisant.

\paragraph{Remarque}

% TODO c'est un point intéressant ↓
% ?? clarifie et développe
% développe la fin et en particulier le problème de sécurité

En pratique, la copie se fait sous un test forçant \texttt{addr} à être entre
\nolinkurl{FIXED\_CODE\_START} et \texttt{FIXED\_CODE\_END}, c'est-à-dire en
espace utilisateur. Cela enlève le potentiel problème de fuite de données, mais
cela reste un problème de sécurité car il est possible de provoquer des fautes
mémoire dans le noyau si la copie par \texttt{memcpy} se fait alors que l'espace
utilisateur n'est pas chargé en mémoire.

\section{Procédure expérimentale}
\label{sec:demo-unif}

Pour utiliser notre système de types, plusieurs étapes sont nécessaires en plus
de traduire le noyau Linux en \langname{}.

Afin de réaliser l'analyse, il faut annoter les sources pour créer un
environnement initial (via la variable \texttt{exttbl} décrite en
section~\ref{sec:ptrtype-archi}). Plus précisément, pour chaque source de
pointeurs utilisateur, on ajoute un commentaire spécial \texttt{!npk
userptr\_fieldp x f}, qui indique que \texttt{x} est un pointeur vers une
structure contenant un pointeur utilisateur dans le champ \texttt{f}. En fait,
il unifie le type de \texttt{x} avec $\{ f: \ptrU{t} ; … \}~*$ où $t$ est une
inconnue de type.

Par rapport au code complet présent dans l'annexe~\ref{cha:code-noyau},
l'expression calculant \texttt{value_ptr} est également simplifiée. Dans le code
d'origine, \texttt{info->value} est transtypé en \texttt{unsigned long} puis en
\texttt{uint32\_t *}. En \newspeak, cela correspond à des opérateurs
\texttt{PtrToInt} \linebreak et \texttt{IntToPtr} mais, si on les autorise, on
casse le typage puisqu'il est alors possible de transformer n'importe quel type
en un autre. De plus, on modifie la définition du type \texttt{struct
drm\_radeon\_info} pour que son champ \texttt{value} ait pour type
\texttt{uint32\_t *} plutôt que \texttt{uint64\_t}. En effet, dans ce cas
d'étude, cet entier est uniquement utilisé en tant que pointeur au cours de
toute l'exécution.

\begin{figure}[p]
\insertcode{ex-drm.c}

\caption{Cas d'étude minimisé et annoté}
\label{fig:ex-drm}
\end{figure}

\begin{figure}[p]

    \insertcode{ex-drm-ok.c}

    \caption{Cas d'étude minimisé et annoté -- version correcte}
    \label{fig:ex-drm-ok}

\end{figure}

En ce qui concerne les fonctions de manipulation de pointeurs fournies
par le noyau (\texttt{get\_user}, \texttt{put\_user},
\texttt{copy\_from\_user}, \texttt{copy\_to\_user}, etc.), on ajoute à
l'environnement global leur type correct.

Enfin, on peut lancer l'inférence de type. Ainsi, sur l'exemple de la
figure~\ref{fig:ex-drm}, on obtient la sortie suivante:

\begin{Verbatim}
05-drm.c:18#1 - Cannot unify qualifiers:
  Kernel
  User
\end{Verbatim}

Cela indique qu'on a essayé d'unifier un type de la forme $\ptrK{t}$ avec un
type de la forme $\ptrU{t}$, en précisant l'emplacement où la dernière
unification a échoué. Le message d'erreur fait apparaître des qualificateurs
explicites: comme on l'explique page~\pageref{page:qualifs-pas-qualifs}, c'est
un reste d'une première présentation basée sur les qualificateurs.
Dans l'implantation, il est cependant agréable de manipuler ces étiquettes
séparément des pointeurs auxquels elles sont attachées.

% TODO idem sur #2

La version correcte minimisée correspond à la figure~\ref{fig:ex-drm-ok}. Pour
celle-ci, l'inférence se fait sans erreur.

% TODO le faire

\section*{Conclusion}

Après voir décrit notre l'implantation de notre solution, on a montré comment
celle-ci peut s'appliquer à détecter un bug dans le noyau Linux. La première
difficulté est de traduire en \newspeak le code source écrit dans le dialecte
GNU C.

En partant d'un bug existant, on montre que la version originale du code
(incluant une erreur de programmation) ne peut pas être typée, alors que sur la
version corrigée on peut inférer des types compatibles.

Le prototype décrit dans le chapitre~\ref{cha:implem} peut donc s'adapter à
détecter des bugs dans le noyau Linux. Pour le moment, il nécessite du code
annoté, mais des travaux sont en cours pour permettre de passer automatiquement
dntes portions plus importantes du noyau Linux. Une partie importante du travail
est déjà faite puisque la phase de compilation depuis le code source supporte
maintenant le dialecte GNU C.

% vim: spelllang=fr
