\long\def\partintro{
Après avoir décrit le contexte général de ces travaux, nous décrivons leurs
enjeux.

Le chapitre~\ref{cha:os} explore plus en détail le fonctionnement d'un système
d'exploitation, y compris la séparation du code en plusieurs niveaux de
privilèges. L'architecture Intel 32 bits est prise comme support. En
particulier, le mécanisme des appels système est décrit et on montre qu'une
implantation naïve de la communication entre espaces utilisateur et noyau casse
toute isolation.

Le chapitre~\ref{cha:etatdelart} consiste en un tour d'horizon des techniques
existantes en analyses de programmes. Ces analyses se centrent autour des
problèmes liés à la vérification de code système ou embarqué, y compris le
problème de manipulation mémoire évoqué dans le chapitre~\ref{cha:os}.

On conclut en présentant notre technique: \langname, un langage permettant de
typer des programmes impératifs, plus précisément en ajoutant des types
pointeurs abstraits.
}
