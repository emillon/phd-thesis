\begin{headingpage}

Le chapitre~\ref{cha:os} décrit le contexte de ces travaux, notamment le
fonctionnement général d'un système d'exploitation et la séparation du code en
plusieurs niveaux de privilèges. Le mécanisme d'appels système est décrit, et on
montre qu'une implantation naïve de la communication entre espaces utilisateur
et noyau casse toute isolation. On présente la situation prise par le noyau
Linux : séparer deux classes de pointeurs sensées être indépendante.

Le chapitre~\ref{cha:etatdelart} consiste en un tour d'horizon des techniques
existantes en analyses de programmes. Ces analyses se centrent sur, mais ne se
limitent pas au problème de manipulation mémoire évoqué dans le
chapitre~\ref{cha:os}.

% TODO
\end{headingpage}
