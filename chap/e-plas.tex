Manipulating user-provided pointers in the kernel of an operating system can
lead to security flaws if done in an incautious manner. We present an efficient
system to detect and prevent this class of erroneous memory manipulation.

At the core of our approach is \langname, an imperative language that we equip
with a qualified type system, where two kinds of pointers are distinguished:
\emph{safe} pointers, whose value is statically proved to be controlled by the
kernel, and \emph{unsafe} ones, whose value comes from userspace through
run-time system calls. Dereferencing unsafe pointers is forbidden in a static
manner by the means of a strong type system.

A concrete case study is described based on a bug that affected a video driver
in the Linux kernel. We also explain a technique to automatically translate GNU
C code to our core language, which will enable us to analyze larger fractions of
the kernel in order to find similar vulnerabilities.

\section{Introduction}

Modern operating systems such as GNU/Linux are comprised of several independent
programs: the kernel, which handles the machine's hardware as well as trusted
core features; and user applications.

These programs run with different privilege levels: low-level operations can
only be executed by the kernel. They also have different views on the memory. If
a user program had access to the kernel's internal data structures, it could use
it to obtain information it is normally denied, or gain extra privileges (for
example getting access to the \texttt{root} account). This is why the kernel
stores its code and data in a place in memory that is not accessible to
unprivileged programs. On an Intel 32-bit system, it corresponds to addresses
higher than \texttt{0xc0000000}, or 3 GiB.

Because user-mode programs cannot run low-level instructions, they cannot do
anything interesting on their own. They have to use the kernel's facilities to
perform input/output or manipulate the operating system's abstractions (such as
creating a new process). To this end, they can communicate with the kernel
via the narrow interface of \emph{system calls}.

For example, every time a user process calls the \texttt{read()} function, an
architecture-specific mechanism invokes the kernel-mode function
\texttt{sys\_read()}. The user passes a file descriptor, a pointer and an amount
of bytes to fill and the kernel will actually perform input/output and copy the
resulting bytes on the user's buffer.

But if the user supplies a pointer whose value is in the kernel's reserved area,
this zone will be overwritten, potentially leading to a security breach. In a
sense, the kernel has been abused because it accessed the memory with its own
privileges instead of the originator's. This is known as the \emph{confused
deputy problem}\cite{hardy88confused}.

When implementing a system call, it is thus necessary to forbid direct
dereference of a pointer whose value can be controlled by the user. In case of a
system call with a pointer argument, one has to dynamically verify that the
address provided by the user lies within the process's memory space.

The object of this work is to detect the places where it is necessary to insert
dynamic checks. To this end, we augment the type system of a subset of the C
programming language with \emph{type qualifiers}: instead of having a single
pointer type, we add syntactic rules to distinguish between safe pointers that
can be dereferenced with the \texttt{*}, \texttt{->} and \texttt{[]} operators ;
and unsafe pointers that have to go through a special API provided by the
kernel, consisting most notably of the two functions \texttt{copy\_from\_user}
and \texttt{copy\_to\_user}.

In Section~\ref{sec:related}, we present the different techniques that have been
used to tackle this issue. Then we present our solution, consisting of an
imperative language (Sections~\ref{sec:lang} and~\ref{sec:eval}) that is
equipped by a static type system (Section~\ref{sec:types}), whose safety is
discussed (Section~\ref{sec:safety}). In Section~\ref{sec:implem}, we present
our implementation, and in Section~\ref{sec:exp} how we used it on a bug
affecting a video driver in the Linux kernel. Section~\ref{sec:limit} describes
the limitations of our current approach. Finally, Section~\ref{sec:concl}
concludes and discusses future work.

\section{Related Work}
\label{sec:related}

Dynamic taint checking consists in adding taint labels to data, so that every
sensitive access is checked at run-time. This goes well with dynamic languages,
with the famous example of Perl's ``tainted mode''~\cite{perlCamelBook}. It is
also possible, but harder, to use this technique for compiled
languages~\cite{clause-etal-issta07}.

Adding a strong type system to C is the core idea of
CCured~\cite{ccured-toplas}. It copes with the loose memory model of C by adding
dynamic checks when static guarantees cannot be obtained. However, it requires
dynamic instrumentation and is meant to stay activated, incurring a
non-negligible run-time cost.

%, similar to the work on typestates~\cite{tse12-typestate}
The CQual system\cite{toplas-quals} provides a generic type qualifier framework
on top of existing type systems. It has been used to infer \texttt{const}
annotations~\cite{pldi99}, tainting analysis for format strings~\cite{usenix01}
and flow-sensitive properties such as locking invariants~\cite{pldi02}. It has even been applied to
the class of user/kernel vulnerabilities discussed here~\cite{cquk-usenix04}.
Because it adds a qualifier at every level, it is more generic but also more
complex than our method.

% ~\cite{Cousot77}
Abstract interpretation is a generic framework to analyze
dynamic properties of programs. As such, it can be used to follow data
flow between objects~\cite{liang2012taint}. It has the disadvantage of detecting
a lot of false positive errors because it requires a relatively precise view of
the program. It is also best used for programs that have a single entry point,
unlike kernel modules.

Saturn~\cite{paste07} is a system for analyzing system code written in C. It
tackles the problem of user pointers using points-to analysis~\cite{oakland08}.
Like abstract interpretation, it aims at being precise at the expense of time in
some corner cases.

For the development of the Linux kernel, the Sparse~\cite{TorvaldsSparse} custom
tool is used to check the origin of pointers. However, it requires the
programmer to manually annotate the program, and is not sound.

\section{Core Language}
\label{sec:lang}

We present \langname, a memory-safe imperative programming language. Let us go
through its major features.

\paragraph{Scalar data:} there are four types of scalars: integers, floats,
pointers and a special ``unit'' value. Integers are 32 bit wide signed integers,
akin to the \texttt{int32\_t} data type in C, but this size is arbitrary. Floats
are 32 bit wide floating point numbers with their usual semantics,
Pointers are used to reference any modifiable piece of data, as embodied by the
concept of \emph{left-values} described below. Finally, the special unit value,
written \eUnit is the one that is returned by procedures that do not return a
useful value. It is similar to the \texttt{unit} type of some functional
languages. A particular $\eTaint{e}$ construct allows us to construct
expressions annotated as coming from userspace, emulating a system call, and two
operators $\uGet{\cdot}{\cdot}$ and $\uPut{\cdot}{\cdot}$ provide safe copy from
and to userspace.

{ \small
\begin{align*}
  \gramdefshort{Expressions}{e}
                 { c               }{ Constant }
                 { lv              }{ Left-value }
                 { \opun~e         }{ Unary operation }
                 { e~\opbin~e      }{ Binary operation }
                 { \&~lv           }{ Pointer }
                 { lv ← e          }{ Assignment }
                 { \{ l_1 : e_1
                    ; …
                    ; l_n : e_n \} }{ Structure }
                 { [ e_1 ;…; e_n ] }{ Array }
                 { \mathrm{fun} (x_1, …, x_n) \{ i \} }{ Function }
                 { e (e_1, …, e_n) }{ Function call }
                 { \eTaint{e}      }{ Tainted value }
                 { \uGet{lv}{e}    }{ Load from userspace }
                 { \uPut{e}{e}     }{ Store to userspace }
                 {END} \\
  \\
  \gramdefshort{Constants}{c}
               { n      }{ Integer }
               { d      }{ Float }
               { \eNull }{ Null pointer }
               { \eUnit }{ Unit value }
               {END}
\end{align*} }%


\paragraph{Composite data:} there are two kinds of composite data: arrays and
structures. Arrays are homogeneous pieces of memory with a fixed, statically
known size. Elements within an array can be accessed with an integer
index.
In the case of an array with $n$ elements,
the first one is accessed with the index $0$,
and the last one with $n-1$.
Trying to access to an element
with any other index results in a run-time error.
Arrays are written $[ e_1; …; e_n ]$ with $n>0$. Unlike in C, there is no
particular relation between arrays and pointers: arrays are plain values that
can be passed around.

The other kind of composite data is structures, that are an heterogeneous
collection of variables of a fixed size. Elements are accessed with a
\emph{label} $l_S$, that is always statically known (labels are not first-class
elements of the language). The subscript $S$ index is placed here by our
translator to guide type inference, and is ignored by the evaluator. Structures
are written $\{ l_1 : e_1 ; … ; l_n : e_n \}$ with $n>0$.

\paragraph{Memory} is organised in variables, that hold scalar or composite
data. It is split between a stack of local variables, and a set of global
variables. Local variables are structured in a list of stack frames that are
used to hold information for each function call. Global variables are on the
other hand an unstructured set accessible anywhere in the program. We suppose
that inside every function, all locals have a distinct name ; same goes for
globals. Local variables from two different function can have the same name.

\paragraph{Left-values:} Inside expressions, it is possible to directly access
visible variables, that is the ones that are part of the most recent stack
frame, and global ones. It is also possible to manipulate other parts of the
memory by the means of a pointer. Pointers hold enough information
to address in a non ambiguous manner any part of the memory.%
{\small \begin{align*}
  \gramdefshort{Left-values}{lv}
                  { x      }{ Variable }
                  { \hspace{-2pt}*lv    }{ Dereference }
                  { lv.l_S }{ Field access }
                  { lv[e]  }{ Indexed access }
                  {END}
\end{align*}}%
\paragraph{Operators} are used to combine expressions together. We support C's
operators on scalar data. To help the type system, we distinguish between
integer operations and floating point operations by writing the latter with an
extra dot: for example $3 + 2 = 5$ and $3.14~+_.~2.72 = 5.86$. Equality (and
difference) tests can be done on scalar data as well as on composite data. In
that case, it is done recursively on their members. Pointers are equal if they
describe the same path in memory. Pointer arithmetic is possible by the means of
operators $+_p$ and $-_p$, that move a pointer inside an array (if it used on
another type of pointer, it results in a run-rime error $\serr{ptr}$).

\paragraph{Functions} are first-class expressions that can be assigned to
variables that can then be called. However, unlike in some other programming
languages, they are not lexical closures, meaning that no capture of variables
from enclosing scopes happen at the site of a function declaration.

This is a compromise between C where functions can only be global, and
functional languages where it is necessary to build closures containing (pointers
to) captured variables at each function call. This also eliminates the need for
special-cased ``function pointers''.

Each function call creates a stack frame to hold room for the function's
parameters and local variables. Upon return, the most recent frame is destroyed.
A specificity of our system is that at this moment, we also clear references
that are about to become invalid. More precisely, we turn to \eNull every
pointer that refer to the latest stack frame. Using this technique, we can avoid
so-called \emph{dangling} pointers.

Note that this is a dynamic step and requires a full memory scan, which is very
inefficient. This is done only for soundness reasons ; as discussed in
Section~\ref{sec:limit}, if a preliminary analysis proves that this is
unnecessary we can skip this step.

\paragraph{Instructions} inside functions includes the empty instruction,
sequence, evaluation of an expression (including assignment), or variable
declation. Control flow is restricted to alternative (\textsc{If}) and loops
(\textsc{While}). The \textsc{Return} construct also exits early from the
current function.
%%%%%
{ \small \begin{align*}
  \gramdefshort{Instructions}{i}
                 { \iPass          }{Empty instruction}
                 { i_1;i_2         }{Sequence}
                 { e               }{Expression}
                 { \iDecl{x}{e}{i} }{Declaration}
                 { \iIf{e}{i}{i}   }{Alternative}
                 { \iWhile{e}{i}   }{Loop}
                 { \iReturn{e}     }{Function return}
                 {END}
\end{align*}}
\vspace{-5mm}
\paragraph{A program} is a sequence of top-level sentences, of the form of
global declarations or expressions. A declaration creates a global variable
that is accessible everywhere after in the whole program.
On the other hand, a plain expression is only evaluated for its side-effects
\footnote{
  There is no implicit entry point such as a $\mathrm{main}()$ function, so
  it is necessary to manually insert a call it.
}
.%
{ \small
\begin{align*}
  \gramdefshort{Toplevel sentences}{s}
                 { x = e }{Global variable}
                 { e     }{Expression evaluation}
                 {END}
  \\
  P \gramisa & (s_1, …, s_n) & {\textrm{\larger \textbf{Program}}}
\end{align*}}%
\begin{figure*}

  \centering
  {\small {

  \begin{minipage}{0.29\textwidth}
  \begin{align*}
  \gramdefshort{Values}{v}
      { \widehat{c}     }{Constant}
      { \widehat{f}     }{Function}
      { \widehat{\&}~φ  }{Reference}
      { \widehat{
         \{ l_i : v_i;… \}
       }                }{Structure}
      { \widehat{
        [ v_i ;… ]
        }               }{Array}
      { Ω               }{Error}
      {END}
  \\
  \gramdefshort{Paths}{φ}
     { a    }{Address}
     { *φ   }{Dereference}
     { φ.l  }{Field}
     { φ[n] }{Index}
     { \vTainted{φ} }{Tainted value}
     {END}
  \end{align*}
  \end{minipage}
  \begin{minipage}{0.4\textwidth}
  \begin{align*}
  \gramdefshort{Stack}{s}
    { [~] }{Empty stack}
    { \{ x_i ↦v_i; …\} :: s}{Extra frame}{END}
  \\
  \gramheadershort{Memory}{m} \\
    & (s, & \textrm{Stack,} \\
    & \{ x_1 ↦v_1; … \}) & \textrm{globals} \\
  \\
  \gramdefshort{Addresses}{a}
     { (n, x) }{Local variable}
     { (x)    }{Global variable}
     {END}
  \end{align*}
  \end{minipage}
  \begin{minipage}{0.3\textwidth}
  \begin{align*}
  \gramdefshort{Errors}{Ω}
    { \serr{array} }{Buffer overflow}
    { \serr{ptr}   }{Invalid dereference}
    { \serr{div}   }{Division by zero}
    { \serr{field} }{Bad field}
    { \serr{taint} }{Isolation error}
    {END}
  \\
  \gramdefshort{Interpreter}{Ξ}
    { \msi{m}{e} }{Expression, memory}
    { \msi{m}{i} }{Instruction, memory}
    { Ω          }{Error}
    {END}
  \end{align*}
  \end{minipage}
}}
  \caption{Memory and values}
  \label{fig:interp}
\end{figure*}%
\vspace{-7mm}
\section{Evaluation}
\label{sec:eval}

Operational semantics described thereafter are a binary relation between states
$Ξ$ of an interpreter described in Figure~\ref{fig:interp}.

Every syntactic constant $c$ has a corresponding value $\widehat{c}$. The same
holds for functions, because as they do not capture their lexical
environment, there is no need to build a closure.

Composite values also have their matching semantic parts: an array of values is
a value, and a structure of values is a value. The same $\widehat{\cdot}$
notation is used to remove the ambiguity.

Pointers are reduced to a memory path $φ$ that is an evaluated left-value: type
information is removed from the field names, and array indices are evaluated to
integers. Also, names are made explicit by adjoining the name of local variables
with the index of the stack frame they belong to. This is necessary in the cases
where two frames hold a variable with the same name; for example in the presence
of recursive functions.

The semantics described here are given in the style of Plotkin's Structural
Operational Semantics\cite{sos-jlap}: expressions are gradually reduced to
constants, and instructions are simplified until a terminal instruction of the
form $\iPass$ or $\iReturn{v}$.

\paragraph{Reading memory}

A variable is evaluated in two steps ; first, its address is computed.%
{ \small \[
  \semrule{Phi-Var}
\] }%
$\mathrm{Lookup} : \textsc{Mem} × \textsc{Id} → \textsc{Addr}$
is an operator mapping variable names to addresses. That is,
it performs name resolution. It returns $(n, x)$ where $n$ is the size of the
stack if a local variable of this name exists ; or then $(x)$ if a global of
this name exists.

Then, an address is a path $φ$, that is used to access the corresponding value
stored in $m$.

{\small \[
  \semrule{Exp-Lv}
\]}%

We omit here a formal definition of $m[\cdot]_Φ$. In a nutshell, it walks the
memory to return the value ultimately described by $φ$. It can raise
$\serr{array}$ if a bad array access occurs or $\serr{ptr}$ if \eNull is
dereferenced.

\paragraph{Writing memory}

The $m[\cdot]_Φ$ notation is extended to perform updates. We note $m[φ ← v]_Φ$
the memory state obtained by replacing the value at $φ$ by $v$. The same kind of
errors can occur.

{\[ \small
  \semrule{Exp-Set}
\] }%

\paragraph{Variable declaration} extends the current stack frame with a new
binding $x↦v$.%
{ \small \[ \semrule{Decl} \]}%
It relies on the following operator:%
{ \small
\begin{align*}
  \mathrm{Extend} : \textsc{Mem}×\textsc{Id}×\textsc{Val} &→ \textsc{Mem}\\
  \mathrm{Extend}(f::fs, g), x) &= (((x↦v)::f) :: fs), g) \\
\end{align*} }%
\paragraph{Function call} uses a complex rule, because it handles the calling
convention (pushing and popping parameters on the stack). We suppose that all
functions exit through a $\iReturn{\cdot}$ instruction.%
{\small \[ \semrule{Exp-Call} \] }%
$\mathrm{Push} : \textsc{Mem} × \textsc{Frame} → \textsc{Mem}$
adds a new stack frame, and
$\mathrm{Pop} : \textsc{Mem} → \textsc{Mem}$
removes it.

$\mathrm{Cleanup} : \textsc{Mem} → \textsc{Mem}$ removes references to dangling
pointers. It walk all the memory and replaces references to the current stack
frame\footnote{That is, every address of the form $(n,x)$
where $n$ is the current size of the stack.}
with $\eNull$. For example:%
{\small \begin{align*}
  \mathrm{Cleanup} &
  ([ [x ↦ 0 ] ; [ p → \{t: \&(2, x); u: 5\} ] ], [~]) \\
                   &
  ([ [x ↦ 0 ] ; [ p → \{t: \eNull; u: 5\} ] ], [~]) \\
\end{align*} }%
\paragraph{System calls} are emulated through the $\eTaint{\cdot}$ operator. For
example, the bridge between the functions \texttt{read()} and
\texttt{sys\_read()} is the following. Integer parameters don't need to be
converted, as they do not have qualifiers.

\begin{verbatim}
read = fun (fd, p, n) {
  Decl up = Taint(p) in {
    sys_read (fd, up, n)
  }
}
\end{verbatim}

Its evaluation is:%
%{\small \[ \semrule{Exp-Tainted} \]} % TODO

To allow a safe copy between userspace and the kernel, the Linux kernel provides
the macros \texttt{get\_user()} or \texttt{put\_user()}, which check and copy
scalar data, or \url{copy_from_user()} and \texttt{copy\_to\_user()} which act
similarly to \texttt{memcpy()}. All of these constructs return the error
\texttt{-EFAULT} when the user pointer is within the bounds of the kernel's
memory. To model them, we provide the $\uGet{\cdot}{\cdot}$ and
$\uPut{\cdot}{\cdot}$ operators that can copy a value at a time.%

Their semantics is not detailed, but it consists in checking that the user
pointer is of the form $\vTainted{φ}$. If it is the case, then a copy from $φ$
(in the case of $⇐_U$) or to $φ$ (in the case of $⇒_U$) is performed and the
expressions reduces to $1$. Else, no copy is done and it reduces to $0$.

\paragraph{Contexts}

To simplify the presentation of evaluation rules, we use reduction contexts, as
described by Felleisen \emph{et al.}~\cite{wright92syntactic,tcs92-fh}. The idea
is that if it is possible to reduce an expression $e_1$ to an expression $e_2$,
then it is possible to reduce a bigger expression where $e_1$ appears by
substituting it by $e_2$.

To make this work we have to proceed in three steps: split an expression
according to a context, apply a reduction under a context, and merge the reduced
expression with a context.

\begin{figure}

  {\small {

\begin{align*}
  C \gramisa & \ctxEmpty \\
     \gramor & \ctxOp{C}{e} \gramorx \ctxOp{v}{C} \gramorx \ctxUnOp{C} \\
     \gramor & \ctxSet{C}{e} \gramorx \ctxSet{φ}{C} \\
     \gramor & \{ l_1:v_1 ; … ; l_i:C ; … ; l_n:e_n \} \\
     \gramor & [ v_1 ; … ; C ; … ; e_n ] \\
     \gramor & C (e_1, …, e_n) \gramorx f (v_1, …, C, …, e_n) \\
     \gramor & \eTaint{C} \\
     \gramor & \uGet{C}{e} \gramorx \uGet{φ}{C} \\
     \gramor & \uPut{C}{e} \gramorx \uPut{v}{C} \\
     \gramor & \&~C  \\
     \gramor & \ctxLvDeref{C}
               \gramorx \ctxLvField{C}{l_S}
               \gramorx \ctxLvIndex{C}{e}
               \gramorx \ctxLvIndex{φ}{C} \\
     \gramor & C;i \\
     \gramor & \iIf{C}{i_1}{i_2} \\
     \gramor & \iReturn{C} \\
     \gramor & \iDecl{x}{C}{i}
\end{align*}}\vspace{-5mm}}%

\caption{Reduction contexts}
\label{fig:ctxs}

\end{figure}

Splitting across a context corresponds to defining a grammar of contexts. By
defining these (Figure~\ref{fig:ctxs}), we make an explicit choice of evaluating some constructs in a
left-to-right order. For example, the contexts for binary operations are
$C~\opbin~e$ and $v~\opbin~C$, so it is impossible to start reducing the right
part before the left one has been fully evaluated.


A similar remark can be done for $n$-ary constructs (structure and array
literals, as well as function calls): because there are always values on the
left hand of $C$, arguments are evaluated left-to-right.

The second step is expressed in the following rule. Any transition between
interpreter states can be lifted in a bigger context.%
{\small \[
    \semrule{Ctx}
\]}%
The $\ctxSub{C}{\cdot}$ operator corresponds to the last operation: it pastes an
expression in place of the unique "hole" $\ctxEmpty{}$ that appears in each
context \footnote{ In the definition of C, every alternative produces exactly
  one nonterminal $C$ or one terminal $\ctxEmpty{}$. So, every derivation tree
is linear, and $\ctxEmpty{}$ appears exactly once in every evaluation context. }
.

For example, let us unroll the evaluation of
$3+x$ to the value $\widehat{5}$
starting from a memory state $m = ([~], \{x ↦ \widehat{2}\}) $.
It is also depicted in Figure~\ref{fig:eval-steps}.

\begin{enumerate}[(a)]
\item % (a)
  Let us first remark that $\mm{m}{3}{m}{\widehat{3}}$
  (because of \textsc{Exp-Cst})
\item % (b)
  By applying \textsc{Ctx} from (a)
  with $C = \ctxEmpty{} + x$, we obtain
  $\mm{m}{3+x}{m}{\widehat{3}+x}$.
  Note that with such an evaluation context, the right hand operand does not
  have to be evaluated.
\item % (c)
  $x$, as a variable name, is a left-value. To evaluate it, the first step is to
  turn it into a path $φ = \mathrm{Lookup}(x, m)$ (because of rule
  \textsc{Phi-Var}). Because there are no local variables and a global variable
  named $x$, $φ = (x)$. Note that in this step, no memory lookup was involved
  (only the names of variables were used). So $\mm{m}{x}{m}{(x)}$.
\item % (d)
  In order to evaluate this path into a value, we have to perform a memory
  lookup. The global variable $(x)$ maps to $2$ in $m$, so according to
  \textsc{Exp-Lv}, $\mm{m}{(x)}{m}{\widehat{2}}$.
\item % (e)
  Because of closure of $→$, we get from (c) and (d) that
  $\mm{m}{x}{m}{\widehat{2}}$.
\item % (f)
  By applying \textsc{Ctx} with $C = \widehat{3} + \ctxEmpty{}$ to (e),
  $\mm{m}{\widehat{3} + x}{m}{\widehat{3} + \widehat{2}}$ holds.
  The fact that the left hand operand is a value is important.
\item % (g)
  Because of \textsc{Exp-BinOp},
  $\mm{m}{
  \widehat{3} + \widehat{2}
  }{m}{
  \widehat{3} \widehat{+} \widehat{2}
  }$, ie
  $\mm{m}{
  \widehat{3} + \widehat{2}
  }{m}{
  \widehat{5}
  }$.
\item % (h)
  By closure of $→$, we get from
  (b), (f) and (g) that
  $\mm{m}{3+x}{m}{\widehat{5}}$
  .
\end{enumerate}

\begin{figure}
  \small

\begin{mathpar}

  \inferrule*[left=({\normalfont b})]{
    \inferrule*[left=({\normalfont a})]{ }{\mms{3}{\widehat{3}}}
  }{
    \mms{3+x}{\widehat{3}+x}
  }

  \inferrule*[left=({\normalfont f})]{
    \inferrule*[left=({\normalfont e})]{
      \inferrule*[left=({\normalfont c})]{ }{ \mms{x}{(x)} }
      \\
      \inferrule*[left=({\normalfont d})]{ }{ \mms{(x)}{\widehat{2}} }
    }{
      \mms{x}{\widehat{2}}
    }
  }{
    \mms{\widehat{3}+x}{\widehat{3}+\widehat{2}}
  }

  \inferrule*[left=({\normalfont h})]
    {
      (b)
      \\
      (f)
      \\
  \inferrule*[left=({\normalfont g})]
    { }
    {
      \mms{\widehat{3}+\widehat{2}}{\widehat{3}\widehat{+}\widehat{2}}
    }
    }
    {
      \mms{3+x}{\widehat{5}}
    }
\end{mathpar}

\caption{Evaluation of $3+x$ into $\widehat{5}$}
\label{fig:eval-steps}

\end{figure}

Evaluation rules are given in annex in Figures~\ref{fig:eval-exp}
and~\ref{fig:rules}. Most are simple, but a few deserve to be explained in
detail.

\paragraph{Error propagation} is done in two cases. First, if an expression
produces an error (seen as a value), then the same error (seen as a state $Ξ$)
propagates to the interpreter. Second, if a sub-expression causes an error, then
a bigger expression causes the same error.%
{ \small
\begin{mathpar}
    \semrule{Exp-Err}

    \semrule{Eval-Err}
\end{mathpar}}%

\paragraph{Programs} are evaluated one sentence after another. Declarations add
a global variable.

\begin{mathpar}
    \semrule{T-Exp}

    \semrule{T-Var}
\end{mathpar}

\section{Typing}
\label{sec:types}

We add a static type system to \langname. The goal is to devise a tractable
syntactic method to reject a class of buggy programs.

\begin{figure}

  \centering
  {\small{

  \begin{align*}
  \gramdefshort{Types}{t}
      { \tInt,\tFloat,\tUnit        }{Ground types}
      { t[~]                        }{Array}
      { t~q*                        }{Qualified pointer}
      { S                           }{Structure}
      { (t_1, …, t_n) \rightarrow t }{Function}
      {END}
  \\
  \gramdefshort{Qualifiers}{q}
    { \qKernel }{Kernel data (safe)}
    { \qUser   }{User data (unsafe)}
    {END}
  \\
  \gramheadershort{Structure}{S} \\
  &  \{ l_1 : t_1; … ; l_n : t_n \} & \textrm{Simple structure} \\
  \\
  \gramdefshort{Typing environment}{Γ}
      { [~]        }{Empty environment}
      { (a, t)::Γ' }{Extension}
      {END}
\end{align*}}\vspace{-0.5cm}}

  \caption{Types and typing environments}

  \label{fig:les-types}

\end{figure}%
%
Our types (Figure~\ref{fig:les-types}) are all ground types ; no polymorphism is
present. The main distinctiveness is that instead of having a plain pointer type
$t~*$, our one is decorated with a type qualifier $q$. This annotation expresses
who controls the value of the pointer. If the kernel controls the value of the
pointer, then it cannot be abused. On the other hand, one has to be careful
with user-controlled pointers, because the caller can abuse the kernel and
access reserved memory. The only safe case when dereferencing such a pointer is
if its value is outside the kernel's memory.

In order to avoid dangerous cases, we have to dynamically check that the
destination of every user-controlled pointer is in userspace. Kernel pointers
(that is to say, kernel-controlled pointers) can be dereferenced without further
check, but user pointers have to be manipulated with a restricted interface that
will check whether their destination is in userspace.

As mentioned before, this is done using the following constructs:%
{\small
\begin{mathpar}
    \disprule{GetU}

    \disprule{PutU}
\end{mathpar}}%
To add qualifiers to a type system, the rules of interest are those that
manipulate pointers: dereferencing, pointer arithmetic and referencing (taking
the address of a left-value).

Dereferencing the easiest one ; our goal is to authorize dereferencing only
\qKernel pointers:

{\small \[ \disprule{Lv-Deref} \]}%

Pointer arithmetic can be done inside a \qUser or \qKernel memory zone. There is
no concern of jumping from userspace to kernelspace, because pointer arithmetic
is checked at runtime: if these operators overflow or are applied to a bad
pointer (such as a pointer to an integer field), $\serr{ptr}$ is raised.

{\small \[ \disprule{Ptr-Arith} \]}%

The reference case is trickier because a type qualifier has to be synthesized.
Because it is created on the kernel stack, it has a \qKernel qualifier in all
cases:
{\small \[
  \disprule{Addr}
\]}%

The $\eTaint{\cdot}$ operator turns a \qUser pointer into a \qKernel pointer.
It is an important rule, because it is the only source of \qUser pointers in the
type system.
%{\small \[ \disprule{Taint} \] }% TODO
The return value of a function is emulated with a virtual left-value $\vRet$.%
{\small \begin{mathpar}
  \disprule{Return}

  \disprule{Fun}
\end{mathpar}}%

Toplevel sentences are typed one after another. Because they can extend an
environment, the general form of toplevel typing is $\typh{Γ}{p}{Γ'}$.

\begin{mathpar}
    \disprule{T-Exp}

    \disprule{T-Var}
\end{mathpar}

The whole set of typing rules can be found in annex in Figures~\ref{fig:typ-exp}
and~\ref{fig:typ-inst}.

\section{Type Safety}
\label{sec:safety}

This type system is sound with respect to the semantics defined in
Section~\ref{sec:eval}; that is, we have the desired property that
\emph{well-typed programs cannot go wrong}.

Let us start by defining a semantic typing relation between typing environments,
memory states, variable names and types.

Variables in memory have store types $τ$, depending only on a memory state $m$.
They are very similar to the previous types $t$, but the arrow function type is
replaced by $\stFun{n}$, denoting a $n$-ary function. This is because at runtime,
no type information for functions is left.

We define a relation $m ⊧ x : τ$ to express that $x$ is semantically typed $τ$
under $m$ (Figure~\ref{fig:st}). Constants are easy to type as there is
enough information to recognize them; functions have a simple dynamic type, only
retaining the arity. For memory references, anything that can be dereferenced to
a $τ$ is a $τ~\qKernel~*$; and user memory references are ``untainted'' to
access the underlying storage.
\begin{figure}[hbt]
  \centering
  \small
\begin{mathpar}
  \irule{S-Int}{ }{m ⊧ n : \tInt}

  \irule{S-Float}{ }{m ⊧ d : \tFloat}

  \irule{S-Unit}{ }{m ⊧ \eUnit : \tUnit}

  \irule{S-Ref} { m ⊧ m [φ]_Φ : τ} { \widehat{\&}~φ : τ~\qKernel~*}

  \irule{S-Null}{ }{m ⊧ \eNull : t~q~*}

  \irule{S-Tainted}
    { m ⊧           φ'  : t~\qKernel~* }
    { m ⊧ \vTainted{φ'} : t~\qUser~* }

  \irule{S-Fun}{ } {m ⊧ \mathrm{fun} (x_1, …, x_n) \{ i \} : \stFun{n}}

  \irule{S-Array}
    { ∀i ∈ [1;n]. m ⊧ v_i : t }
    { m ⊧ \widehat{ [ v_1;…;v_n ] } : t[~] }

  \irule{S-Struct}
    { ∀i ∈ [1;n]. m ⊧ v_i : t_i
   \\ S = \{ l_1:t_1;…;l_n:t_n \}
    }
    { m ⊧ \widehat{ \{ l_1:v_1;…;l_n:v_n \} } : S }
\end{mathpar}

\caption{Semantic typing rules}
\label{fig:st}
\end{figure}%

We also need a $\tComp{\cdot}{\cdot}$ relation to express compatibility between
types of expressions and semantic types (of values): ground types are
compatible with themselves, type constructors are compatible if their components
are compatible, and we add a rule for functions:%
\[ \tComp{\stFun{n}}{(t_1, …, t_n) → t} \]

Then a memory state $m$ is well-typed under a typing environment $Γ$, which we
will write $Γ ⊧ m$, if the semantic types of visible variable coincide with
their static types
\footnote{
  $\mathrm{dom}(Γ)$ returns the set of variables present in
  $Γ$, which also corresponds to visible variables.
}
:%
{ \small \[
  ∀ x ∈ \mathrm{dom}(Γ). ∃ τ,t.
  \begin{cases}
      \tComp{τ}{t} \\
      m ⊧ x : τ \\
      Γ ⊢ x : t
  \end{cases}
\] }%

\begin{theorem}[Progress]

  Let us suppose that $Γ ⊢ e : t$.

  Then let $m$ be a memory state such that
  $Γ ⊧ m$.

  Then one of the following cases holds:

\begin{itemize}
  \item $∃ v ≠ Ω, e = v$
  \item $∃ (e', m'), Γ ⊧ m' ∧ \mm{m}{e}{m'}{e'}$
  \item $∃ Ω ∈ \{\serr{div},\serr{array},\serr{ptr}\}, \msi{m}{e} → Ω$
\end{itemize}

  That is, either:

\begin{itemize}
  \item $e$ is fully evaluated
  \item an evaluation step is possible, that preserves memory compatibility
  \item a division, array or pointer error occurs
\end{itemize}
\end{theorem}

\begin{theorem}[Preservation]

  If an expression is well-typed, then an evaluation step does not modify its
  type:

  If $Γ ⊢ e : t$, $Γ ⊧ m$ and $\mm{m}{e}{m'}{e'}$, then $Γ ⊢ e' : t$.
\end{theorem}

This proves that no term stays stuck, and that that typing is well-behaved we
respect to the semantics. It means that the type is really a contract between
expressions and functions: if their evaluation converge, then a value of the
inferred type will be produced.

\section{Implementation}
\label{sec:implem}

The language described above, as well as a type inference algorithm, have been
implemented in OCaml as part of the Newspeak framework of program
analysis\cite{newspeak}. It is released under the GNU Lesser General Public
License, and is available on \texttt{http://penjili.org} (directory
\texttt{src/ptrtype} in the distribution). Our implementation consists of the
following steps.

The first one consists in preprocessing the C files that compose the kernel. We
have to manually define macros that are usually defined by the kernel's build
system. The configuration system for Linux uses \texttt{cpp} macros too (named
\texttt{CONFIG\_*}), so we have make a choice of configuration file. We just
used the default one, but to analyze larger parts of the kernel, it may be
necessary to define a ``maximal'' configuration file (which is impossible
because of incompatibilities between some options).

Once we have a set of preprocessed C files\footnote{
  As \texttt{cpp}'s \texttt{\#include} statement is textual, headers get
  included in every file, leading to a lot of duplicate work for the next
  passes.
},
we run it through the \texttt{c2newspeak} utility that compiles it to the
Newspeak. In that step, types and
names are resolved, and the program is decorated with enough annotation to make
the next steps context-free.

Because Linux is written in the GNU C dialect, it required to adapt
\texttt{c2newspeak}. Two kinds of extensions were encountered: the main
particularity is the \texttt{\_\_attribute\_\_} notation that
decorates declarations. In the vast majority of cases, they can be ignored as
they are either compiler hints (e.g. \texttt{used} disables the ``variable is
not used'' warning), optimizations (e.g. \texttt{hot} groups variable together
so they are put in cache together) or too low-level to be taken in account (e.g.
\texttt{aligned(n)} specifies that a variable needs to be aligned on \texttt{n}
bits). Besides attributes, GNU C allows advanced uses of \texttt{typeof}, as
well as statement-expressions (expressions of the form \texttt{(\{stmt\})}), and
first-class labels (\texttt{\&\&lbl} and \texttt{goto *p} constructs) that had
to be properly supported.

Newspeak is designed for static analysis by abstract interpretation and has
quite a low-level view on programs. For example, it does not distinguish between
field access and indexed access: as in an assembly language, it represents both
of them with a numeric offset (static or dynamic) from the start of a memory
zone. To remove such an ambiguity, we have to hook into \texttt{c2newspeak}'s
internal data structures, where this information is still available, and with it
we can emit \langname code.

Then, different files are linked together. This step consists in ensuring that
the assumptions made by the different files are consistent. \texttt{static}
variables, not visible outside the file they are defined in, are renamed so that
they have a unique name.

Finally, the implementation of an inference algorithm for the above type system
is quite straightforward. As it is similar enough to simply typed lambda
calculus, we can use a variant of Damas and Milner's Algorithm
W\cite{DamasMilner}. As usual, instead of explicit substitutions we rely on
sharing of references to implement unification. As our type system is
monomorphic, we raise an error whenever free type variables occur.

At the end of this step, we have either a fully annotated program, or a type
error.

Let us run the analysis on a small example:

\begin{verbatim}
int f(int *x) { return (*x + 1); }
\end{verbatim}

Running our analyzer outputs a fully annotated program:

\begin{verbatim}
 % ptrtype example.c
f : (Ptr (Int)) -> (Int)
Int (example.c:1#4)^f(Ptr (Int) x) {
  (.c:3#4)^!return =(int32)
    (coerce[-2147483648,2147483647]
      ( ( ([(x_Ptr (Int) : Ptr (Int))]32_Int
            : Int
          )
          + (1 : Int)
        ) : Int
      ) : Int
    );
}
\end{verbatim}

The \texttt{coerce[a,b]} operator is a Newspeak artifact to detect integer
overflows when doing a value analysis by abstract interpretation. For the
purposes of this paper, it can be seen as the identity function typed
$(\tInt) → \tInt$.

\emph{A contrario}, when it is not possible to type check, the analyzer bails
out with an error.

\begin{verbatim}
void f(int *p) {
    /*!npk userptr p */
    *p = 3;
}
\end{verbatim}

The \texttt{/*!npk userptr p */} comment is interpreted by the analyzer and
causes it to unify the type of \texttt{p} with $t~\qUser~*$, effectively
forcing its qualifier.

\begin{verbatim}
04-addrof.c:4#4 - Cannot unify qualifiers:
  Kernel
  User
\end{verbatim}

We just signal the location where a unification failed, but it is possible to
keep track of all the unification sites to create a trace, easier to debug.

\section{Experimentation}
\label{sec:exp}

We now describe the case of a linux kernel video driver that involved a
user/kernel pointer bug. It is listed on the \url{http://freedesktop.org} bug
tracker as \#29340.

To switch graphic modes, GPU drivers can support Kernel Mode Setting (KMS). To
configure a device, the user has to communicate with the kernel driver with a
mechanism known as \emph{ioctls} Named after the function \texttt{ioctl()} for
Input/Output Control, they are a sort of device-specific system call mechanism.
As such, these class of functions manipulate user-supplied pointers and need to
be written with care.

The following code is part of the KMS driver for AMD Radeon devices:

\begin{verbatim}
/* from drivers/gpu/drm/radeon/radeon_kms.c */
int radeon_info_ioctl(struct drm_device *dev,
                      void *data,
                      struct drm_file *filp) {
	struct radeon_device *rdev =
            dev->dev_private;
	struct drm_radeon_info *info;
	struct radeon_mode_info *minfo =
            &rdev->mode_info;
	uint32_t *value_ptr;
	uint32_t value;
	struct drm_crtc *crtc;
	int i, found;

	info = data;
	value_ptr = (uint32_t *)
            ((unsigned long)info->value);
	value = *value_ptr;
        [...]
}
\end{verbatim}

We can see that the argument \texttt{data} is converted to a \texttt{struct
drm\_radeon\_info *}. A pointer \texttt{value\_ptr} is extracted from its
\texttt{value} field, and finally this pointer is dereferenced.

However, the argument \texttt{data} is a pointer to a (kernel-allocated)
structure of the following type, whose fields come from a userland
\texttt{ioctl} invocation.

\begin{verbatim}
/* from include/drm/radeon_drm.h */
struct drm_radeon_info {
  uint32_t    request;
  uint32_t    pad;
  uint64_t    value;
};
\end{verbatim}

To show that this is buggy, we annotated the function
\texttt{radeon\_info\_ioctl} so that its second parameter is a kernel pointer to
a structure containing a \qUser field \texttt{value}. This is possible because
we erase the types present in the C program before translation. So, an pointer
cannot be distinguished from a integer casted to a pointer. Using this, we
obtain a type error at line 16.

The good way to proceed was recorded as commit number \texttt{d8ab3557} by
replacing line 16 by:

\begin{verbatim}
if (DRM_COPY_FROM_USER(&value, value_ptr,
                       sizeof(value)))
        return -EFAULT;
\end{verbatim}

(\texttt{DRM\_COPY\_FROM\_USER} being a simple wrapper macro for
\texttt{copy\_from\_user}). In that case, we obtain no type error.

\section{Limitations}
\label{sec:limit}

Our system has several limitations.

First, as we used static typing, it is impossible to capture a lot of constructs
that are possible or even idiomatic in C: unions, type casts, type punning. The
first two ones are equivalent constructs. We can replace each cast from a type
$t_1$ to a type $t_2$ by the call of a function $\mathrm{cast}_{t_1,t_2}$ but
this means adding a flaw in the type system: it would be typed $(t_1) → t_2$,
in other words the dreaded type $α → β$ of \texttt{Obj.magic} in OCaml
or \texttt{unsafeCoerce} in Haskell.

Type punning consists in modifying the bit pattern of some data to manipulate it
in an efficient way. For example, it is common to define a set of macros to
access low-level parts of IEEE754 floats, for exemple their exponent. Usually, a
combination of unions and bit masks is used. In such cases, strong typing is of
course impossible. To handle this, we could wrap the manipulation in a macro and
add an explicit type information such as $\texttt{float\_exponent} : (\tFloat) →
\tInt$.

The operational semantics uses runtime support for several cases. Contrary to C,
we dynamically check for buffer overflows and \eNull dereference, but this is
not critical to our approach: if we suppose that the programs we analyze are
correct with respect to these properties, we could provide a lower-level
implementation without those checks.

A more concerning place where we add a runtime behaviour is when the memory is
searched for dangling pointers at every function return (with the
$\mathrm{Cleanup(\cdot)}$ operator). Removing this check makes the analysis
unsound, because it is then possible to make reference to the current stack
frame with a former one. However, contrary to type punning, there are few cases
where such a behaviour is desired. If we can have a static guarantee that the
address of a local variable will not be accessible after the function has
returned, we can remove the $\mathrm{Cleanup(\cdot)}$ step. This can be done
with a dedicated static analysis\cite{ifm10}.

Dynamic memory allocation, embodied in userspace by the \texttt{malloc()} and
\texttt{free()} functions, has not been implemented. Similarly to
\texttt{get\_\{from,to\}\_user()}, it manipulates data as memory zones, with a
pointer and a size in bytes. We would have to define a higher level operator
taking an expression and returning the address of a freshly allocated cell in a
heap. Then typing is straightforward:

{ \small
\begin{mathpar}
  \inferrule*
    { Γ ⊢ e : t }
    { Γ ⊢ \textsc{Alloc}(e) : t~\qKernel~* }

  \inferrule*
    { Γ ⊢ e : t~\qKernel~* }
    { Γ ⊢ \textsc{Free}(e) }
\end{mathpar}}%

To define evaluation, we can expand the memory states to include an explicit
heap of values, mapping unique labels to values. Deallocation is problematic
because we rely on the programmer not to access deallocated memory zones. Once
again, it is possible to dedicate an analysis like~\cite{ifm10} to enforce this.
It is also sound, albeit impractical, to ignore deallocation constructs.

Another limitation is that our system does not allow recursive structures,
containing a pointer to a field of the same type. For simple (not mutually)
recursive structures, it is possible to extend the grammar of type annotation to
allow the following, where $p$ binds a pointer type to the structure.

{\small \[
    S \gramisa … \gramorx \mathrm{fix}(p → S)
\]}%

\section{Conclusion}
\label{sec:concl}

We showed that type theory can be a useful tool for verifying the absence of
certain run-time properties. While adding static labels to variables seems to be
a crude approximation of reality, in some cases it has enough power to capture
real-world problems.

In this particular example, we work around C's lack of abstract types in order
to disallow dereference for a certain class of pointers, distinguished by
syntactic rules.

We defined an imperative language with an explicit stack, and described
operational semantics for it modelling that of the C programming language. It
includes a memory model that expresses the separation between user and kernel
spaces present in most operating systems. We added a type system that is sound
with respect to a property of isolation between this two memory spaces.

Finally, we demonstrate an implementation on this analysis on a bug that
affected the Linux kernel.

A first step towards making this analysis more practical is to demonstrate its
scalability by running it on larger fragments of the kernel.

There are also several places where we can improve significantly the
expressivity of our type system. For example, our current type system is only
monomorphic; but it would make sense to generalize free qualifier variables in
the type of global functions.

