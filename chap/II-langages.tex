Dans cette partie, nous allons présenter un langage impératif modélisant le
langage C. Le chapitre~\ref{cha:lang} décrit sa syntaxe, ainsi que sa
sémantique. À ce point, de nombreux programmes sont acceptés mais qui provoquent
des erreurs à l'exécution.

Afin de rejeter ces programmes incorrects, on définit ensuite dans le
chapitre~\ref{cha:typbase} une sémantique statique s'appuyant sur un système de
types simples. Des propriétés de sûreté de typage sont ensuite établies,
permettant de catégoriser l'ensemble des erreurs à l'exécution possibles.

Le chapitre~\ref{cha:qualifs} commence par étendre notre langage avec une
nouvelle classe d'erreurs à l'exécution, modélisant les accès à la mémoire
utilisateur catégorisé comme dangereux dans le chapitre~\ref{cha:os}. Une
extension au système de types du chapitre~\ref{cha:typbase} est ensuite établie,
et on prouve que les programmes ainsi typés ne peuvent pas atteindre ces cas
d'erreur.

\begin{center}
\begin{tabular}{ll|ccc}
\toprule
Langage          & Types     & E typage    & E sécurité  & E mémoire   \\
\langname        & sans      & \CheckedBox & N/A         & \CheckedBox \\
\langname        & simples   & \Square     & N/A         & \CheckedBox \\
\langname Kernel & simples   & \Square     & \CheckedBox & \CheckedBox \\
\langname Kernel & qualifiés & \Square     & \Square     & \CheckedBox \\
\bottomrule
\end{tabular}

% TODO ajouter une ligne horitontale sous la première ligne
\end{center}
