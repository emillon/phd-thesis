\long\def\partintro{
Dans cette partie, nous allons présenter un langage impératif modélisant le
langage C. Le chapitre~\ref{cha:lang} décrit sa syntaxe, ainsi que sa
sémantique. À ce point, de nombreux programmes acceptés peuvent provoquer
des erreurs à l'exécution.

Afin de rejeter ces programmes incorrects, on définit ensuite dans le
chapitre~\ref{cha:typbase} une sémantique statique s'appuyant sur un système de
types simples. Des propriétés de sûreté de typage sont ensuite établies,
permettant de catégoriser l'ensemble des erreurs à l'exécution possibles.

Le chapitre~\ref{cha:qualifs} commence par étendre notre langage avec une
nouvelle classe d'erreurs à l'exécution, modélisant les accès à la mémoire
utilisateur catégorisé comme dangereux dans le chapitre~\ref{cha:os}. Une
extension au système de types du chapitre~\ref{cha:typbase} est ensuite établie,
et on prouve que les programmes ainsi typés ne peuvent pas atteindre ces cas
d'erreur.

Trois types d'erreurs à l'exécution sont possibles:

\begin{itemize}
\item
les erreurs liées aux valeurs: lorsqu'on tente d'appliquer à une opération des
valeurs incompatibles (additionner un entier et une fonction par exemple).
L'accès à des variables qui n'existent pas rentre aussi dans cette catégorie.
\item
les erreurs mémoire, qui résultent d'un débordement de tableau, du
déréférencement d'un pointeur invalide ou d'arithmétique de pointeur invalide.
\item
les erreurs de sécurité, qui consistent en le déréférencement d'un pointeur dont
la valeur est contrôlée par l'espace utilisateur. Celles-ci sont uniquement
possibles en contexte noyau.
\end{itemize}

L'introduction des types simples enlève la possibilité de rencontrer le premier
cas. Il reste en revanche toujours possible de rencontrer des erreurs mémoire
(ainsi que des divisions par zéro). En présence d'extensions permettant de
manipuler des pointeurs utilisateurs, une extension naïve du système de types
introduit des erreurs de sécurité, qui sont empêchées par l'ajout de règles de
typage supplémentaires.
}
