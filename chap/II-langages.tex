\begin{headingpage}
Dans cette partie, nous allons présenter un langage impératif modélisant le
langage C. Le Chapitre~\ref{cha:lang} décrit sa syntaxe, ainsi que sa
sémantique. À ce point, de nombreux programmes sont acceptés mais qui provoquent
des erreurs à l'exécution.

Afin de rejeter ces programmes incorrects, on définit ensuite dans le
Chapitre~\ref{cha:typbase} une sémantique statique s'appuyant sur un système de
types simples. Des propriétés de sûreté de typage sont ensuite établies,
permettant de catégoriser l'ensemble des erreurs à l'exécution possibles.

Le Chapitre~\ref{cha:qualifs} commence par étendre notre langage avec une
nouvelle classe d'erreurs à l'exécution, modélisant les accès à la mémoire
utilisateur catégorisé comme dangereux dans le Chapitre~\ref{cha:os}. Une
extension au système de types du Chapitre~\ref{cha:typbase} est ensuite établie,
et on prouve que les programmes ainsi typés ne peuvent pas atteindre ces cas
d'erreur.

Trois types d'erreurs à l'exécution sont possibles:

\begin{itemize}
\item
  les erreurs de typage (dynamique), lorsqu'on tente d'appliquer à une
  opération des valeurs incompatibles (additionner un entier et une
  fonction par exemple).
\item
  les erreurs de sécurité, qui consistent en le déréférencement d'un
  pointeur dont la valeur est contrôlée par l'espace utilisateur.
  Celles-ci sont uniquement possibles en contexte noyau.
\item
  les erreurs mémoire, qui résultent d'un débordement de tableau, du
  déréférencement d'un pointeur invalide ou d'arithmétique de pointeur
  invalide.
\end{itemize}

En résumé, l'introduction des types simples enlève la possibilité de rencontrer
des erreurs de typage dynamique, et l'ajout des qualificateurs interdit les
erreurs de sécurité.

\begin{center}
\begin{tabular}{ll@{\hskip 15mm}ccc}
\toprule
\multirow{2}{*}{Langage} & \multirow{2}{*}{Types}  & \multicolumn{3}{c}{Erreurs possibles}   \\
                         &  & Typage      & Sécurité    & Mémoire     \\
\midrule
\langname{}       & sans      & \CheckedBox{} & N/A           & \CheckedBox{} \\
\langname{}       & simples   & \Square{}     & N/A           & \CheckedBox{} \\
\langname{} noyau & simples   & \Square{}     & \CheckedBox{} & \CheckedBox{} \\
\langname{} noyau & qualifiés & \Square{}     & \Square{}     & \CheckedBox{} \\
\bottomrule
\end{tabular}

% TODO ligne 3 s/simples/naifs/ ?

\end{center}
\end{headingpage}
