On rappelle ici l'ensemble des règles de typage décrites dans les
chapitres~\ref{cha:typbase} et~\ref{cha:qualifs}.

\section{Règles de typage des constantes et valeurs gauches}

\begin{framed}

\begin{mathpar}
    \ruleheader{$Γ ⊢ c : t$}

    \disprule{Cst-Int}

    \disprule{Cst-Float}

    \disprule{Cst-Null}

    \disprule{Cst-Unit}

    \ruleheader{$Γ ⊢ lv : t$}

    \disprule{Lv-Var}

    \disprule{Lv-Deref}

    \disprule{Lv-Index}

    \disprule{Lv-Field}
\end{mathpar}
\end{framed}

\newpage
\section{Règles de typage des opérateurs}
\begin{framed}
\begin{mathpar}
    \ruleheader{$Γ ⊢ \opun~e : t$}

    \disprule{Unop-Plus-Int}

    \disprule{Unop-Plus-Float}

    \disprule{Unop-Minus-Int}

    \disprule{Unop-Minus-Float}

    \disprule{Unop-Not}

    \ruleheader{$Γ ⊢ e_1~\opbin~e_2 : t$}

    \disprule{Op-Int}

    \disprule{Op-Float}

    \disprule{Op-Eq}

    \disprule{Ptr-Arith}

    \ruleheader{$\textsc{Eq}(t)$}

    \disprule{Eq-Num}

    \disprule{Eq-Ptr}

    \disprule{Eq-Array}

    \disprule{Eq-Struct}
\end{mathpar}
\end{framed}

\newpage
\section{Règles de typage des expressions et instructions}
\begin{framed}
\begin{mathpar}
    \ruleheader{$Γ ⊢ e : t$}

    \disprule{Addr}

    \disprule{Struct}

    \disprule{Call}

    \disprule{Set}

    \disprule{Array}

    \disprule{Fun}

    \ruleheader{$Γ ⊢ i$}

    \disprule{Pass}

    \disprule{Seq}

    \disprule{Exp}

    \disprule{Decl}

    \disprule{If}

    \disprule{While}

    \disprule{Return}

    \ruleheader{$\typh{Γ}{p}{Γ'}$}

    \disprule{T-Exp}

    \disprule{T-Var}
\end{mathpar}

\end{framed}

\newpage
\section{Règles de typage des valeurs}
\begin{framed}
  \hspace{-2.1em}
  \ruleheader{$\semtyp{m}{v}{τ}$}

  \begin{mathpar}
    \disprule{S-Int}

    \disprule{S-Float}

    \disprule{S-Unit}

    \disprule{S-Null}

    \disprule{S-Ptr}

    \disprule{S-Array}

    \disprule{S-Struct}

    \disprule{S-Fun}
  \end{mathpar}

\end{framed}

\section{Règles de typage des extensions noyau}
\begin{framed}

\begin{mathpar}
  \disprule{Addr-User}

  \disprule{GetU}

  \disprule{PutU}
\end{mathpar}

\end{framed}
