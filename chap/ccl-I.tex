% TODO

Cette thèse part de deux constats.

D'une part, le problème de manipulation sûre de pointeurs utilisateur est
éliminé si on interdit certaines opérations sur ces pointeurs, en en faisant un
type à part. En effet, la seule opération dangereuse est le déréférencement. Si
on interdit les déréférencements syntaxiques (opérateur \texttt{*}) et qu'on
restreint les cas nécessaires à des fonctions sûres comme
\texttt{copy\_to\_user}, on élimine les comportements dangereux.

D'autre part, le système de types de C est trop primitif pour pouvoir
garantir une véritable isolation entre deux types de même représentation. Le
fait d'introduir un nouveau nom pour un type avec \texttt{typedef} n'est qu'un
raccourci syntaxique, puisque le compilateur ne peut pas considérer un programme
sans avoir la définition complète des types (le seul cas où cela est possible
est pour manipuler des structures inconnues à travers des pointeurs).

Le but de cette thèse est donc de définir un langage intermédiaire proche de C,
mais bien typable, et de l'adjoindre d'un système de types tel que les
programmes bien typés manipulent les pointeurs utilisateur sans causer de
problèmes de sécurité.
