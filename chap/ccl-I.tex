
Nous avons montré que l'écriture de noyaux de systèmes d'exploitation nécessite
de manipuler des données provenant d'une zone non sûre, l'espace utilisateur.
Parmi ces données, il arrive de récupérer des pointeurs qui servent à passer des
données par référence à l'appelant, dans certains appels système. Si on
déréférence ces pointeurs sans vérifier qu'ils pointent bien vers une zone
mémoire également contrôlée par l'appelant, on risque de lire ou d'écrire dans
des zones mémoires réservées au noyau seul.

Nous proposons une technique de typage pour détecter ces cas dangereux. Pour ce
faire, il faudra tout d'abord définir un langage impératif bien typable que nous
appellerons \langname.
Celui-ci s'appuie sur le langage \newspeak, qui est un langage intermédiaire
développé par EADS dans le but de vérifier la sûreté de programmes C embarqués.
À ce titre, il existe un compilateur qui est capable de traduire du code C vers
\newspeak.

Définir la syntaxe et la sémantique de \langname permet d'écrire et d'évaluer
des programmes. Mais cela reste trop permissif, car on ne rejette pas les
programmes qui manipulent les données de manière incohérente. La première étape
est donc de définir un système de types pour classifier les expressions et
fonctions selon le type de valeurs que leur évaluation produit.

Une fois \langname défini et étendu d'un système de types, nous lui ajoutons des
constructions permettant d'écrire du code noyau, et en particulier on lui ajoute
des pointeurs utilisateur. il s'agit de pointeurs dont la valeur est contrôlée
par un utilisateur interagissant avec le programme via un appel système. Ces
pointeurs ont un type distinct des pointeurs habituels.

En résumé, le but de cette thèse est donc de définir un langage intermédiaire
proche de C, mais bien typable, et de lui adjoindre un système de types tel que
les programmes bien typés manipulent les pointeurs utilisateur sans causer de
problèmes de sécurité.

% TODO[E] phrase compliquée

% TODO[S] concordance des temps

% vim: spelllang=fr
