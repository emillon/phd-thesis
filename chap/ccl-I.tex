% TODO

Cette thèse part de deux constats.

D'une part, le problème de manipulation sûre de pointeurs utilisateur est
éliminé si on interdit certaines opérations sur ces pointeurs, en en faisant un
type à part. En effet, la seule opération dangereuse est le déréférencement. Si
on interdit les déréférencements syntaxiques (opérateur \texttt{*}) et qu'on
restreint les cas nécessaires à des fonctions sûres comme
\texttt{copy\_to\_user}, on élimine les comportements dangereux. Bien sûr, un
examen manuel exhaustif de tous les déréférencements est possible, mais trop
long et source d'erreurs.

D'autre part, le système de types de C est trop primitif pour pouvoir garantir
une véritable isolation entre deux types de même représentation : il n'y a pas
de types abstraits. Certes, \texttt{typedef} permet d'introduire un nouveau nom
pour un type, mais ce n'est qu'un raccourci syntaxique. Le compilateur ne peut
en effet pas considérer un programme sans avoir la définition quasi-complète des
types qui y apparaissent. La seule exception concerne les manipulations de
structures par pointeurs : si on ne fait que passer des pointeurs, il n'est pas
nécessaire de connaître la taille ni la disposition de la structure, donc il est
possible de ne pas connaître ces informations. Cette technique, connue sous le
nom de \emph{pointeurs opaques}, n'est pas applicable aux autres types.

Le but de cette thèse est donc de définir un langage intermédiaire proche de C,
mais bien typable, et de l'adjoindre d'un système de types tel que les
programmes bien typés manipulent les pointeurs utilisateur sans causer de
problèmes de sécurité.

% vim: spelllang=fr
