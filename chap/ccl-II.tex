On vient de décrire en détail un langage impératif, \langname: tout d'abord sa
syntaxe et sa sémantique d'évaluation dans le chapitre~\ref{cha:lang}. Une des
spécificités de cette sémantique est l'utilisation de lentilles pour modifier
les valeurs composées en profondeur.

\label{page:ccl-II-monad}
Il y a plusieurs alternatives à cette présentation. La première est la solution
classique qui consiste à décrire les modifications de la mémoire en extension.
C'est en général long et laborieux puisqu'il faut définir les accès en lecture
et écriture à chaque étape (avec des lentilles on décrit ces deux opérations
uniquement sur les briques du calcul, et la composition fait le reste). La
seconde solution est d'employer une sémantique monadique. Les transisitions sont
alors encodées comme des actions monadiques qui représentent les modifications
de la mémoire. Un des avantages de cette solution est qu'elle est très
extensible. Par exemple, la propagation des erreurs ou l'ajout de continuations
légères (c'est-à-dire le support des fonctions \texttt{setjmp} et
\texttt{longjmp}) peuvent facilement être exprimés dans un formalisme monadique.
Nous avons préféré une présentation classique qui reste plus accessible, et
suffisante compte tenu de la simplicité des constructions à interpréter dans le
langage.

% TODO[E] ↑ et pour les programmeurs C

Ensuite, dans le chapitre~\ref{cha:typbase} nous avons ajouté un système de
types à \langname. Le but est de restreindre le genre d'erreurs qui peuvent
arriver lors de l'évaluation d'un programme. Par le théorème de progrès
(théorème~\ref{thm:progres} , page~\pageref{thm:progres}), on interdit les
erreurs qui signalent une manipulation de valeurs incompatibles, l'accès à un
champ de structure inconnu, et l'accès à une variable inexistante. Et le
théorème de préservation (théorème~\ref{thm:preservation},
page~\pageref{thm:preservation}) formalise le résultat classique qu'une étape
d'évaluation ne modifier pas le typage.
Une particularité de \langname est que son état mémoire est structuré, avec une
pile de variables locales explicite. On retrouve donc cette distinction dans le
typage: les variables globales et les variable locales sont séparées dans les
environnements de typage $Γ$ (page~\pageref{page:gamma-split}).

Enfin, dans le chapitre~\ref{cha:qualifs}, on a étendu le langage pour exprimer
la notion de pointeurs utilisateur. Cela permet d'écrire des fonctions qui se
comportent comme des appels système. On a commencé par montrer qu'une extension
naïve du système de types ne suffit pas, car le théorème de progrès est alors
invalidé. On ajoute donc un type dédié aux pointeurs utilisateur. Les valeurs de
ce type sont créées explicitement et passées aux appels système. La règle de
typage du déréférencement est restreinte aux pointeurs noyau, ce qui permet de
ré-établir les théorèmes de progrès et préservation.

Notre technique de typage permet donc d'exprimer correctement les problèmes liés
à la manipulation mémoire lors des appels système, ainsi que décrits dans le
chapitre~\ref{cha:os}: c'est une méthode simple pour détecter et empêcher les
problèmes de sécurité qui proviennent des pointeurs utilisateur.

Comme nous l'avons fait remarquer dans le chapitre~\ref{cha:etatdelart},
utiliser une technique de typage pour étudier des propriétés sur les données a
déjà été explorée dans l'outil CQual~\cite{pldi99}, en particulier sur les
problèmes de pointeurs utilisateurs~\cite{cquk-usenix04}.

En effet, si on remplace \enquote{$\ptrK{t}$} par \enquote{$\qKernel~t~*$} et
\enquote{$\ptrU{t}$} par \enquote{$\qUser~t~*$}, on obtient un début de système
de types qualifiés.

En revanche, il y a une différence importante : CQual modifie fondamentalement
l'ensemble du système de types, pas \langname. Le jugement de typage de CQual a
pour forme générale $Γ ⊢ e : q~t$ (où $Γ$ est un environnement de typage, $e$
une expression, $q$ un qualificateur et $t$ un type), alors que le nôtre a la
forme plus classique $Γ ⊢ e : t$.

En intégrant $q$ à la relation de typage, on ajoute un qualificateur à chaque
type, même les expressions pour lesquels il n'est pas directement pertinent de
déterminer qui les contrôle (comme par exemple, un entier). Dans CQual, ceci
permet de traiter de manière correcte le transtypage. Par exemple, si $e$ a pour
type qualifié $\qUser~\tInt$, alors $(\tFloat~*)~e$ aura pour type qualifié
$\qUser~\tFloat~*$, et déréférencer cette expression produira une erreur de
typage. \langname, dans son état actuel, ne permet pas de traiter les
conversions de type et ne permet donc pas de traiter ce cas.

% TODO[E] n'est-ce pas possible en passant aussi au noyau le cast devient fct de
% conversion

Nous prenons, au contraire, l'approche de ne modifier le système de types que là
où cela est nécessaire, c'est-à-dire sur les types pointeurs. Cela permet de ne
pas avoir à modifier en profondeur un système de types existant.

Le modèle d'exécution est aussi très différent. CQual s'appuie sur un langage
proche de ML: un noyau de lambda-calcul avec des références. Le système de types
sous-jacent est proche de celui d'OCaml: du polymorphisme de premier ordre (avec
la restriction habituelle de généralisation des références) et du sous-typage
structurel. En outre, leur approche repose sur une gestion automatique de la
mémoire. De notre côté, nous nous appuyons sur un modèle mémoire plus proche de
C, reposant sur une pile de variables et des pointeurs manipulés à la main.

Une autre différence fondamentale est que le système de types de CQual fait
intervenir une relation de sous-typage. Le cas particulier du problème de
déréférencement des pointeurs utilisateur peut être traité dans ce cadre en
posant $\qKernel \preceq \qUser$ pour restreindre certaines opérations aux
pointeurs $\qKernel$.

Notre approche, au contraire, n'utilise pas de sous-typage, mais consiste à
définir un type abstrait $\ptrU{t}$ partageant certaines propriétés avec
$\ptrK{t}$ (comme la taille et la représentation) mais incompatible avec
certaines opérations. C'est à rapprocher des types abstraits dans les langages
comme OCaml et Haskell.

Les perspectives de travaux futurs sont également très différentes. Dans le cas
des pointeurs, même si le noyau Linux (et la plupart des systèmes
d'exploitation) ne comportent que deux espaces d'adressage, il est commun dans
les systèmes embarqués de manipuler des pointeurs provenant d'espaces mémoire
indépendants : par exemple, de la mémoire flash, de la RAM, ou une EEPROM de
configuration. Ces différentes mémoires possèdent des adresses, et un pointeur
est interprété comme faisant référence à une ou l'autre selon le code dont il
est tiré. Lorsqu'il y a plus de deux espaces mémoire, il devient impossible
d'encoder cette propriété dans un qualificateur, qui n'a que deux valeurs
possibles.

\added{Lorsqu'il...}

% vim: spelllang=fr
