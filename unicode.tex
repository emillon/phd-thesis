\DeclareUnicodeCharacter{00D7}{\times}            % ×
\DeclareUnicodeCharacter{0393}{\Gamma}            % Γ
\DeclareUnicodeCharacter{03A6}{\Phi}              % Φ
\DeclareUnicodeCharacter{03A9}{\Omega}            % Ω
\DeclareUnicodeCharacter{03B5}{\varepsilon}       % ε
\DeclareUnicodeCharacter{03BB}{\lambda}           % λ
\DeclareUnicodeCharacter{03BC}{\mu}               % μ
\DeclareUnicodeCharacter{03C3}{\sigma}            % σ
\DeclareUnicodeCharacter{03C4}{\tau}              % τ
\DeclareUnicodeCharacter{03C6}{\varphi}           % φ
\DeclareUnicodeCharacter{2010}{-}                 % ‐
\DeclareUnicodeCharacter{2023}{\vartriangleright} % ‣
\DeclareUnicodeCharacter{2026}{\ldots}            % …
\DeclareUnicodeCharacter{2115}{\mathbb{N}}        % ℕ
\DeclareUnicodeCharacter{2124}{\mathbb{Z}}        % ℤ
\DeclareUnicodeCharacter{2190}{\leftarrow}        % ←
\DeclareUnicodeCharacter{2191}{\uparrow}          % ↑
\DeclareUnicodeCharacter{2192}{\rightarrow}       % →
\DeclareUnicodeCharacter{219D}{\leadsto}          % ↝
\DeclareUnicodeCharacter{21A6}{\mapsto}           % ↦
\DeclareUnicodeCharacter{21D2}{\Rightarrow}       % ⇒
\DeclareUnicodeCharacter{2200}{\forall}           % ∀
\DeclareUnicodeCharacter{2203}{\exists}           % ∃
\DeclareUnicodeCharacter{2205}{\emptyset}         % ∅
\DeclareUnicodeCharacter{2208}{\in}               % ∈
\DeclareUnicodeCharacter{2209}{\notin}            % ∉
\DeclareUnicodeCharacter{2227}{\wedge}            % ∧
\DeclareUnicodeCharacter{222A}{\cup}              % ∪
\DeclareUnicodeCharacter{2260}{\neq}              % ≠
\DeclareUnicodeCharacter{2264}{\le}               % ≤
\DeclareUnicodeCharacter{2265}{\ge}               % ≥
\DeclareUnicodeCharacter{2286}{\subseteq}         % ⊆
\DeclareUnicodeCharacter{2295}{\oplus}            % ⊕
\DeclareUnicodeCharacter{22A2}{\vdash}            % ⊢
\DeclareUnicodeCharacter{22D8}{\lll}              % ⋘
\DeclareUnicodeCharacter{22D9}{\ggg}              % ⋙
