\DeclareUnicodeCharacter{00D7}{\times}        % ×
\DeclareUnicodeCharacter{0393}{\Gamma}        % Γ
\DeclareUnicodeCharacter{03B5}{\epsilon}      % ε
\DeclareUnicodeCharacter{03BB}{\lambda}       % λ
\DeclareUnicodeCharacter{03BC}{\mu}           % μ
\DeclareUnicodeCharacter{03C3}{\sigma}        % σ
\DeclareUnicodeCharacter{03C4}{\tau}          % τ
\DeclareUnicodeCharacter{2010}{-}             % ‐
\DeclareUnicodeCharacter{2026}{\ldots}        % …
\DeclareUnicodeCharacter{2190}{\leftarrow}    % ←
\DeclareUnicodeCharacter{2191}{\uparrow}      % ↑
\DeclareUnicodeCharacter{2192}{\rightarrow}   % →
\DeclareUnicodeCharacter{21A6}{\mapsto}       % ↦
\DeclareUnicodeCharacter{21D2}{\Rightarrow}   % ⇒
\DeclareUnicodeCharacter{2200}{\forall}       % ∀
\DeclareUnicodeCharacter{2205}{\emptyset}     % ∅
\DeclareUnicodeCharacter{2208}{\in}           % ∈
\DeclareUnicodeCharacter{222A}{\cup}          % ∪
\DeclareUnicodeCharacter{2260}{\neq}          % ≠
\DeclareUnicodeCharacter{2264}{\le}           % ≤
\DeclareUnicodeCharacter{2265}{\ge}           % ≥
\DeclareUnicodeCharacter{2286}{\subseteq}     % ⊆
\DeclareUnicodeCharacter{2295}{\oplus}        % ⊕
\DeclareUnicodeCharacter{22A2}{\vdash}        % ⊢
\DeclareUnicodeCharacter{22D8}{\lll}          % ⋘
\DeclareUnicodeCharacter{22D9}{\ggg}          % ⋙
