\documentclass{article}
\usepackage[utf8]{inputenc}

\usepackage{mathpartir}
\usepackage{syntax}

\newcommand{\irule}[3]{ \inferrule*[right=(#1)]{#2}{#3} }

\newcommand{\tyint} {\mathop{\mathrm{int}}}
\newcommand{\tybool}{\mathop{\mathrm{bool}}}

\newcommand{\nselect}[2]{ \mathop{\mathrm{select}}(#1, #2) }
\newcommand{\nguard} [1]{ \mathop{\mathrm{guard}}(#1) }
\newcommand{\nloop}  [1]{ \mathop{\mathrm{loop}}(#1) }
\newcommand{\npass}  [0]{ \mathop{\mathrm{pass}} }
\newcommand{\ndo}    [2]{ \mathop{\mathrm{do}} (#1) \mathop{\mathrm{with}} : #2 }
\newcommand{\ngoto}  [1]{ \mathop{\mathrm{goto}}(#1) }

\DeclareUnicodeCharacter{0393}{\Gamma}
\DeclareUnicodeCharacter{03B5}{\epsilon}
\DeclareUnicodeCharacter{03C3}{\sigma}
\DeclareUnicodeCharacter{2190}{\leftarrow}
\DeclareUnicodeCharacter{2191}{\uparrow}
\DeclareUnicodeCharacter{2192}{\rightarrow}
\DeclareUnicodeCharacter{21D2}{\Rightarrow}
\DeclareUnicodeCharacter{2205}{\emptyset}
\DeclareUnicodeCharacter{2208}{\in}
\DeclareUnicodeCharacter{2260}{\neq}
\DeclareUnicodeCharacter{2295}{\oplus}
\DeclareUnicodeCharacter{22A2}{\vdash}

\begin{document}

\section{Syntaxe}

\begin{grammar}
<programme> ::= ( fonctions , globales , <bloc> )

     <bloc> ::= <instr> ; <bloc>
           \alt $ε$

    <instr> ::= <lval> $←$ <expr>
           \alt <lval> $←$ funexp (args)
           \alt  $∅ ←$ funexp (args)
           \alt $↑$ nom \{ <bloc> \}
           \alt if (<expr>) \{ <bloc> \}
           \alt \{ <bloc> \} label:
           \alt goto label
           \alt forever \{ <bloc> \}

    <expr> ::= <lval>
          \alt unop <expr>
          \alt <expr> binop <expr>
          \alt cst
          \alt \& <lval>
          \alt \& fonction

    <lval> ::= var
          \alt <lval> . champ
          \alt <lval> [ <expr> ]
          \alt * <expr>

\end{grammar}

\section{Sémantique}

Sémantique concrète : un état est constitué d'un point de contrôle $p$ et d'un
état mémoire $σ$.

\subsection{Valeurs}

Les valeurs peuvent être :

\begin{itemize}
\item $n$ : un entier
\item $\&a$ : un pointeur vers l'adresse $a$
\item $\&f$ : un pointeur vers la fonction $f$
\end{itemize}

\subsection{Jugements}

Les jugements ont les formes suivantes :

\begin{itemize}

\item $σ ⊢ lv ⇒ a$ :

  la left-value lv correspond à l'adresse mémoire a.

\item $σ ⊢ e ⇒ v$ :

  l'expression e s'évalue en v.

\item $<l, instr, l'>$ :

  on peut passer du point l au point l' en effectuant
  l'instruction instr.

\item $<l, cond, l'>$ :

  on peut passer du point l au point l' si cond est vraie.

\item $(l, σ) → (l', σ')$ :

  permet de définit la fonction de transition principale

\end{itemize}

\subsection{Sémantique des left-values}

\[ \irule{Eval-Lv-Var}{
  (v, a) ∈ σ
}{
  σ ⊢ v ⇒ a
} \]

\[ \irule{Eval-Lv-Field}{
  σ ⊢ lv ⇒ a
}{
  σ ⊢ lv.f ⇒ a + f
} \]

\[ \irule{Eval-Lv-Array}{
  σ ⊢ lv ⇒ a \\
  σ ⊢ e ⇒ n \\
}{
  σ ⊢ lv[e] ⇒ a + n
} \]

\[ \irule{Eval-Lv-Deref}{
  σ ⊢ e ⇒ \&a
}{
  σ ⊢ *e ⇒ a
} \]

\subsection{Sémantique des expressions}

\[ \irule{Eval-Cst}{
}{
  σ ⊢ c ⇒ c
} \]

\[ \irule{Eval-Lv}{
  σ ⊢ lv ⇒ a \\
  (a, v) ∈ σ
}{
  σ ⊢ lv ⇒ v
} \]

\[ \irule{Eval-Unop}{
  σ ⊢ e ⇒ v
}{
  σ ⊢ op e ⇒ \widehat{op} v
} \]

\[ \irule{Eval-Binop}{
  σ ⊢ e_1 ⇒ v_1 \\
  σ ⊢ e_2 ⇒ v_2
}{
  σ ⊢ e_1 op e_2 ⇒ v_1 \widehat{op} v_2
} \]

\[ \irule{Eval-AddrOf}{
  σ ⊢ lv ⇒ a
}{
  σ ⊢ \&lv ⇒ \&a
} \]

\[ \irule{Eval-AddrOfFun}{
}{
  σ ⊢ \&f ⇒ \&f
} \]

\subsection{Sémantique des instructions}

\[ \irule{Instr-Assign}{
  \langle l, lv ← e, l' \rangle \\
  σ ⊢ lv ⇒ a \\
  σ ⊢ e ⇒ v
}{
  (l, σ) → (l', σ ⊕ \{ a → v \})
} \]

TODO fcalls

\[ \irule{Instr-Decl}{
  \langle l, ↑x \{ b \}, l' \rangle \\
  σ' = σ ⊕ \{ x → \top \} \\
  \langle l_b, b, l_b' \rangle \\
  (l_b, σ') → (l_b', σ'')
}{
  (l, σ) → (l', σ'' \backslash x)
} \]

\subsection{Sémantique des conditions}

\[ \irule{Cond}{
  \langle l, e, l' \rangle \\
  σ ⊢ e ⇒ n \\
  n ≠ 0
}{
  (l, σ) → (l', σ)
} \]

\section{Typage}



\end{document}
