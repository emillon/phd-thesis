\section{Conclusion}
% TODO 2è bug, comparaisons
% TODO comment convaincre les indus

\begin{frame}{Conclusion}
    \begin{itemize}
    \item Problème réel:
        \begin{itemize}
            \item Janvier 2014:
                \enquote{Linux 3.4+: arbitrary write with \texttt{CONFIG\_X86\_X32} (CVE-2014-0038)}
        \end{itemize}
    \item Difficultés à analyser du code noyau:
        \begin{itemize}
            \item structures de données complexes
            \item pas de \texttt{main()}
            \item difficile par interprétation abstraite
        \end{itemize}
    \end{itemize}
\end{frame}

\begin{frame}{Conclusion}
    \begin{itemize}
    \item Application du typage à la sûreté mémoire:
        \begin{itemize}
            \item délicat
            \item nécessite des restrictions
            \item mais fonctionne!
        \end{itemize}
    \item Approche par types abstraits:
        \begin{itemize}
            \item Correct pour empêcher certaines opérations
            \item Alternative: sous-typage (CQual)
        \end{itemize}
    \end{itemize}
\end{frame}

\begin{frame}{Contributions}
\begin{itemize}
\item Un langage impératif bien typé
\item Une sémantique basée sur les lentilles
\item Un système de types abstraits
\item Un prototype d'analyseur statique
\end{itemize}
\end{frame}

\begin{frame}{Perspectives}
    \begin{itemize}
        \item automatiser l'annotation
            \begin{itemize}
                \item Comment détecter les appels système?
                \item \texttt{SYSCALL\_DEFINEx}, …
                \item le reste: \texttt{ioctl}s, FS, …
            \end{itemize}
        \item Autres propriétés: ex \emph{bitmasks}
        \item Types abstraits
            \begin{itemize}
                \item extension: comment le type est représenté
                \item intention: à quoi il sert
                \item \texttt{int}, \texttt{int32\_t}
                \item \texttt{off\_t}, \texttt{size\_t}, \texttt{intptr\_t}
                \item Types \enquote{numéro de ligne}, \enquote{numéro de colonne}
            \end{itemize}
        \end{itemize}
\end{frame}
