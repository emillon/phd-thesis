\section{Conclusion}
  \begin{frame}<beamer>
   \tableofcontents[currentsection]
 \end{frame}

% TODO comment convaincre les indus

\begin{frame}{Contributions}
\begin{itemize}
\item \langname: un langage impératif bien typé
\item Une sémantique basée sur les lentilles
\item Un système de types abstraits
\item Un prototype d'analyseur statique
\end{itemize}
\end{frame}

\begin{frame}{Conclusion}
    \begin{itemize}
    \item Problème réel:
        \begin{itemize}
            \item bug de \texttt{ptrace} sur Blackfin
            \item Janvier 2014:
                \enquote{Linux 3.4+: arbitrary write with \texttt{CONFIG\_X86\_X32} (CVE-2014-0038)}
        \end{itemize}
    \item Application du typage à la sûreté mémoire:
        \begin{itemize}
            \item délicat
            \item nécessite des restrictions
            \item mais fonctionne!
        \end{itemize}
    \item Approche par types abstraits:
        \begin{itemize}
            \item Correct pour empêcher certaines opérations
            \item Alternative: sous-typage (CQual)
        \end{itemize}
    \end{itemize}
\end{frame}

\begin{frame}{Perspectives}
    \begin{itemize}
        \item Automatiser l'annotation: repérer les appels système
        \item Allocation dynamique
        \item Faire collaborer Penjili et \texttt{ptrtype}
        \item Autres propriétés:
            \begin{itemize}
                %\item \emph{bitmasks}: \texttt{!x \& y} $⇔$ \texttt{!(x \& y)}
                    %(cf manuscrit)
                \item différencier \texttt{size\_t} et \texttt{int}: \texttt{memset(p, 0, n)} $⇔$ \texttt{memset(p, n, 0)}
            \end{itemize}
        \item Types abstraits: permettre au programmeur de définir ses propres
            analyses
        \end{itemize}
\end{frame}
