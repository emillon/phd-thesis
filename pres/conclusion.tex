\section{Conclusion}
\subsection{Conclusion}

% 2è bug
% TODO comment convaincre les indus

\begin{frame}{Conclusion}
    \begin{itemize}
    \item Problème réel:
        \begin{itemize}
            \item Janvier 2014:
                \enquote{Linux 3.4+: arbitrary write with \texttt{CONFIG\_X86\_X32} (CVE-2014-0038)}
        \end{itemize}
    \item Difficultés à analyser du code noyau:
        \begin{itemize}
            \item structures de données complexes
            \item pas de \texttt{main()}
            \item difficile par interprétation abstraite
        \end{itemize}
    \end{itemize}
\end{frame}

\begin{frame}{Conclusion}
    \begin{itemize}
    \item Application du typage à la sûreté mémoire:
        \begin{itemize}
            \item délicat
            \item nécessite des restrictions
            \item mais fonctionne!
        \end{itemize}
    \item Approche par types abstraits:
        \begin{itemize}
            \item Correct pour empêcher certaines opérations
            \item Alternative: sous-typage (CQual)
        \end{itemize}
    \end{itemize}
\end{frame}

%\begin{frame}{Contributions}
%\begin{itemize}
%\item Un langage impératif bien typé
%\item Une sémantique basée sur les lentilles
%\item Un système de types abstraits
%\item Un prototype d'analyseur statique
%\end{itemize}
%\end{frame}

\begin{frame}{Perspectives}
    \begin{itemize}
        \item Automatiser l'annotation
            \begin{itemize}
                \item Comment détecter les appels système?
                \item \texttt{SYSCALL\_DEFINEx}, …
                \item le reste: \texttt{ioctl}s, FS, …
            \end{itemize}
        \item Autres propriétés:
            \begin{itemize}
                \item \emph{bitmasks}: \texttt{!x \& y} $⇔$ \texttt{!(x \& y)}
                \item \texttt{size\_t} / \texttt{int}: \texttt{memset(p, 0, n)} $⇔$ \texttt{memset(p, n, 0)}
            \end{itemize}
        \item Types abstraits: permettre au programmeur de définir ses propres
            analyses
        \end{itemize}
\end{frame}
