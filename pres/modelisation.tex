\section{Modélisation}

% TODO Remttre un coup sur le critique

\subsection{\langname}

% TODO titres à chaque slide

\begin{frame}{Syntaxe -- Expressions}
  \begin{align*}
  \gramdef{Expressions}{e}
                 { c               }{ Constante }
                 { \opun~e         }{ Opération unaire }
                 { e~\opbin~e      }{ Opération binaire }
                 { lv              }{ Accès mémoire }
                 { lv ← e          }{ Affectation }
                 { \& lv           }{ Pointeur }
                 { f               }{ Fonction }
                 { e (e_1, …, e_n) }{ Appel de fonction }
                 { \eStruct{
                    l_1 : e_1
                    ; …
                    ; l_n : e_n }  }{ Structure }
                 { \eArray{e_1 ;…; e_n} }{ Tableau }
                 {END}
  \end{align*}
\end{frame}

\begin{frame}{Syntaxe -- Instructions}
  \begin{align*}
  \gramdef{Instructions}{i}
                 { \iPass          }{Instruction vide}
                 { i;i             }{Séquence}
                 { e               }{Expression}
                 { \iDecl{x}{e}{i} }{Déclaration de variable}
                 { \iIf{e}{i}{i}   }{Alternative}
                 { \iWhile{e}{i}   }{Boucle}
                 { \iReturn{e}     }{Retour de fonction}
                 {END}
  \end{align*}
\end{frame}
\subsection{Sémantique}

\begin{frame}{Sémantique impérative}

\begin{itemize}
\item
  Système de transitions entre états
\item
  expression + mémoire
\item
  $\mm{m}{e}{m'}{e'}$
\item expressions $→$ valeurs
\end{itemize}

\end{frame}

\begin{frame}{Valeurs gauches}
    \begin{itemize}
        \item valeurs gauches (\emph{lvalues}): \texttt{t[20] = 1}
        \item valeurs gauches $→$ chemins $φ$
        \item formes:
            \begin{itemize}
                \item \texttt{x} $→$ \texttt{(n, x)}
                \item \texttt{lv.l} $→ φ.l$
                \item \texttt{lv[e]} $→ φ[n]$
                \item \texttt{*e} $→ φ$
            \end{itemize}
        \item $\mm{m}{lv ← e}{m'}{v}$ ?
        \item ex: $x.f[2] ← 3$
        \item Mise à jour complète : $m' = m[ x ← 0]$
        \item Mise à jour partielle : $m' = m[x.f ← 0]$
    \end{itemize}
\end{frame}

% TODO cibler l'ex sur x.f[2]

% TODO dessiner un état mémoire et des ronds imbriqués

\tikzset{ stackframe/.style={ draw
                            , minimum width=4.5cm
                            }
        }
\tikzset{ twovars/.style={ rectangle split
                         , rectangle split parts=2
                         }
        }

\long\def\figmemstate{
\begin{scope}
    [ start chain=going below
    , every join/.style={<-, thick}
    , node distance=5mm
    ]
    \node[stackframe, on chain, join] (stackbot) {$n ↦ 0$};
    \node[stackframe, on chain, join] {$n ↦ 1$};
    \node[stackframe, on chain, join] {$n ↦ 2$};
    \node[stackframe, on chain, join, twovars]
    (stackf) {\parbox{4.2cm}{ $ x ↦ \left\{
               \begin{array}{ll}
                 f &: \eArray{1;6;1;8}\\
                 g &: 3 \\
               \end{array}
   \right\} $ }
         \nodepart{two} $y ↦ 4$};
    \node[stackframe, on chain, join] (stacktop) {$z ↦ 2$};
\end{scope}
\node[stackframe, twovars, right=5cm of stackbot.north, anchor=north]
    (globvars)
    { $a ↦ 2$ \nodepart{two} $b ↦ 3.14$ };
}

\begin{frame}{Structure de la mémoire}
    \centering
    \begin{tikzpicture}
        \figmemstate{}
    \end{tikzpicture}
\end{frame}

\begin{frame}{Structure de la mémoire}

\begin{itemize}
\item $m = (s, g) ∈ \sMem$
\item $\sMem = \sList{\sFrame} × \sFrame$
\item liste de cadres de pile, cadre de globales
\item cadre: $c = (x_1 ↦ v_1, …, x_n ↦ v_n) ∈ \sFrame$
\item Comment exprimer:
  \begin{itemize}
  \item $m[φ]$
  \item $m[φ←v]$
  \end{itemize}
\end{itemize}
\end{frame}

\begin{frame}{À la main}

\begin{align*}
(s, g)[\texttt{Local}~(2, x).f[2] ← 3] &= (s', g) \\
                   s'_i &= \begin{cases}
                             s_i & \mbox{ si } i ≠ 2\\
                             c'  & \mbox{ sinon} \\
                           \end{cases}\\
                         c &= s_{2} \mbox{ où } c_i &= (x_i, v_i) \\
                   c'_i &= (x_i, v'_i) \\
                   v'_i &= \begin{cases}
                              v_i &  \mbox{ si } i ≠ 1 \\
                              3   &  \mbox{ sinon }\\
                           \end{cases} \\
                    … & …
\end{align*}
\end{frame}

\subsection{Lentilles}

\begin{frame}{Lentilles}

\begin{itemize}
\item
  Relation entre objet et sous-objet
\item
  $ℒ ∈ \setLens{R}{A}$
\item
  $ℒ = \begin{cases}           \mathrm{get}_ℒ : R → A \\           \mathrm{put}_ℒ : (A × R) → R \\          \end{cases}$
\item
  $\lensGet{ℒ}{\lensNodeBig{green!30,ultra thick}{blue!30, ultra thick}} = \lensInner{blue!30, ultra thick}$
\item
  $\lensPut{ℒ}{\lensInner{red!30, ultra thick}}{\lensNodeBig{green!30, ultra thick}{blue!30, ultra thick}} = \lensNodeBig{green!30, ultra thick}{red!30, ultra thick}$
\end{itemize}

\end{frame}

\begin{frame}{Composition de lentilles}

\begin{center}
\definecolor{figcola}{HTML}{EFEF52}
\definecolor{figcolb}{HTML}{52EFA1}
\definecolor{figcolc}{HTML}{52A1EF}
\definecolor{figcold}{HTML}{EFA152}
\begin{tikzpicture}
[node distance=2.5cm
,bignode/.style={draw,shape=rectangle,minimum size=1.5cm,fill=figcola}
,smallnode/.style={draw,shape=circle,minimum size=1cm,fill=figcolb}
,trinode/.style={draw,shape=regular polygon,regular polygon sides=3}
,>=triangle 45
,ultra thick
]
\node[bignode] (A) {};
\node[right of=A,smallnode] (B) {};
\node[right of=B,trinode,fill=figcolc] (C) {};
\node[below of=B,smallnode] (D) {};
\node[below of=A,node distance=5cm,bignode] (E) {};

\draw[->] (A) to node[auto] {$\mathrm{get}_{ℒ_1}$} (B);
\draw[->] (B) to (D);
\draw[->] (A) to (E);
\draw (C) |- node[auto,near end,swap] {$\mathrm{put}_{ℒ_2}$} ($(B)!0.5!(D)$);
\draw (D) |- node[auto,near end,swap] {$\mathrm{put}_{ℒ_1}$} ($(A)!0.75!(E)$);

\node[smallnode] at (A) {};
\node[trinode,fill=figcold] at (A) {};

\node[trinode,fill=figcold] at (B) {};

\node[trinode,fill=figcolc] at (D) {};

\node[smallnode] at (E) {};
\node[trinode,fill=figcolc] at (E) {};

\node[left of=A, node distance=4cm,bignode] (GA) {};
\node[below of=GA,smallnode] (GB) {};
\node[below of=GB,trinode,fill=figcold] (GC) {};

\node[smallnode] at (GA) {};

\node[trinode, fill=figcold] at (GA) {};
\node[trinode, fill=figcold] at (GB) {};

\draw [->] (GA) to node[auto] {$\mathrm{get}_{ℒ_1}$} (GB);
\draw [->] (GB) to node[auto] {$\mathrm{get}_{ℒ_2}$} (GC);

\end{tikzpicture}

\vspace{-1cm}\hspace{7cm} \fbox{$ℒ_1 \ggg ℒ_2$}

\end{center}

\end{frame}

\begin{frame}{Lentilles en sémantique}

    \begin{align*}
       Φ &∈ \sPath → \setLens{Mem}{Val} \\
        \\
       φ &= (2, x).f[2] \\
    Φ(φ) &= \texttt{fst} \ggg I(2) \ggg L(x) \ggg F(f) \ggg T(2) \\
        \\
      m' &= \lensPut{Φ(φ)}{v}{m} \\
         &\eqdef m[φ ← v]_Φ
    \end{align*}

\end{frame}

\begin{frame}{Application des lentilles}

    {\centering

    \begin{tikzpicture}
        \figmemstate{}

        \only<2->{
        \draw ($ (stackbot.north west) + (-3mm, 3mm) $)
    rectangle ($ (stacktop.south east) + (3mm, -3mm) $);
        }

        \only<3->{
        \draw ($ (stackf.north west) + (-3mm, 3mm) $)
    rectangle ($ (stackf.south east) + (3mm, -3mm) $);
        }

        \only<4->{
        \draw ($ (stackf.north west) + (-3mm, 3mm) $)
    rectangle ($ (stackf.text split east) + (3mm, -3mm) $);
        }

        \only<5->{
        \draw ($ (stackf.north west) + (22mm, -1mm) $)
    rectangle ($ (stackf.text east) + (-7.5mm, 0) $);
        }

        \only<6->{
        \draw ($ (stackf.north west) + (30.3mm, -1mm) $)
    rectangle ($ (stackf.text east) + (-12.3mm, 0) $);
        }

        \node[below=1cm of globvars] {
            \begin{minipage}{4cm}
            \begin{align*}
                Φ(φ) &= \only<2->{ \texttt{fst} } \\
         \only<3->{  &\ggg I(2)      \\}
         \only<4->{  &\ggg L(x)      \\}
         \only<5->{  &\ggg F(f)      \\}
         \only<6->{  &\ggg T(2)      \\}
            \end{align*}
            \end{minipage}
        };
    \end{tikzpicture}
    }
\end{frame}

\begin{frame}{Lentilles en sémantique}

\begin{itemize}
\itemsep1pt\parskip0pt\parsep0pt
\item
  par tableau

  \begin{itemize}
  \itemsep1pt\parskip0pt\parsep0pt
  \item
    $T(n) ∈ \setLens{Val}{Val}$
  \item
    $T(3) \equiv [1, 2, 3, \circled{5}]$
  \end{itemize}
\item
  par champ

  \begin{itemize}
  \itemsep1pt\parskip0pt\parsep0pt
  \item
    $F(l) ∈ \setLens{Val}{Val}$
  \item
    $F(y) \equiv \{ x: 0; y: \circled{-3} \}$
  \end{itemize}
\end{itemize}

\end{frame}

\begin{frame}{Exemple : par variable ($V$)}

\[V(v) ∈ \setLens{Mem}{Val}\]

\begin{itemize}
\itemsep1pt\parskip0pt\parsep0pt
\item
  Pour $m ∈ Mem$, $m = ((s_1, …, s_n), g)$
\end{itemize}

\begin{align*}
V(\texttt{Global}~x) &= \texttt{snd} \ggg L(x) \\
V(\texttt{Local}~(n, x)) &= \texttt{fst} \ggg I(n) \ggg L(x) \\
\end{align*}

\end{frame}

\begin{frame}{Exemple : par chemin ($Φ$)}

\[Φ(φ) ∈ \setLens{Mem}{Val}\]

\begin{align*}
    Φ(v)    &= V(v) \\
  Φ(φ.l)    &= Φ(φ) \ggg F(l) \\
  Φ(φ[i])   &= Φ(φ) \ggg T(i) \\
  \end{align*}

\end{frame}

\subsection{Système de types}

% TODO caractéristiques du système de types. Comparaison  CQUAL

% TODO figure statique/dynamique à mettre ici et en intro

% TODO grammaire des types

\begin{frame}{Deux types de pointeurs}

\begin{itemize}
\item séparer deux types de pointeurs
\item pointeurs noyau : $\ptrK{t}$

  \begin{itemize}
  \item valeur fixée à la compilation ou par le runtime
  \item ex: \texttt{\&x}, \texttt{malloc(n)}
  \item peuvent être déréférencés (\texttt{*p}, \texttt{memcpy()})
  \end{itemize}
\item pointeurs utilisateur : $\ptrU{t}$

  \begin{itemize}
  \item valeur provient d'un utilisateur non privilégié
  \item ex: paramètres d'appels systèmes
  \item doivent être vérifiés dynamiquement!
  \end{itemize}
\end{itemize}

\end{frame}

\begin{frame}{Règles de typage}
    \begin{mathpar}
       \irule{Addr}
         { Γ ⊢ lv : t }
         { Γ ⊢ \& lv : \colK{\ptrK{t}} }

       \irule{Lv-Deref}
         { Γ ⊢ \colK{e} : \colK{\ptrK{t}} }
         { Γ ⊢ \colK{*~e} : t }

       \irule{User-Get}
         { Γ ⊢ \colK{e_d} : \colK{\ptrK{t}}
        \\ Γ ⊢ \colU{e_s} : \colU{\ptrU{t}}
         }
         { Γ ⊢ \uGet{\colK{e_d}}{\colU{e_s}} : \tInt }

       \irule{User-Put}
         { Γ ⊢ \colU{e_d} : \colU{\ptrU{t}}
        \\ Γ ⊢ \colK{e_s} : \colK{\ptrK{t}}
         }
         { Γ ⊢ \uPut{\colU{e_d}}{\colK{e_s}} : \tInt }
    \end{mathpar}
\end{frame}

\begin{frame}{Sûreté du typage}

\begin{theorem}[Progrès]
  Supposons que $Γ ⊢ e : t$. Soit $m$ un état mémoire tel que $\mcomp{Γ}{m}$.
  Alors l'un des cas suivants est vrai:

\begin{itemize}
  \item $∃ v ≠ Ω, e = v$
  \item $∃ (e', m'), \mm{m}{e}{m'}{e'}$
  \item $∃ Ω ∈ \{\serr{div},\serr{array},\serr{ptr}\}, \msi{m}{e} → Ω$
\end{itemize}
\end{theorem}
\end{frame}

\begin{frame}{Sûreté du typage}

\begin{theorem}[Préservation]
    Si $Γ ⊢ e : t$ et $\mm{m}{e}{m'}{v}$,
    alors:

    $\mcomp{Γ}{\cCleanup{m'}}$ et $m' ⊧ v : τ$ où $\tComp{τ}{t}$.
\end{theorem}
\end{frame}
