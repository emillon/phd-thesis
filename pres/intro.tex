\section{Le problème des pointeurs}

\subsection{L'arnaque du serrurier}

\begin{frame}

\begin{itemize}
\item
  J'appelle un serrurier
\item
  Il peut m'ouvrir la porte
\item
  Et si je lui dis que j'habite chez mon voisin? à la banque?
\end{itemize}

\end{frame}
\begin{frame}

\begin{itemize}
\item problème de confiance
\item il ne doit pas appliquer ses droits…
\item …mais ceux de l'appelant
\item il doit vérifier que j'ai bien les droits sur la maison
\end{itemize}

\end{frame}

\subsection{Isolation entre utilisateur et noyau}

\begin{frame}{Espaces utilisateur et noyau}
    \begin{itemize}
        \item espace utilisateur: programmes (navigateur, …)
        \item espace noyau: pilotes
        \item matériel
    \end{itemize}
\end{frame}

% Memory zone
%
% #1 - start
% #2 - end
% #3 - color
\newcommand{\mzone}[3]{
  \path[#3] (#1,0) rectangle (#2,1);
}

% Address label
%
% #1 - x position
% #2 - text
\newcommand{\alabel}[2]{
  \path (#1,1) -- ++(0,0.3) node [pos=1] {\small \tt #2};
}
\newcommand{\addrlabel}[2]{
  \draw[<-] (#1,-0.1) -- ++(0,-0.3) node [auto,pos=1] {\small \tt #2};
  \draw[pattern=north east lines] (#1,0) rectangle ++(0.2,1);
}
\long\def\memzones{

  % exec
  \mzone{0.5}{1}{user}

  % lib
  \mzone{2.5}{3.2}{user}

  % stack
  \mzone{3.7}{4}{user}

  % stack
  \mzone{5}{5.5}{user}

  % kernel
  \mzone{6}{8}{kernel}

  % contour
  \draw (0,0) rectangle (8,1);

  \alabel{0}{0}
  \alabel{6}{3 Go}
  \alabel{8}{4 Go}

}

\begin{frame}[fragile]{Séparation}
    %TODO à dégager
\begin{tikzpicture}
  [user/.style={fill=MellonGreen}
  ,kernel/.style={fill=MellonPink}
  ]

  \memzones{}

  \node at (2, -0.7) {Programme};
  \node at (7, -0.7) {Noyau};

  \draw[>=triangle 45,<->,ultra thick] (3.8, -0.7) -- node[auto] {Appels système} ++(2, 0);
  \draw[>=triangle 45,<->,ultra thick] (7, -1) -- node[auto,swap] {Instructions
  privilégiées} ++(0, -1) node (Htop) {};

  \path (Htop) -- ++(-1,-1) node (Htopleft) {};
  \draw (Htopleft) rectangle ++(2, -2);

  \foreach \x in {0,...,9} {
      \draw ($ (Htopleft) + (0.1 + 0.2*\x,  0) $) -- ++(0,  0.3);
      \draw ($ (Htopleft) + (0.1 + 0.2*\x, -2) $) -- ++(0, -0.3);
      \draw ($ (Htopleft) + (0, -0.2*\x - 0.1) $) -- ++(-0.3, 0);
      \draw ($ (Htopleft) + (2, -0.2*\x - 0.1) $) -- ++( 0.3, 0);
   }

  \path (Htop) -- ++(0, -2) node { Matériel };

\end{tikzpicture}
\end{frame}

\begin{frame}
    \insertcode{gettimeofday.c}

\begin{tikzpicture}
  [user/.style={fill=MellonGreen}
  ,kernel/.style={fill=MellonPink}
  ]

  \memzones{}
  \addrlabel{2.7}{ptv}
\end{tikzpicture}
\end{frame}

\begin{frame}
    \insertcode{gettimeofday-bad.c}

\begin{tikzpicture}
  [user/.style={fill=MellonGreen}
  ,kernel/.style={fill=MellonPink}
  ]

  \memzones{}
  \addrlabel{6.7}{ptv}
\end{tikzpicture}
\end{frame}

% TODO exemple KMS à la place de gettimeofday

\begin{frame}
\begin{itemize}
    \item il faut vérifier tous les pointeurs
    \item fonctions \texttt{copy\_from\_user}, \texttt{copy\_to\_user}
    \item test + \texttt{memcpy}
    \item ou retournent une erreur ($< 0$)
    \item but: détecter les cas où on devrait utiliser \texttt{copy\_*\_user}
\end{itemize}
\end{frame}

% TODO ici: mettre un coup sur pourquoi c'est intéressant/ context

% TODO slide ctx/ partie avec NPK sur la chaîne
