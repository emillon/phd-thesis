\section*{Français}


\section*{English}

% TODO (PLAS suit)

Manipulating user-provided pointers in the kernel of an operating system can
lead to security flaws if done in an incautious manner. We present an efficient
system to detect and prevent this class of erroneous memory manipulation.

At the core of our approach is \langname, an imperative language that we equip
with a qualified type system, where two kinds of pointers are distinguished:
\emph{safe} pointers, whose value is statically proved to be controlled by the
kernel, and \emph{unsafe} ones, whose value comes from userspace through
run-time system calls. Dereferencing unsafe pointers is forbidden in a static
manner by the means of a strong type system.

A concrete case study is described based on a bug that affected a video driver
in the Linux kernel. We also explain a technique to automatically translate GNU
C code to our core language, which will enable us to analyze larger fractions of
the kernel in order to find similar vulnerabilities.
