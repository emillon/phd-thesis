\begin{headingpage}
\thispagestyle{empty}

\begin{center}\Large \textbf{Résumé}\end{center}

Lorem ipsum dolor sit amet, consectetuer adipiscing elit, sed diam nonummy nibh
euismod tincidunt ut laoreet dolore magna aliquam erat volutpat. Ut wisi enim ad
minim veniam, quis nostrud exerci tation ullamcorper suscipit lobortis nisl ut
aliquip ex ea commodo consequat. Duis autem vel eum iriure dolor in hendrerit in
vulputate velit esse molestie consequat, vel illum dolore eu feugiat nulla
facilisis at vero eros et accumsan et iusto odio dignissim qui blandit praesent
luptatum zzril delenit augue duis dolore te feugait nulla facilisi. Nam liber
tempor cum soluta nobis eleifend option congue nihil imperdiet doming id quod
mazim placerat facer possim assum. Typi non habent claritatem insitam; est usus
legentis in iis qui facit eorum claritatem. Investigationes demonstraverunt
lectores legere me lius quod ii legunt saepius. Claritas est etiam processus
dynamicus, qui sequitur mutationem consuetudium lectorum. Mirum est notare quam
littera gothica, quam nunc putamus parum claram, anteposuerit litterarum formas
humanitatis per seacula quarta decima et quinta decima. Eodem modo typi, qui
nunc nobis videntur parum clari, fiant sollemnes in futurum

%\begin{center}\Large \textbf{Abstract}\begin{center}

% TODO (PLAS suit)

%Manipulating user-provided pointers in the kernel of an operating system can
%lead to security flaws if done in an incautious manner. We present an efficient
%system to detect and prevent this class of erroneous memory manipulation.

%At the core of our approach is \langname, an imperative language that we equip
%with a qualified type system, where two kinds of pointers are distinguished:
%\emph{safe} pointers, whose value is statically proved to be controlled by the
%kernel, and \emph{unsafe} ones, whose value comes from userspace through
%run-time system calls. Dereferencing unsafe pointers is forbidden in a static
%manner by the means of a strong type system.

%A concrete case study is described based on a bug that affected a video driver
%in the Linux kernel. We also explain a technique to automatically translate GNU
%C code to our core language, which will enable us to analyze larger fractions of
%the kernel in order to find similar vulnerabilities.

\end{headingpage}
