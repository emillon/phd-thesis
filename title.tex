\usepackage{url}
\ifpdf
\usepackage{pdfcolmk}
\fi
%% check if using xelatex rather than pdflatex
\ifxetex
\usepackage{fontspec}
\fi
\usepackage{graphicx}
%%\usepackage{hyperref}
%% drawing package
\usepackage{tikz}
%% for dingbats
\usepackage{pifont}
\providecommand{\HUGE}{\Huge}% if not using memoir
\newlength{\drop}% for my convenience
%% specify the Webomints family
\newcommand*{\wb}[2]{\fontsize{#1}{#2}\usefont{U}{webo}{xl}{n}}
%% select a (FontSite) font by its font family ID
\newcommand*{\FSfont}[1]{\fontencoding{T1}\fontfamily{#1}\selectfont}
%% if you don’t have the FontSite fonts either \renewcommand*{\FSfont}[1]{}
%% or use your own choice of family.
%% select a (TeX Font) font by its font family ID
\newcommand*{\TXfont}[1]{\fontencoding{T1}\fontfamily{#1}\selectfont}
%% Generic publisher’s logo
\newcommand*{\plogo}{\fbox{$\mathcal{PL}$}}

%% Some shades
\definecolor{Dark}{gray}{0.2}
\definecolor{MedDark}{gray}{0.4}
\definecolor{Medium}{gray}{0.6}
\definecolor{Light}{gray}{0.8}
%%%% Additional font series macros
\makeatletter
%%%% light series
%% e.g., kernel doc, section s: line 12 or thereabouts
\DeclareRobustCommand\ltseries
{\not@math@alphabet\ltseries\relax
\fontseries\ltdefault\selectfont}
%% e.g., kernel doc, section t: line 32 or thereabouts
\newcommand{\ltdefault}{l}
%% e.g., kernel doc, section v: line 19 or thereabouts
\DeclareTextFontCommand{\textlt}{\ltseries}
% heavy(bold) series
\DeclareRobustCommand\hbseries
{\not@math@alphabet\hbseries\relax
\fontseries\hbdefault\selectfont}
\newcommand{\hbdefault}{hb}
\DeclareTextFontCommand{\texthb}{\hbseries}
\makeatother


\usetikzlibrary{fadings}

\newcommand*{\titleUL}{
  \newgeometry{left=2cm, right=2cm, top=2cm, bottom=2cm}%
  \begingroup% University of Liege
\drop=0.1\textheight
\begin{center}
{\textsc{Université Pierre et Marie Curie\\
École Doctorale Informatique, Télécommunications et Électronique
}}\\
\vspace{1cm}
\begin{tikzpicture}
  \begin{scope}
    \node (A) {\includegraphics[height=\textwidth]{curies.pdf}};
    \path[opacity=0.2, fill=white]
      (A.north west) rectangle (A.south east);
    %\path[opacity=0.8, fill=white, path fading=north, fading transform={rotate=-45}]
    %(A.north west) rectangle (A.south east);
    %\clip (A.center) circle(8.5cm);
    %\shade[opacity=0.2, shading=axis,bottom color=black!10,top color=black,shading angle=-45]
    %(A.south west) rectangle (A.north east);
  \end{scope}
\end{tikzpicture}
\vspace{-13cm}
\\[\drop]
\begin{minipage}{13cm}
  \begin{center}
\rule{\textwidth}{1pt}\par
\vspace{0.5\baselineskip}
{\huge\bfseries \thetitle\\
\vspace{5mm}
\large --- \thesubtitle ---
}\\[0.5\baselineskip]
\rule{\textwidth}{1pt}
  \end{center}
\end{minipage}
\par
\vspace{1cm}
{{\Large\textsc{\theauthor}} \\
sous la direction d'Emmanuel Chailloux et de Sarah Zennou
}
\vspace{16mm}\\
{{\LARGE\textsc{Thèse}}\\
pour obtenir le titre de\\
{\Large Docteur en Sciences}\\
{\large mention Informatique}
}
\vspace{2cm}\\
{\large Soutenue le \thedate{} devant un jury composé de\\

aaa\\
aaa\\
aaa\\
aaa\\
aaa
}
\end{center}
\endgroup
\restoregeometry
}

