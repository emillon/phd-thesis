% Fig memory zones (needed ?)
% Memory zone
%
% #1 - start
% #2 - end
% #3 - color
\newcommand{\mzone}[3]{
  \path[#3] (#1,0) rectangle (#2,1);
}

% Address label
%
% #1 - x position
% #2 - text
\newcommand{\alabel}[2]{
  \path (#1,1) -- ++(0,0.3) node [pos=1] {\small \ttfamily #2};

}

% Pointer
%
% #1 - start address
% #2 - label
\newcommand{\ptr}[2]{
  \path [draw,open triangle 60-] (#1, -0.2) -- ++ (0,-0.5) node [below] {#2};
}

\newcommand{\memoryframe}[1]{%
\begin{tikzpicture}
  [user/.style={fill=green!40}
  ,kernel/.style={fill=red!40}
  ]

  % exec
  \mzone{0.5}{1}{user}

  % lib
  \mzone{2.5}{3.2}{user}

  % stack
  \mzone{3.7}{4}{user}

  % stack
  \mzone{5}{5.5}{user}

  % kernel
  \mzone{6}{8}{kernel}

  % contour
  \draw (0,0) rectangle (8,1);

  \alabel{0}{0}
  \alabel{6}{3 Go}
  \alabel{8}{4 Go}

  % User struct
  \ifthenelse{#1=1}{\mzone{2.6}{3.1}{draw, fill=green}}{}
  \ifthenelse{#1=1}{\ptr{2.6}{pu}}{}

  \ifthenelse{#1=2}{\mzone{7}{7.5}{draw, fill=red}}{}
  \ifthenelse{#1>1}{\ptr{7}{pk}}{}

\end{tikzpicture}
} % End of newcommand \memoryframe

