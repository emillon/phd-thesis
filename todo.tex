\section{État de l'art}%{{{

% difficultés : récursion

%}}}

\section{Évaluateur} % {{{

\begin{itemize}
\item Exp-AddrOf sur toutes les lv
\item figures gramdef : singulier ou pluriel?
\end{itemize}

Limitations :

\begin{itemize}
\item tableaux de taille dynamique ?
\end{itemize}

\subsection*{Extrait PLAS}



% }}}

\section{Typage}%{{{

\subsection*{Fonctions}

\begin{itemize}
\item
  page 50 règle CALL une remarque disant que cette règle doit être
  utilisée avec une autre qui va typer le corps de la fonction (mettre
  la ref) parce que sinon ça surprend
\end{itemize}

\subsection*{Rq}
(passe Sarah 17/01)

\begin{itemize}
\item
  p53 le terme de dérivation (première phrase de la preuve) n'a jamais
  été expliqué ?
\item
  lemme 5.3 ça ne marche que si une variable n'apparaît qu'une fois dans
  le contexte ou avec toujours le même type (cf rem ci-dessous sur 5.3)
\item
  théormèe 5.2 rappeler où a été défini l'évaluation d'une expression et
  dans quel cas elle produit des valeurs
\end{itemize}%}}}

\section{Qualificateurs}

Sarah 13/02

\begin{itemize}
\item
\item
  6.2 pourquoi ne pas nommer cette extension de C\_ML
\end{itemize}

\section{Étude de cas}%{{{

Le paramètre \texttt{data} provient de l'espace utilisateur via un appel
système. Un appelant malveillant peut se servir de cette fonction pour lire la
mémoire du noyau à travers le message d'erreur.

Le problème est modélisé de la façon suivante : on associe à chaque variable
\texttt{x} un type de données \texttt{t}, ce que l'on note \texttt{x:t}. En
plus des types présents dans le langage C, on ajoute une distinction
supplémentaire pour les pointeurs. D'une part, les pointeurs «noyau» (de type
\texttt{t~*}) sont créés en prenant l'adresse d'un objet présent dans le code
source. D'autre part, les pointeurs «utilisateurs» (leur type est noté
\texttt{t user*}) proviennent des interfaces avec l'espace utilisateur.

Il est sûr de déréférencer un pointeur noyau, mais pas un pointeur
utilisateur. L'opérateur \texttt{*} prend donc un \texttt{t *} en entrée
et produit un \texttt{t}.

Pour faire la vérification de type sur le code du programme, on a besoin de
quelques règles. Tout d'abord, les types suivent le flot de données.
C'est-à-dire que si on trouve dans le code \texttt{a = b}, \texttt{a} et
\texttt{b} doivent avoir un type compatible. Ensuite, le qualificateur
\texttt{user} est récursif : si on a un pointeur utilisateur sur une structure,
tous les champs pointeurs de la structure sont également utilisateur. Enfin, le
déréférencement s'applique aux pointeurs noyau seulement : si le code contient
l'expression \texttt{*x}, alors il existe un type \texttt{t} tel que
\texttt{x:t*} et \texttt{*x:t}.

Appliquons ces règles à l'exemple de la figure~X : on suppose que l'interface
avec l'espace utilisateur a été correctement annotée. Cela permet de déduire que
\texttt{data:void user*}. En appliquant la première règle à la ligne 6, on en
déduit que \texttt{info:struct drm\_radeon\_info user*} (comme en C, on peut
toujours convertir de et vers un pointeur sur \texttt{void}).

Pour déduire le type de \texttt{value\_ptr} dans la ligne 7, c'est la
deuxième règle qu'il faut appliquer : le champ \texttt{value} de
la structure est de type \texttt{uint32\_t~*} mais on y accède à travers
un pointeur utilisateur, donc \texttt{value\_ptr:uint32\_t user*}.

% faux

À la ligne 8, on peut appliquer la troisième règle : à cause du déréférencement,
on en déduit que \texttt{value\_ptr:t *}, ce qui est une contradiction puisque
d'après les lignes précédentes, \texttt{value\_ptr:uint32\_t user*}.

Si la ligne 3 était remplacée par l'appel à \texttt{copy\_from\_user}, il n'y
aurait pas d'erreur de typage car cette fonction peut accepter les arguments
\texttt{(uint32\_t~*, uint32\_t user*, size\_t)}.

Le principe de cette technique (associer des types aux valeurs puis restreindre
les opérations sur certains types) peut être repris. Par exemple, si on définit
un type «numéro de bloc» comme étant un nouvel alias de \texttt{int}, on peut
considérer que multiplier deux telles valeurs est une erreur.%}}}

% Sarah 9/5

%chapeau de l'intro
%- manque de structures/liens entre les petits paragraphes.

%1.2
    %- tu pourrais ajouter un ex d'appel système en C pour expliquer le pb du déréférencement

%1.3
   %- typage fort et faible. On ne comprend pas la différence
   %- manquent des mots dans la dernière phrase de cette partie

%1.4
  %- pourquoi parler de C++ et Ada alors que tu ne vas traiter que le C ?

%1.5

%- il faudrait pour chaque front-end dire pourquoi ça n'est pas la solution retenue

%Pourquoi ne pas présenter informellement le cas d'étude dans l'intro
%pour montrer que de vrais bugs existent ?

% Emmanuel 6/5

  %Sur la première partie, je dirai que le point le plus important est de
  %justifier ta proposition à la fin du ch +apitre 3,
